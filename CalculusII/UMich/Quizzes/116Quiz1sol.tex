\documentclass[11pt,letterpaper]{article}
\usepackage{fullpage}
\usepackage{multicol}
\usepackage{amsmath}
\usepackage{amsfonts}
\usepackage{amssymb}
%\usepackage{mathabx}
%\usepackage{pstricks, pst-node, pst-plot}

\newcommand{\ds}{\displaystyle}
\newcommand{\bv}{\mathbf}
\newcommand{\lv}{\langle}
\newcommand{\rv}{\rangle}

\begin{document}
\flushleft
\begin{multicols}{2}


\begin{large}\textbf{Math 116 Quiz 1: $\oint$ 5.1-5.4, 6.1-6.2 \\
Tue 11 Sep 2012}\end{large}

\textbf{Name:  }\underline{\hspace{35ex}}

\vspace{.5in}

\end{multicols}

\pagestyle{empty}


\flushleft

You have 15 minutes to complete this quiz.  Eyes on your own paper and good luck!

\begin{enumerate}
\item  \textbf{Definitions/Concepts.} (3 pts) Write down the Fundamental Theorem of Calculus.  

If $f$ is continuous on the interval $[a,b]$ and $f(t)=F'(t)$, then
\[\int_a^bf(t)dt=F(b)-F(a).\]

\vspace{1.5pc}

\item \textbf{Questions/Problems.} 
\begin{enumerate}
\item (2 pts) Recall that when we want to estimate area under a curve for a function $f(t)$ over the interval $t\in[a,b]$ we can use a left-hand or right-hand approximation.  Let $n$ denote the number of equally-sized subdivsions we use to divde the interval $[a,b]$.  Then
\[\Delta t=\frac{b-a}{n}\]
and we can let $t_0=a,\;t_1=t_0+\Delta t,\;t_2=t_1+\Delta t$, etc.   

Suppose you have the data:
\begin{table}[h]
\centering
\begin{tabular}{|c|c|c|c|c|c|}
\hline
t & 0 & 4 & 8 & 12 & 16 \\ [0.5ex]
\hline 
f(t) & 25 & 23 & 22 & 20 & 17 \\ [0.5ex]
\hline
\end{tabular}
\caption{number of students awake after $t$ minutes into a boring lecture}
\end{table}

Use this data to fill in the missing information:
\[\begin{array}{| l l l l l |}
\hline
n=4 & & & & \\ [1ex]
\Delta t=4\hspace{0.5 cm} & & & & \\ [1ex]
a=0\hspace{0.5 cm} & b=16\hspace{0.5 cm} & & &\\ [2ex]
t_0=0\hspace{0.5 cm} & t_1=4\hspace{0.5 cm} & t_2=8\hspace{0.5 cm} & t_3=12\hspace{0.5 cm} & t_4=16\hspace{0.5 cm} \\ [2ex]
f(t_0)=25\hspace{0.5 cm} & f(t_1)=23\hspace{0.5 cm} & f(t_2)=22\hspace{0.5 cm} & f(t_3)=20\hspace{0.5 cm} & f(t_4)=17\hspace{0.5 cm} \\ [2ex]
\hline
 n=2 & & & & \\ [1ex]
\Delta t=8\hspace{1 cm} & & & & \\ [1ex]
a=0\hspace{0.5 cm} & b=16\hspace{0.5 cm} & & &\\ [2ex]
t_0=0\hspace{0.5 cm} & t_1=8\hspace{0.5 cm} & t_2=16\hspace{0.5 cm} &  &  \\ [2ex]
f(t_0)=25\hspace{0.5 cm} & f(t_1)=22\hspace{0.5 cm} & f(t_2)=17\hspace{0.5 cm} &  &  \\ [2ex]
\hline
\end{array}\]

\hfill{\bf MORE QUIZ ON THE BACK --\textgreater}
\vfill

\item (3 pts each) Write out the entire word, either True or False.  No justification is needed.
\begin{enumerate}
\item If $\int_0^2(3f(x)+1)dx=8$, then $\int_0^2f(x)dx=2$. \\
\vspace{0.5pc}
True.  Use the linearity property of integrals (Theorem 5.3 from the book) and algebraic manipulation to solve for $\int_0^2f(x)dx$:
\begin{align*}
\int_0^2(3f(x)+1)dx &= 8 \\
3\int_0^2f(x)dx+\int_0^21dx &=8 \\
3\int_0^2f(x)dx+\left.x\right|_0^2 &=8 \\
3\int_0^2f(x)dx &=6 \\
\int_0^2f(x)dx &=2
\end{align*}
%\item If $\int_a^bf(x)dx=2$ and $\int_a^bg(x)dx=-3$, then $\int_a^bf(x)g(x)dx=-6$.
%\vfill
\item If $f(x)=\int_{-2x}^0(1+t^4)dt$, then $f(x)$ is decreasing. \\  
\vspace{0.5pc}
False.  The bounds of integration are allowed to have a different variable from the one we're integrating over.  Any symbols other than $t$ we just treat like a constant while integrating:
\begin{align*}
f(x)=\int_{-2x}^0(1+t^4)dt &=\left.\left(t+\frac{t^5}{5}\right)\right|_{-2x}^0 \\
&= 0-\left((-2x)+\frac{(-2x)^5}{5}\right) \\
&= \frac{42}{5}x,
\end{align*}
a line with positive slope everywhere.

\item If $f(x)\leq g(x)$ for $x\in[0,1]$, then $\int_0^1f(x)dx\leq\int_0^1g(x)dx$. \\
\vspace{0.5pc}
True.  Theorem 5.4 in the text says so if we put $a=0$ and $b=1$.

\item If $g(x)$ is odd and $\int_1^3g(x)dx=2$, then $\int_{-3}^{-1}g(x)dx=2$. \\
\vspace{0.5pc}
False.  An odd function is symmetric about the origin, meaning the left of the vertical axis is a mirror image of the right, turned upside down.  The value of $g$ for $x\in[1,3]$ is negated for $-x\in[-3,-1]$.  That means $\int_{-3}^{-1}g(x)dx=-2$.   

\item If $f(t)$ is measured in dollars per year, and $t$ in measured in years, then $\int_a^bf(t)dt$ is measured in dollars per years squared. \\
\vspace{0.5pc}
False.  Think of the integral symbol as a sum, and $dt$ as a tiny change in $t$.  Then to see the units we can write:
\[\int_a^bf(t)dt=\text{``sum over $t\in[a,b]$ of }\left(f(t)\,\frac{\text{dollars}}{\text{year}}\right)\cdot\left(dt\text{ years}\right)"\]
The measurements in years cancel, so the integral is just measured in dollars.

\end{enumerate}

\end{enumerate}
\item \textbf{Computations/Algebra.} 

\it -none this week-

\end{enumerate}

\end{document}


