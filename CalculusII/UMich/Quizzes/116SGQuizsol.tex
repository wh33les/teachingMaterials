\documentclass[11pt,letterpaper]{article}
\usepackage{fullpage}
\usepackage{multicol}
\usepackage{amsmath}
\usepackage{amsfonts}
\usepackage{amssymb}
%\usepackage{pstricks, pst-node, pst-plot}

\newcommand{\ds}{\displaystyle}
\newcommand{\bv}{\mathbf}
\newcommand{\lv}{\langle}
\newcommand{\rv}{\rangle}

\begin{document}
\flushleft
\begin{multicols}{2}


\begin{large}\textbf{Math 116 Quiz 0: Student Guide \\
Fri 7 Sep 2012}\end{large}

\textbf{Name:  }\underline{\hspace{35ex}}

\vspace{.5in}

\end{multicols}

\pagestyle{empty}


\flushleft

Decide whether each of the statements below is \emph{True} or \emph{False}.  Write the entire word \emph{True} or \emph{False}.  If the statement is false, briefly explain why.  


\begin{enumerate}
\item  Math 116 is graded on a normal 10-point scale, that is 90-100=A, 80-90=B, etc.  

\vspace{.5pc}
False.  After each exam, a letter grade will be assigned to your uniform component score using a scale determined by the course coordinator specifically for that exam.  We do {\em not} use the ``10-point scale" often seen in high school courses in which scores in the 90's get an A, in the 80's get a B, and so forth; the level of difficulty of exams will be considered.

\item  In general, a student should expect to do well in this course if he/she completes all assignments, reads each section before class, and puts in 3-4 hours of study time a week.

\vspace{.5pc}
False.  The faculty at the University of Michigan expects you to study a minimum of two hours outside class for each credit hour, which means that we expect you to spend at least eight hours a week outside of class working on mathematics.

\item If a particular type of problem or concept is not discussed in class, then it will not be on the exam.

\vspace{.5pc}
False.  There will be times when you have to learn topics which will not be formally discussed in the classroom...  Also, your instructor will be counting on you to read the text, since he or she will not be lecturing very much and will be relying on you to have seen the material before you work with it in class.  Like other courses outside mathematics (but perhaps unlike other mathematics you have taken), not every small point on which you will be tested will be covered by in-class examples.

\item In order to emphasize working as a team, it is sometimes ideal for each member of a team homework group to submit a different problem in compiling the whole assignment for submission.

\vspace{.5pc}
False.  The scribe is responsible for writing up the single final version of the homework to be handed in.  This is the only set of solutions which will be accepted or graded... The reporter writes a record of how the homework session went, how long the team met, what difficulties or successes the team may have had (with math or otherwise).  If there is disagreement about the solution of a problem, the reporter should present sketches of alternate solutions and explain the difference of opinion.  The report should list the members of the team who attended the session and their roles.

\item If you do not participate in team homework and quizzes, your course grade will be based solely on your exam scores and web homework.

\vspace{.5pc}
False.  Students who have not seriously attempted to contribute to the section component of the course (i.e., quizzes, team homework, etc.) may have their final course grade lowered by up to a \emph{full} letter grade.

\item The section component of the course helps your instructor determine if your final grade may be altered by one third of a letter grade.  While there is the opportunity for your grade to be raised, you instructor is not allowed to use section component scores to justify lowering your final grade.

\vspace{.5pc}
False.  If you have participated in section activities but your section component is significantly lower than your uniform grade, your course grade may be lowered by one third of a letter grade.  Students who have not seriously attempted to contribute to the section component of the course (i.e., quizzes, team homework, etc.) may have their final course grade lowered by up to a {\em full} letter grade.

\item As long as you do well (i.e., above the median) in the section component according to the guidelines you instructor has set, you are likely to get the one third letter grade boost at the end of the term. 

\vspace{.5pc}
False.  If... you have struggled on an exam and your in-class performance is significantly higher than the uniform component grade, your instructor may in some cases adjust your grade upward by one third of a letter grade.  This raise is generally only given for students whose uniform component places them near the top (or at the ``cusp") of a letter grade category.  The majority of students will find that their in-class performance and their exam scores are quite reflective of one another.  Thus, in the majority of cases, no adjustment is made to the uniform course grade.

\item You can only take the online practice gateway exams twice a day, so it is important you start studying early.

\vspace{.5pc}
False.  You may practice each test online as many times as you like, and you may take a test for a score as often as twice per day without penalty until the deadline.

\item The gateway component is worth 5\% of your grade, so the scores you get make a difference, however small, in your grade. 

\vspace{.5pc} 
False.  Because the gateway tests cover skills that every student must have, the gateway tests do not raise your baseline grade; instead, if they are not passed by the deadline, your final grade in the course will be automaticaly reduced.

\item  At the end of the semester, every section of Math 116 will have the same number of A's, B's, etc.

\vspace{.5pc}
Course policy is that a section's average final letter grade cannot differ too much from that section's average baseline letter grades.  This means that the better your entire section does on the uniform exams, the higher average letter grade your instructor can assign in your section.

\end{enumerate}
\end{document}


