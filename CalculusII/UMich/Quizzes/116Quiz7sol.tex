\documentclass[11pt,letterpaper]{article}
\usepackage{fullpage}
\usepackage{multicol}
\usepackage{amsmath}
\usepackage{amsfonts}
\usepackage{amssymb}
%\usepackage{pstricks, pst-node, pst-plot}

\ifx\pdfoutput\undefined
% we are running LaTeX, not pdflatex
\usepackage{graphicx}
\else
% we are running pdflatex, so convert .eps files to .pdf
\usepackage[pdftex]{graphicx}
\usepackage{epstopdf}
\fi

\newcommand{\ds}{\displaystyle}
\newcommand{\bv}{\mathbf}
\newcommand{\lv}{\langle}
\newcommand{\rv}{\rangle}

\begin{document}
\flushleft
\begin{multicols}{2}

\begin{large}\textbf{Math 116 Quiz 7: $\oint$ 9.1-9.3 \\ (Sequences and Series) \\
Tue 20 Nov 2012}\end{large}

\textbf{Name:  }\underline{\hspace{4pc}{\bf SOLUTIONS}\hspace{4pc}}

\vspace{.5in}

\end{multicols}

\pagestyle{empty}

\flushleft

You have 35 minutes to complete this quiz.  Eyes on your own paper and good luck!

\begin{enumerate}
\item  \textbf{Definitions/Concepts.} (1 pt ea) Decide whether each of the statements below is \emph{True} or \emph{False}.  Write the entire word \emph{True} or \emph{False}.  If the statement is false, briefly explain why. 
\begin{enumerate}
\item A convergent sequence is bounded.

True
\vfill
\item A bounded sequence converges.

False; the sequence $1,-1,1,-1,\dots$ is bounded but does not converge.
\vfill 
\item Changing a finite number of terms in a series does not change whether or not it converges, although it may change the value of its sum if it does converge.

True
\vfill
\item If $\sum_{n=1}^{\infty}a_n$ converges, then $\lim_{n\to\infty}a_n=0$.

True
\vfill
\end{enumerate}
 
\item \textbf{Questions/Problems.} {\it (from April 2011 Final Exam)} You are trapped on an island, and decide to build a signal fire to alert passing ships.  You start the fire with 200 pounds of wood.  During the course of a day, $40\%$ of the wood pile burns away (so $60\%$ remains).  At the end of each day, you add another 200 pounds of wood to the pile.  Let $W_n$ denote the weight of the wood pile immediately after adding the $n$th load of wood (the inital 200-pound pile counts as the first load).

\begin{enumerate}
\item (3 pts) Find expressions for $W_1$, $W_2$, $W_3$.

{\it -see the solution posted on the course website -}
\vspace{10pc}
\item (3 pts) Find a closed form expression for $W_n$ (a {\it closed form} expression means your answer should not contain a large summation).

{\it -see the solution posted on the course website -}
%\vspace{8pc}
\item (2 pts) Instead of starting with 200 pounds of wood and adding 200 pounds every day, you decide to start with $P$ pounds of wood and add $P$ pounds every day.  If you plan to continue the fire indefinitely, determine the largest value of $P$ for which the weight of the wood pile will never exceed 1000 pounds.

{\it -see the solution posted on the course website -}
\vspace{8pc}
\end{enumerate}

\vspace{1pc}
%\hfill{\bf MORE QUIZ ON THE BACK --\textgreater}

\item \textbf{Computations/Algebra.} 
 
\begin{enumerate}
\item (1 pt) Find a formula for the general term of the sequence $\frac{1}{3},\,\frac{2}{5},\,\frac{3}{7},\,\frac{4}{9},\,\frac{5}{11},\dots$.
\[s_n=\frac{n}{2n+1}\hspace{7pc}\]

%\vspace{0.5pc}
\item (2 pts) Does the sequence given by $s_n=\frac{2n+(-1)^n5}{4n-(-1)^n3}$ converge or diverge?  If it converges then find its limit.

To see if the limit exists, write
\begin{align*}
\lim_{n\to\infty}s_n &= \lim_{n\to\infty}\frac{2n+(-1)^n4}{4n-(-1)^n3} \\
&= \frac{\lim_{n\to\infty}2n+(-1)^n4}{\lim_{n\to\infty}4n-(-1)^n3} \\
&= \frac{\lim_{n\to\infty}2n}{\lim_{n\to\infty}4n},
\end{align*}
because whether $n$ is even or odd, we are still adding or subtracting the same constant, no matter how large $n$ gets.  Therefore the end behavior will be dominated by the linear terms $2n$ and $4n$. 
\begin{align*}
&= \lim_{n\to\infty}\frac{2n}{4n} \\
&= \lim_{n\to\infty}\frac{2}{4}=\frac{1}{2}.
\end{align*}
So the sequence converges, to $\frac{1}{2}$.

%\vspace{5pc}
\item (3 pts) Find the first three terms of the sequence of partial sums for the series $\sum_{n=1}^{\infty}\frac{1}{n(n+1)}$.
\begin{align*}
S_1 &=\frac{1}{1(1+1)} \\
&=\frac{1}{2} \\
S_2 &= S_1+\frac{1}{2(2+1)} \\
&= \frac{1}{2}+\frac{1}{6}=\frac{4}{6} \\
&= \frac{2}{3} \\
S_3 &=S_2+\frac{1}{3(3+1)} \\
&=\frac{2}{3}+\frac{1}{12} =\frac{9}{12} \\
&=\frac{3}{4}
\end{align*}
%\vspace{8pc}
\item (2 pts) Does the series $\sum_{n=1}^{\infty}\frac{n+1}{2n+3}$ converge or diverge?

We can try the integral test to check for convergence.  The series is positive.  To see if it is decreasing beyond a certain point, check the accuracy of the following statement:
\begin{align*}
\frac{n+1}{2n+3} &> \frac{(n+1)+1}{2(n+1)+3}=\frac{n+2}{2n+5} \\
(n+1)(2n+5) &> (n+2)(2n+3) \\
2n^2+8n+5 &> 2n^2+7n+6 \\
8n+5 &> 7n+6 \\
n&>1.
\end{align*}
From this we can conclude the function in the series is decreasing for all $n>1$.  

Evaluate the corresponding integral:
\begin{align*}
\int_1^{\infty}\frac{x+1}{2x+3}dx &=\frac{1}{2}\int_5^{\infty}\frac{u-3}{2u}du \\
& \text{(by setting $u=2x+3$)} \\
&= \frac{1}{2}\left(\int_5^{\infty}\frac{1}{2}du-\int_5^{\infty}\frac{3}{2u}du\right),
\end{align*}
but both terms in the parentheses diverge by the $p$-test ($p=0$ and then $p=1$) so we don't need to evaluate any further, and the sequences diverges.

\end{enumerate}

\end{enumerate}

\end{document}


