\documentclass[11pt,twoside]{article}
\pagestyle{empty}

\usepackage{latexsym}
\usepackage{amssymb}
\usepackage{amsfonts}
\usepackage{amstext}
\usepackage{multicol}
%%%%%%%%%%%%%%%%%%%%%%%%%%%%%%%formatting%%%%%%%%%%%%%%%%%%%%%%%%%%%%%%%%%%%%%%%%%%%%%%%%%%%%%%%
\setlength{\topmargin}{-.1in}        %%%  This sets all the spacing stuff to use the page more
\setlength{\oddsidemargin}{0in}    %%%  efficiently than the normal "article" setup would.
\setlength{\evensidemargin}{0in}   %%%  It's OK to play with these some.
\setlength{\textheight}{8.5in}     %%%
\setlength{\textwidth}{6.25in}     %%%
\setlength{\headsep}{0in}          %%%
\setlength{\headheight}{0in}       %%%
%\setlength{\footskip}{0in}         %%%

%%%%%%%%%%%%%%%%%%%%%%%%%%%%%%%%%%%%%%%%%%%%%%%%%%%%%%%%%%%%%%%%%%%%%%%%%%%%%%%%%%%%%%%%%%%%%%%

\begin{document}

\begin{center}
{\bf \Large \underline{Math 116-004 -- Fall 2012} \\
last updated: \today}
\end{center}

\vspace{.05in}

\begin{description}
\item[\bf Instructor:] Ashley Wheeler

\item[\bf Email:]  wheeles@umich.edu

\item[\bf Office:] 4848 East Hall

\item[\bf Website:] www-personal.umich.edu/$\sim $wheeles
\end{description}

\begin{description}
\item[\bf Office Hours:] 
\begin{tabular}{l}
Tue 11a-12p, Thu 1-2p (Math Lab), Fri 11a-12p, or by appointment
\end{tabular}

\vspace{.1in}
\item[\bf Text:] {\it Calculus} by Hughes-Hallet, Gleason, et al.,
5th Edition, published by John Wiley and Sons

Bring the book to class everyday.  READ THE BOOK!  Class time supplements the book and vice versa. 

\vspace{.1in}

\item[\bf Calculator:] TI-84 or equivalent. If you have another
model, you are responsible for knowing how to use it.  Bring your calculator to class each day and to the Uniform Exams.  No devices with QWERTY keyboard are allowed at the exams.

\vspace{.1in}

\item[\bf Course Website:] \emph{http://www.math.lsa.umich.edu/courses/116/}\\
Everything is here.  You can find information about grading and course policies (Student Guide) as well as the webwork, practice problems, assignments, and exam information.  

\vspace{.1in}

\item[\bf Course Content:] This semester we will cover sections 5.1-5.4, 6.1, 6.2, 6.4, 7.1, 7.2, 7.5, 7.7, 7.8, 4.8, 8.1-8.5, 8.7, 8.8, 9.1-9.5, 10.1-10.3, and 11.1-11.6 from the textbook.  Sections 6.3, 6.5, 7.3, 7.4, 7.6, 10.4, 10.5, and 11.7-11.11 are not covered.

\item[\bf Homework:]  Daily webwork will be assigned from each section we cover.  In order to do well in class, you must keep up with the daily assignments.  In addition, you will be given regular team homework assignments, and a large portion of your in-class grade will be based on these group assignments.  {\bf Team Homeworks are due Fridays at the beginning of class.}

\vspace{.1in}
\item[\bf Quizzes:]  There will be short weekly quizzes.  Quizzes are on Tuesdays, and are designed to check you are up to speed on reading.  No make-up quizzes will be given, but I will drop your lowest quiz score at the end of the term.

\vspace{.1in}
\item[\bf Uniform Exams:]\mbox{\ }

\begin{tabular}{lll}
Exam 1 &  Wed 10 Oct, 6-7:30p & 5.1-5.4, 6.1, 6.2, 6.4, 7.1, 7.2, 7.5, 8.1, 8.2, 8.4, 8.5 \\
Exam 2 & Wed 14 Nov, 6-7:30p & 11.1-11.6, 4.8, 8.3, 7.7, 7.8, 8.7, 8.8 \\
Final &  Fri 14 Dec, 8-10a & 9.1-9.5, 10.1-10.3, all Exam 1 and Exam 2 material
\end{tabular}

Dates for the exams are fixed.  Only students with a regularly scheduled class are accommodated at an alternate time.  If you have a regularly scheduled class during any of these times please let your instructor know as soon as possible.  \textbf{Travel is \emph{not} a sufficient excuse to have an exam scheduled on a different day.}

Notes on exam coverage:
\begin{itemize}
\item The sections indicated for each exam may change if the schedule requires it. If this
occurs it will be announced in class prior to the exam.
\item Although Exam 2 is not explicitly cumulative, a thorough understanding of the material
covered in Exam 1 is needed to master the material covered in Exam 2. However,
Exam 2 will primarily focus on the material covered since Exam 1.
\item The Final Exam is cumulative.
\item On all exams, standard graphing calculators (those without a full alphanumeric
keypad) are allowed. Problems will be written with the expectation that these
calculators will be used. Other or more powerful calculators must be approved. On all
exams, students are allowed to bring notes written on both sides of one 3'' by 5'' card.
\end{itemize}

\vspace{.1in}
\item[\bf Grading Policy:]  All sections of Math 116 use the same grading guidelines to standardize the evaluation process.  The three uniform exams are worth 25\% (Exam 1), 30\% (Exam 2), and 40\% (Final) of the ``Uniform Component" of each student's grade, and the
web homework is worth 5\% of the Uniform Component. The final course grade will be
primarily determined by the Uniform Component grade for each student. However, for some
students, the final course grade may be modified by the section component grade or the
gateway exams.  A complete explanation of the grading policy is given in the Student Guide on the course web site. 

\vspace{.1in}
\item[\bf In-Class Grade (a.k.a. Section Component):]  Look carefully at the Student Guide for an explanation of how the section component grading policy works.  Your in-class grade is your instructor's objective way of determining if your grade can be ``bumped" up or down by one third of a letter grade.  The in-class grade is broken down as follows:  

\begin{tabular}{rl}
 Team Homework & 60\% \\
 Quizzes & 40\% \\
\end{tabular}

\vspace{.1in}
\item[\bf Gateway Exams:]  There are two gateway tests for Math 116 students. The first reviews
differentiation and the second covers techniques of integration. Practice gateway exams will
be available for you to help you prepare. Students will lose a third of a letter grade (from their
Uniform Component grade) for failing to pass the first gateway and a third of a letter grade for
failing to pass the second. Gateways have to be taken in the gateway lab (B069 EH) and
may be taken many times, but no more than twice per day. The opening and closing dates
for the Gateway Exams will be announced in class and posted at the course website (under
Key Dates).

\vspace{.1in}
\item[\bf Math Lab:]  Free tutoring from the Mathematics Department:

\begin{tabular}{ll}
{\bf Hours:} & Mon through Thu 11a-4p, 7-10p \\
             & Fri 11a-4p \\
             & Sun 7-10p \\
{\bf Location:} & B860 East Hall
\end{tabular}

\vspace{.1in}
\item[\bf Other Important Dates:]
\begin{tabular}{ll}
Last day to drop without a W & Mon 24 September \\
Fall Study Break & Mon-Tue 15-16 October \\
LSA Drop Deadline (with a W) & Fri 9 November \\
Thanksgiving Recess & Thu-Fri 22-23 November \\
Last Day of Classes & Mon 11 December
\end{tabular}

\vspace{.4in}
\item[\bf Disabilities:]  Any student with a documented disability should contact me as soon as possible so that we can discuss arrangements to fit your needs.
\end{description}

\end{document}
