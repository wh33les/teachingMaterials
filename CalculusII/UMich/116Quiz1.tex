\documentclass[11pt,letterpaper]{article}
\usepackage{fullpage}
\usepackage{multicol}
\usepackage{amsmath}
\usepackage{amsfonts}
\usepackage{amssymb}
%\usepackage{pstricks, pst-node, pst-plot}

\newcommand{\ds}{\displaystyle}
\newcommand{\bv}{\mathbf}
\newcommand{\lv}{\langle}
\newcommand{\rv}{\rangle}

\begin{document}
\flushleft
\begin{multicols}{2}


\begin{large}\textbf{Math 116 Quiz 1: $\oint$ 5.1-5.4, 6.1-6.2 \\
Tue 11 Sep 2012}\end{large}

\textbf{Name:  }\underline{\hspace{35ex}}

\vspace{.5in}

\end{multicols}

\pagestyle{empty}


\flushleft

You have 15 minutes to complete this quiz.  Eyes on your own paper and good luck!

\begin{enumerate}
\item  \textbf{Definitions/Concepts.} (3 pts) Write down the Fundamental Theorem of Calculus.  
\vfill

\vspace{5pc}

\item \textbf{Questions/Problems.} 
\begin{enumerate}
\item (2 pts) Recall that when we want to estimate area under a curve for a function $f(t)$ over the interval $t\in[a,b]$ we can use a left-hand or right-hand approximation.  Let $n$ denote the number of equally-sized subdivsions we use to divde the interval $[a,b]$.  Then
\[\Delta t=\frac{b-a}{n}\]
and we can let $t_0=a,\;t_1=t_0+\Delta t,\;t_2=t_1+\Delta t$, etc.   

Suppose you have the data:
\begin{table}[h]
\centering
\begin{tabular}{|c|c|c|c|c|c|}
\hline
t & 0 & 4 & 8 & 12 & 16 \\ [0.5ex]
\hline 
f(t) & 25 & 23 & 22 & 20 & 17 \\ [0.5ex]
\hline
\end{tabular}
\caption{number of students awake after $t$ minutes into a boring lecture}
\end{table}

Use this data to fill in the missing information:
\[\begin{array}{| l l l l l |}
\hline
n=4 & & & & \\ [1ex]
\Delta t=\hspace{1 cm} & & & & \\ [1ex]
a=\hspace{1 cm} & b=\hspace{1 cm} & & &\\ [2ex]
t_0=\hspace{1 cm} & t_1=\hspace{1 cm} & t_2=\hspace{1 cm} & t_3=\hspace{1 cm} & t_4=\hspace{1 cm} \\ [2ex]
f(t_0)=\hspace{1 cm} & f(t_1)=\hspace{1 cm} & f(t_2)=\hspace{1 cm} & f(t_3)=\hspace{1 cm} & f(t_4)=\hspace{1 cm} \\ [2ex]
\hline
 n=2 & & & & \\ [1ex]
\Delta t=\hspace{1 cm} & & & & \\ [1ex]
a=\hspace{1 cm} & b=\hspace{1 cm} & & &\\ [2ex]
t_0=\hspace{1 cm} & t_1=\hspace{1 cm} & t_2=\hspace{1 cm} &  &  \\ [2ex]
f(t_0)=\hspace{1 cm} & f(t_1)=\hspace{1 cm} & f(t_2)=\hspace{1 cm} &  &  \\ [2ex]
\hline
\end{array}\]

\hfill{\bf MORE QUIZ ON THE BACK --\textgreater}
\vfill

\item (3 pts each) Write out the entire word, either True or False.  No justification is needed.
\begin{enumerate}
\item If $\int_0^2(3f(x)+1)dx=8$, then $\int_0^2f(x)dx=2$.
\vfill
%\item If $\int_a^bf(x)dx=2$ and $\int_a^bg(x)dx=-3$, then $\int_a^bf(x)g(x)dx=-6$.
%\vfill
\item If $f(x)=\int_{-2x}^0(1+t^4)dt$, then $f(x)$ is decreasing.
\vfill
\item If $f(x)\leq g(x)$ for $x\in[0,1]$, then $\int_0^1f(x)dx\leq\int_0^1g(x)dx$.
\vfill
\item If $g(x)$ is odd and $\int_1^3g(x)dx=2$, then $\int_{-3}^{-1}g(x)dx=2$.
\vfill
\item If $f(t)$ is measured in dollars per year, and $t$ in measured in years, then $\int_a^bf(t)dt$ is measured in dollars per years squared.
\vfill
\end{enumerate}

\end{enumerate}
\item \textbf{Computations/Algebra.} 

\it -none this week-

\end{enumerate}

\end{document}


