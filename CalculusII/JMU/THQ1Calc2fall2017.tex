\documentclass[%margin%,line,pifont,palatino,courier
]{article}
\usepackage{fullpage}
\usepackage{lastpage}
\usepackage[top=1in,bottom=1in,margin=1in]{geometry}
\usepackage{supertabular}
\usepackage{graphicx,tikz}	
%\usepackage{tkz-euclide}
%\usetkzobj{all}
%\usetikzlibrary{calc}
\usepackage{array,multicol}
\usepackage{amsmath,amssymb}
\usepackage{enumitem}

\usepackage{fancyhdr}
\pagestyle{fancy}

\addtolength{\topmargin}{-0.25in}

\newcommand{\vect}[1]{\mathbf{#1}}
\DeclareMathOperator{\proj}{proj}

\fancypagestyle{plain}{
	\addtolength{\headheight}{0.485in}
	\rhead{\bf MATH 236 (Calculus II) \\
		%\vspace{0.5pc}
		due Mon 11 Sep 2017 \\}
	\rfoot{\footnotesize THQuiz 1, p. \thepage\ (of \pageref{LastPage})
	}
\renewcommand{\headrulewidth}{0pt}
}
\fancyhf{}
\renewcommand{\headrulewidth}{0pt}
\rfoot{\footnotesize THQuiz 1, p. \thepage\ (of \pageref{LastPage})$\;$}

\title{\vspace{-3.5pc} 
	\flushleft \bf \Large Take-Home Quiz 1: %\\ 
	The Fundamental Theorem of Calculus (\S 4.5, 4.7) and introduction to integrals (\S 5.1-5.2)}
\date{}

% % % % %
\begin{document}
\maketitle

\vspace{-3pc}
\noindent{\bf Directions:} This quiz is due on September 11, 2017 at the beginning of lecture.  You may use whatever resources you like -- e.g., other textbooks, websites, collaboration with classmates -- to complete it \textbf{but YOU MUST DOCUMENT YOUR SOURCES}.  Acceptable documentation is enough information for me to find the source myself.  Rote copying another's work is unacceptable, regardless of whether you document it.  

\noindent\hrulefill

\begin{enumerate}
% % %
\item \begin{enumerate}
	\item {\bf 4.5 \#18} In the proof of the Fundamental Theorem of Calculus (version that everyone remembers -- given in class), the Mean Value Theorem is used to choose values $x_k^*$.  
	\begin{enumerate}
		\item Use the Mean Value Theorem in the same way to find the corresponding values $x_1^*,x_2^*,x_3^*$ for a Riemann sum approximation of $\int_0^3x^3dx$ with three rectangles.  
		\item Round the answer of the approximation
		\[
		\int_0^3 x^3\ dx \approx \sum_{k=1}^3 (x_k^*)^3 (1)
		\]
		to one decimal place.  
		\item What is the exact value of $\int_0^3x^3 dx$?
		\end{enumerate}
	%
	\item {\bf 4.7 \#16} Your classmate is trying to understand the proof of the Second Fundamental Theorem of Calculus (i.e., the version that everyone forgets -- given in the text, but not in class) and asks you how we get the inequality 
	\[
	f(m_h)h\leq \int_x^{x+h} f(t)\ dt\leq f(M_h)h.
	\]
	Explain how, including what each quantity ($f(m_h)h$, the integral in the middle, $f(M_h)h$) represents. 
	\end{enumerate}
	
% % %
\item {\bf 4.7 \#22} \begin{enumerate}
	\item Express the signed area between the graph of $y=\frac{1}{x}$ and the $x$-axis from $x=\frac{1}{4}$ to $x=1$ in terms of logarithms.  
	\item Sketch the graph of $y=\frac{1}{x}$ with the area from part (a) labeled.
	\item Evaluate the area, rounding to one decimal place.
	\end{enumerate}
	
% % %
\item {\bf 4.7 \#62}  {\it For this problem, it is helpful to use a graphing utility like \verb+desmos.com/calculator+.}  Ian is climbing every day, using a camp at the base of a snowfield.  His only supply of water is a trickle that comes out of the snowfield.  The trickles dries at night, because the temperature drops and the snow stops melting.

Ian entertains himself by using measurements of the water in his cooking pot to model the flow as 
\[
w(t)=15\left(1+\cos\left(\frac{2\pi(t-16)}{24} \right) \right),
\]
where $t$ is the time in hours after midnight and $w(t)$ is the rate at which water drips into his pot, in gallons per hour.
	\begin{enumerate}
	\item At 4p Ian puts his empty pot under the trickle.  About how long does it take to fill the 2-quart pot?  Round your answer to one decimal place.  What time is it when the pot has become full?
	\item What is the total amount of water that flows out of the snowfield in a single 24-hour day?  Write a simplified expression for the exact answer, along with the answer rounded to one decimal place.
	\item Write an expression for the total amount of water that flows from the snowfield between any starting time $t_0$ and ending time $t$.
	\item Use your formula from part (c) to answer: If Ian stashes his empty pot under the trickle at 5 in the morning, how long must he wait until he comes back to a full pot?  What time will it be?
	\item What time of day is the water flowing the fastest?
	\item What time should Ian stash his empty pot under the trickle to have it filled in the shortest possible time?  What time will it be once it's filled?
	\end{enumerate}

% % %
\item {\bf 5.1 \#86} A mass hanging at the end of a spring oscillates up and down from its equilibrium position with velocity 
\[
v(t)=3\cos\left(\frac{3t}{\sqrt 2}\right)-3\sqrt 2\sin\left(\frac{3t}{\sqrt 2}\right)
\]
centimeters per second.  The mass is at its equilibrium at $t=0$.  Use definite integrals to determine whether the mass will be above or below its equilibrium position at times $t=4$ and $t=5$.

% % %
\item {\bf 5.2 \#90} Prove the integral formula
\[
\int \arcsin x\ dx = x\arcsin x +\sqrt{1-x^2} + C
\]
	\begin{enumerate}
	\item by applying integration by parts to $\int \arcsin x\ dx$;
	\item by differentiating $\sqrt{1-x^2}+x\arcsin x$.
	\end{enumerate}
% % % % %
\end{enumerate}
\end{document}