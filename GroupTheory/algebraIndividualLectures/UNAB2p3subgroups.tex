\documentclass[../algebraNotesMSRI-UP2016.tex]{subfiles}

\begin{document}

\section[\S \thesection]{Subgroups}\label{sec:2p3subgroups}
% % % % %
\subsection[\subsecname]{Definition}
% % %
\begin{frame}[c]{\subsecname}
\begin{dfn}
Let $(G,\star)$ denote a group.  A \vocab{subgroup} $H< G$ is a subset of $G$ such that $(H,\star)$ is also a group.  
\end{dfn}
\end{frame}

% % %
\begin{frame}
\begin{ex}\label{ex:6Z}
The set $6\Z\subset \Z$ of multiples of $6$ forms a subgroup of $\Z$.  
\end{ex}
%
%\smallGap
\pf We must verify the group axioms:
\begin{itemize}
\item Is $+$ binary on $6\Z$?  (Verifying this condition is sometimes called ``showing $6\Z$ is \vocab{closed under addition}".)  

\smallGap
Suppose $g,h\in 6\Z$.  Since $g$ and $h$ are multiples of $6$ then there exist (integers) $n$ and $m$ such that $g=6n$ and $h=6m$.  We must show $g+h\in 6\Z$:
\begin{align*}
g+h &= 6n+6m \\
	&= \underbrace{n+\cdots +n}_{\text{$6$ times}}+\underbrace{m+\cdots +m}_{\text{$6$ times}} \\
	&= \underbrace{(n+m)+\cdots +(n+m)}_{\text{$6$ times}}\text{ (by associativity in $G$)} \\
	&= 6(n+m).
\end{align*}
\end{itemize}
\end{frame}

% % %
\begin{frame}
\begin{itemize}
\item[] Since $\Z$ is a group $n+m\in\Z$ and hence $g+h=6(n+m)$ is an integer multiple of $6$.  Therefore $g+h\in 6\Z$.
\item Is $+$ associative in $6\Z$?  

\smallGap
In general we never have to prove this property since $6\Z\subset \Z$ and therefore $+$ inherits associativity from $\Z$.
\item Does $6\Z$ contain the identity element?

\smallGap
In our case we need to show the identity element is a multiple of $6$.  The additive identy of $\Z$ is $0=6\cdot 0$, so $0\in 6\Z$.
\item Does every element in $6\Z$ have an inverse in $6\Z$ (i.e., is $6\Z$ is ``closed under inverses")?

\smallGap
Suppose $g\in 6\Z$, and write $g=6n$ for some integer $n$.  Then $g+(-g)=(6n)+(-6n)=0=(-6n)+(6n)$.  We conclude $-g=-6n$.  And, $-g=-6n=6(-n)\in 6\Z$.
\end{itemize}
\qed
\end{frame}

% % %
\begin{frame}[c]
Example \ref{ex:6Z} also holds when we replace $6$ with any integer $k$ (see also, Example \ref{ex:kZcyclic}).  
\end{frame}

% % %
\begin{frame}[c]
The following is a shortcut for proving a subset is a subgroup.

\smallGap
\begin{prop}\label{prop:subgroupShortcut}
Suppose $G=(G,\star)$ is a group and $H$ is a non-empty subset of $G$.  Then $H< G$ if $gh^{-1}\in H$ for every $g,h\in H$. 
\end{prop}

\smallGap
\begin{que}
Why do we suppose $H$ is non-empty?
\end{que}

\smallGap
\begin{exe}\label{exe:subgroupShortcut}
Prove Proposition \ref{prop:subgroupShortcut}.  Is the converse true?
\end{exe}
\end{frame}

% % %
\begin{frame}[c]
\begin{ex}[cf. Problem 49]\label{ex:prob49}
Suppose $H$ is a subset of $G=(G,\star)$ satisfying the following:
\begin{enumerate}[(i)]
\item\label{ex:prob49-1} $H$ is closed under $\star$.
\item\label{ex:prob49-2} If $g\in H$ then $g\1\in H$.
\end{enumerate}
Then $H< G$.
\end{ex}

\smallGap
\pf
We must verify the group axioms.  Item \lilRefP{ex:prob49-1} implies $\star$ is binary on $H$ and associativity is inherited from $G$.  We must check the identity element $e$ is in $H$.  From \lilRefP{ex:prob49-2}, if an element $g$ is in $H$, then so is its inverse.  Combine that with \lilRefP{ex:prob49-1} then $gg\1=e\in H$.  Finally, the existence of inverse elements is given by \lilRefP{ex:prob49-1}.
\qed

\smallGap
\textbf{Quicker Proof:} 
Suppose $g,h\in H$.  By \lilRefP{ex:prob49-1} $h\1\in H$ and by \lilRefP{ex:prob49-2} $gh\1\in H$.  It follows from Proposition \ref{prop:subgroupShortcut} that $H<G$.
\qed
\end{frame}

% % %
\begin{frame}[c]{}{}
\begin{exe}[cf. Problem 50]\label{exe:prob50}
Let $G=\Z_{12}$, as defined in Exercise \ref{exe:prob41}.  Show that $H=\{0,3,6,9\}$ is a subgroup of $G$.
\end{exe}
\end{frame}

% % % % %
\subsection[\subsecname]{Cyclic subgroups}\label{subsec:cyclic}
% % %
\begin{frame}[c]{\subsecname}
%Cyclic subgroups are a very important class of groups.
\begin{dfn}\label{dfn:cyclicSubgroup}
Suppose $G$ is a group and $g\in G$.  Define
\[
\langle g\rangle:=\{g^n\mid n\in\Z\}
\]
as the \vocab{cyclic subgroup} of $G$ \vocab{generated by} $g$.
\end{dfn}

\smallGap
\begin{exe}\label{exe:cyclicSubgroup}
For Definition \ref{dfn:cyclicSubgroup} to make sense, we must check $\langle g\rangle$ actually is a subgroup.  
\end{exe}
\end{frame}

% % %
\begin{frame}[c]
As alternative notation, authors may write $gG=\langle g\rangle$ to denote the subgroup in $G$ generated by $g$ (see Example \ref{ex:6Z}).

%\smallGap
%\begin{que}
%Why?
%\end{que}

\smallGap
\begin{ex}\label{ex:kZcyclic}
For any integer $k$, the subgroup $k\Z<\Z$ is cyclic.  

\smallGap
(Recall, from Section \ref{subsec:identitiesInverses}, the subgroups $0\Z<\Z$ and $1\Z<\Z$.)
\end{ex}

\smallGap
\begin{que}
What is $0\Z$?  What is $1\Z$?
\end{que}
\end{frame}

% % %
\begin{frame}
We can define subgroups with more than one generator, though we do not describe such subgroups as cyclic.   

\smallGap 
\begin{dfn}\label{dfn:generators}
Let $S=\{s_1,\dots,s_k\}$ denote some set of elements in the group $G$.  The subgroup generated by $S$ is defined as
\[
\langle S\rangle=SG
:=\{s_1^{n_1}\cdots s_k^{n_k}\mid n_i\in \Z \text{ for all }i=1,\dots k\}.
\]
Elements in $\langle S\rangle$ are called \vocab{words}.  
\end{dfn}

\smallGap
\begin{que}
How would you rewrite Definition \ref{dfn:generators} in additive notation?  
\end{que}

\smallGap
\begin{exe}[cf. Problem 51]\label{exe:prob51}
Prove $\langle S\rangle$ in Definition \ref{dfn:generators} is a subgroup of $G$. 
\end{exe}
\end{frame}

% % %
\begin{frame}
\begin{ex}[cf. Problem 52]
In $\Z_{12}$, we list the elements of the subgroup $H=\langle 2,3\rangle$ by writing down \vocab{$\Z$-linear combinations} of the generators, i.e., all possible elements of the form $n_1\cdot 2+n_2\cdot 3$ for $n_1,n_2\in\Z$:

{\footnotesize
\begin{tabular}{p{0.2\textwidth}p{0.2\textwidth}p{0.2\textwidth}p{0.2\textwidth}}
{\begin{align*}
1\cdot 2 +0\cdot 3 &= 2 \\
2\cdot 2 +0\cdot 3 &= 4 \\
3\cdot 2 +0\cdot 3 &= 6 \\
4\cdot 2 +0\cdot 3 &= 8 \\
5\cdot 2 +0\cdot 3 &= 10 \\
6\cdot 2 +0\cdot 3 &= 0
\end{align*} 
} & 
{\begin{align*}
0\cdot 2 +1\cdot 3 &= 3 \\
0\cdot 2 +3\cdot 3 &= 9 
\end{align*}
} & 
{\begin{align*}
1\cdot 2 +1\cdot 3 &= 5 \\
2\cdot 2 +1\cdot 3 &= 7 \\
4\cdot 2 +1\cdot 3 &= 11
\end{align*} 
} &
{\begin{align*}
(-1)\cdot 2+1\cdot 3 &= 1
\end{align*}
}
\end{tabular}
}
Having exhausted all possible elements, we conclude $\langle 2,3\rangle=\Z_{12}$.
\end{ex}
\end{frame}

% % % % %
\answerKey
% % %
\begin{frame}{\subsecname}
\exeSol[(cf. notes)]{exe:subgroupShortcut}
Take $g,h\in H$.  We shall exploit the hypothesis statement: $gh\1\in H$.  In the case where $g=h$, we have $gg\1=e\in H$.  Using the hypothesis again, $e,h\in H$ implies $eh\1=h\1 \in H$.  Therefore every element in $H$ has an inverse in $G$.  Finally, we must show closure of the binary operation, i.e., that $gh\in H$.  Since $h\in H$, so is $h\1$.  Then the hypothesis says $g(h\1)\1=gh\in H$.
\qed

\smallGap
The converse states that if $H$ is a subgroup then for all $g,h\in H$, we have $gh\1\in H$.  This statement is \textbf{true} and follows directly from the group axioms for $H$.

\smallGap
\exeSol[(cf. Problem 50)]{exe:prob50}
By Example \ref{ex:prob49} it suffices to show closure under $\oplus_{12}$ and the presence of inverse elements.  $H$ consists of multiples of $3$ modulo $12$, and so adding two of them results in a multiple of $3$ as well.  The inverses are $-0=0$, $-3=9$, $-6=6$.
\qed
%\smallGap
\end{frame}

% % %
\begin{frame}{}
\exeSol{exe:cyclicSubgroup}
We appeal to Example \ref{ex:prob49}.  Take $g^n,g^m\in \langle g\rangle$.  Then $g^ng^m=g^{n+m}\in G$, i.e., $\langle g\rangle$ is closed under the operation in $G$.  For inverses, the definition of $\langle g\rangle$ includes inverses, since $(g^n)\1=g^{-n}$.
\qed

\smallGap
\exeSol[(cf. Problem 51)]{exe:prob51}
We modify the arguments used in Exercise \ref{exe:cyclicSubgroup}.  If $s_1^{n_1}\cdots s_k^{n_k}$ and $s_1^{m_1}\cdots s_k^{m_k}$ are in  $\langle S\rangle$, then their product is
\[
(s_1^{n_1}\cdots s_k^{n_k})\cdot (s_1^{m_1}\cdots s_k^{m_k})=s_1^{n_1+m_1}\cdots s_k^{n_k+m_k}\in \langle S\rangle.
\]
For the presence of inverses, put
\[
s:=(s_1^{n_1}\cdots s_k^{n_k})\1=s_k^{-n_k}\cdots s_1^{-n_1}.
\]
By definition, each of $s_i^{-n_i}\in \langle S\rangle$, for $i=1,\dots,k$.  Since we showed closure under the group operation, $s\in\langle S\rangle$.
\qed
\end{frame}

% % % % % % % % % % % % % % % % % % % %
\end{document}