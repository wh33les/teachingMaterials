\documentclass[../algebraNotesMSRI-UP2016.tex]{subfiles}

\begin{document}

\section[\S \thesection]{Quotient groups}\label{sec:2p7quotientGroups}
% % % % % 
\subsection[\subsecname]{Cosets}
% % %
\begin{frame}{\subsecname}
\begin{dfn}
Let $(G,\star)$ denote a group with subgroup $H< G$ and suppose $g\in G$.  The set
\[
g\star H:=\{g\star h\mid h\in H\}
\]
is called a \vocab{left coset} of $H$ in $G$.  The set
\[
H\star g:=\{h\star g\mid h\in H\}
\]  
is called a \vocab{right coset} of $H$ in $G$.
\end{dfn}
%
%\smallGap
When using multiplicative notation we may write $gH=g\star H$; likewise with additive notation we write $g+H=g\star H$.

\begin{que}
What is the condition for a coset to be a subgroup?  (In general, a coset is NOT a subgroup!)
\end{que}
\end{frame}

% % %
\begin{frame}{}{}
\begin{ex}\label{ex:D3}
Let $G=D^3$, the dihedral group of order $3$ (see Example \ref{ex:D4}).  Take $H=\{e,s\}<G$.  The left cosets of $H$ are 
\[
\begin{aligned}
eH &= H \\
sH &= H
\end{aligned}
\quad
\begin{aligned}
rH &= \{r,rs \} \\
rsH &= \{rs, r\}
\end{aligned}
\quad
\begin{aligned}
r^2H &= \{r^2,r^2s\} \\
r^2sH &= \{r^2s, r^2\}.
\end{aligned}
\]
The right cosets of $H$ are
\[
\begin{aligned}
He &= H \\
Hs &= H
\end{aligned}
\quad
\begin{aligned}
Hr &= \{r, r^2s \} \\
Hrs &= \{rs, r^2\}
\end{aligned}
\quad
\begin{aligned}
Hr^2 &= \{r^2,rs\} \\
Hr^2s &= \{r^2,r\}.
\end{aligned}
\]
\end{ex}

\smallGap
In particular, the left and right cosets are not equal.
\end{frame}

% % %
\begin{frame}
\begin{ex}\label{ex:alternating}
In $S_n$, a permutation that switches two elements $a$ and $b$ and leaves the rest fixed is called a \vocab{transposition}, denoted $(a b)$.  It can be shown that every permutation can be decomposed as a product of transpositions.  A permutation is \vocab{even} (resp., \vocab{odd})  means its product of transpositions has an even (resp. odd) number of factors.  

\smallGap
The set $A_n$ of even permutations is a subgroup in $S_n$ (check!) called the \vocab{alternating group of order $n$}.  $A_n$ has $2$ cosets: itself, and the set of odd permutations.  In fact, for any $\sigma\in S_n$, $\sigma A_n=A_n\sigma$.  However, $S_n$ is not abelian, nor is $A_n$:
\[
(12)(23)\left(1,2,3\right)=(3,1,2)\quad \text{but}\quad (23)(12)\left(1,2,3\right)=(2,3,1).
\]
\end{ex}

\smallGap
Example \ref{ex:alternating} demonstrates that equality of left and right cosets is still a weaker condition than abelianness.
\end{frame}

% % %
\begin{frame}
\begin{dfn}\label{dfn:normal}
A subgroup $H<G$ is \vocab{normal} in $G$ means $gH=Hg$ for all $g\in G$.  We write $H\norml G$.
\end{dfn}

\smallGap We may also say a subgroup $H<G$ is normal if and only if $H$ is closed under \vocab{conjugacy}, meaning, $g\1Hg=H$ for all $g\in G$.  Evidently this is equivalent to Definition \ref{dfn:normal}.  The operation 
\begin{align*}
g:\{\text{subgroups in $G$}\} &\to \{\text{subgroups in $G$}\} \\
 H &\mapsto g\1Hg
\end{align*} 
for fixed $g\in G$ is called \vocab{conjugation}.  We can also conjugate individual elements $h\in G$, i.e., $g\1hg$.  For example, in Linear Algebra conjugating a matrix $A$ by the ``correct" matrix $B$ can result in a simpler way to express the linear transformation given by $A$ (see Section \ref{subsec:normalizing}).
\end{frame}

% % %
\begin{frame}
\begin{ex}\label{ex:trivialNormal}
For any group $G$, the two \vocab{trivial} subgroups, $\langle 1_G\rangle<G$ and $G<G$, are normal.
\end{ex}

\smallGap
\begin{UNABexe}\label{exe:normalKernel}
Prove the kernel of a group homormorphism is always normal in its source.
\end{UNABexe}

\smallGap
\begin{UNABexe}\label{exe:normalSubgroups}
Prove all subgroups of an abelian group are normal.
\end{UNABexe}

\smallGap
Nearly all the groups we consider are abelian.  By default we shall work with left cosets, keeping in mind all statements we make have right coset analogues.  
\end{frame}

% % %
\begin{frame}
\begin{ex}
In $\Z_{12}$, the cosets of $H:=\{0,4,8\}=4\Z_{12}$ are:
\begin{align*}
H = 0\oplus_{12}H &= 4\oplus_{12}H = 8\oplus_{12}H = \{0,4,8\}, \\
	 1\oplus_{12}H &= 5\oplus_{12}H = 9\oplus_{12}H = \{1,5,9\},	\\
	 2\oplus_{12}H &= 6\oplus_{12}H = 10\oplus_{12}H = \{2,6,10\}, \\
	 3\oplus_{12}H &= 7\oplus_{12}H = 11\oplus_{12}H = \{3,7,11\} 
\end{align*}
\end{ex}

\smallGap
\begin{que}
What are some observations you can make?
\end{que}
\end{frame}

% % %
\begin{frame}
We use $G/H$ to denote the collection of distinct cosets of $H$ in $G$, called \vocab{$G$ modulo $H$}.  The cosets of a subgroup \vocab{partition} the group:
\begin{prop}\label{prop:cosetsPartition}
Let $G$ denote a group with subgroup $H< G$.
\begin{enumerate}[(a)]
\item The union of all (left, respectively, right) cosets of $H$ in $G$ is the entire group $G$.
\item For any two cosets $g\star H,h\star H\in G/H$, either
\begin{enumerate}[(i)]
	\item $g\star H=h\star H$ or
	\item $g\star H\cap h\star H=\emptyset$.
\end{enumerate}
\end{enumerate}
\end{prop}

\smallGap
\pf
\begin{itemize}
\item[(a)] The union of all cosets of $H$ is contained in $G$; on the other hand, for any $g\in G$, $gH$ contains $g$ since $1_G\in H$.
\end{itemize}
\end{frame}

% % %
\begin{frame}[c]
\begin{itemize}
\item[(b)] Suppose $g_1H\cap g_2H\neq \emptyset$ and $g\in g_1H\cap g_2H$.  Then there exist $h_1,h_2\in H$ such that 
\[
g=g_1h_1=g_2h_2.
\]
Multiplying on the right by $h_1\1$ shows $g_1\in g_2H$, and so $g_1H\subseteq g_2H$.  Likewise, multiplying on the right by $h_2\1$ shows $g_2\in g_1H$ and so $g_2H\subseteq g_1H$.  Therefore $g_1H=g_2H$.
\end{itemize}
\qed
\end{frame}

% % %
\begin{frame}[c]
\begin{exe}[cf. Problem 66]\label{exe:prob66}
Let $G=\Z_{30}$ and put $H=5G$.  Using Proposition \ref{prop:cosetsPartition}, list the elements of $G/H$.  
\end{exe}

\smallGap
\begin{exe}[cf. Problem 67]\label{exe:prob67}
Let $G=\Z_2\times \Z_4$ and let $H=(1,1)G< G$.  List the elements of $H$, then list the cosets of $H$.
\end{exe}
\end{frame}

% % %
\begin{frame}
The following proposition gives a way to prove two cosets are equal.

\smallGap
\begin{prop}\label{prop:equalCosets}
Suppose $(G,\star)$ is a group with subgroup $H< G$ and suppose $g,h\in G$.  Then:
\begin{enumerate}[(a)]
\item $g\star H=H$ if and only if $g\in H$
\item $g\star H=h\star H$ if and only if $g\1h\in H$.
\item $g\star H=h\star H$ if and only if $h\in g\star H$.
\end{enumerate}
\end{prop}

\smallGap
\begin{exe}[cf. Problem 68]\label{exe:prob68}
Prove Proposition \ref{prop:equalCosets}.  \textit{Hint: Prove {\usebeamercolor[fg]{block title}(a)} first, then use it to prove {\usebeamercolor[fg]{block title}(b)}, then use {\usebeamercolor[fg]{block title}(b)} to prove {\usebeamercolor[fg]{block title}(c)}.}
\end{exe}
\end{frame}

% % % % %
\subsection[\subsecname]{Quotient groups}
% % %
\begin{frame}{\subsecname}
Let $G=(G,\star)$ denote a group with subgroup $H< G$.  The notation used thusfar suggest a group structure on $G/H$ with a binary operation $\star_{/H}$ well-defined ``up to", or \emph{modulo} elements in $H$.  The natural choice is to define
\begin{gather}\label{eq:quotientOperation}
\begin{split}
\star_{/H}: G/H\times G/H &\to G/H \\
(g\star H,h\star H) &\mapsto (g\star h)\star H.
\end{split}
\end{gather} 

\smallGap
Given $g_1,g_2,h_1,h_2\in G$, we must verify 
\begin{align*}
(g_1\star H,h_1\star H) &=(g_2\star H,h_2\star H) \\
\implies (g_1\star H)\star_{/H}(h_1\star H) &=(g_2\star H)\star_{/H}(h_2\star H) \\
\implies (g_1\star h_1)\star H &= (g_2\star h_2)\star H.
\end{align*}
\end{frame}

% % %
\begin{frame}
Component-wise, we have, by hypothesis, 
\[
g_1\star H=g_2\star H\quad\text{ and }\quad h_1\star H=h_2\star H.
\]
Along with associativity,
\begin{align*}
(g_1\star h_1)\star H = g_1\star(h_1\star H) \\
	= g_1\star(h_2\star H) &= g_1\star(H\star h_2) \tag{\alert{!!!}}\\
	&= (g_1\star H)\star h_2 \\
	&= (g_2\star H)\star h_2 \\
	&= g_2\star (H\star h_2) = g_2\star (h_2\star H) \tag{\alert{!!!}}\\
	&\phantom{= g_2\star (H\star h_2)} \;=(g_2\star h_2)\star H;
\end{align*}
provided the lines tagged with (\alert{!!!}) are correct.

\smallGap
\begin{que}
What additional hypothesis do we need?
\end{que}
\end{frame}

% % %
\begin{frame}[c]
\begin{thm}\label{thm:quotientGroup}
Let $G=(G,\star)$ denote a group with normal subgroup $H\norml G$.  The set $G/H$ is a group, called the \vocab{quotient group} of $G$ by $H$, equipped with the operation $\star_{/H}$ defined in Equation \eqref{eq:quotientOperation}.
\end{thm}

\smallGap
\begin{UNABexe}\label{exe:quotientGroup}
Prove Theorem \ref{thm:quotientGroup}.
\end{UNABexe}

\smallGap
The alternate notation $\Z/n\Z$ to $\Z_n$ arises exactly because it is a quotient group.  A more rigorous statement is given in Section \ref{sec:2p8firstIsomorphismTheorem}. 
\end{frame}

% % % 
\begin{frame}[c]
\begin{exe}[cf. Problems 69-70]\label{exe:probs69-70}
Write down the addition table for $G/H$ in 
\begin{enumerate}[(a)]
\item Exercise \ref{exe:prob66}.
\item Exercise \ref{exe:prob67}.
\end{enumerate}
\end{exe}
\end{frame}

% % %
\begin{frame}[c]
We saw in Example \ref{ex:D3} the subgroup $\langle s\rangle< D_3$ is not normal.  As a consequence, the operation $\star$ defined in Equation \eqref{eq:quotientOperation} is not well-defined on its cosets.

\smallGap
Take $r\in r\langle s\rangle$ and $r^2\in r^2\langle s\rangle$.  Since $(r)(r^2)=e$, we should have 
\[
r\langle s\rangle \star r^2\langle s\rangle.
\]
On the other hand, take $rs\in r\langle s\rangle$ and $r^2s\in r^2\langle s\rangle$.  Then 
\[
(rs)(r^2s)=r^2\in r^2\langle s\rangle\neq \langle s\rangle.
\]
\end{frame}

% % % % %
\subsection[\subsecname]{Non-obvious isomorphisms}
% % %
\begin{frame}{\subsecname}
\begin{ex}\label{ex:isoToZ6}
Let $G=\Z\times \Z$ and define 
\begin{align*}
H=(3,0)G+(0,2)G &= \{m(3,0)+n(0,2)\mid m,n\in \Z\} \\
	&= \{(3m,2n)\mid m,n\in \Z\}.
\end{align*}
\end{ex}
Think of the elements in $H$ as movements on a grid indexed by $\Z\times \Z$.  The generator $(3,0)$ is right by 3; the generator $(0,2)$ is up by 2.  

\smallGap 
The cosets of $H$ in $G$ are: 
\smallGap
\[
\begin{aligned}
\die{6} &:=(0,0)+H=H \\
\die{1} &:=(1,1)+H \\
\die{2} &:=(2,0)+H
\end{aligned} \qquad	
\begin{aligned}
\die{3} &:=(0,1)+H \\
\die{4} &:=(1,0)+H \\
\die{5} &:=(2,1)+H
\end{aligned}
\]
\end{frame}

% % %
\begin{frame}[c]
Compare the addition table for $G/H$ to the one for $\Z/6\Z$:
\[
\begin{array}{c | c c c c c c }
 G/H  & \die{6} & \die{1} & \die{2} & \die{3} & \die{4} & \die{5} \\
 \hline
 \die{6} & \die{6} & \die{1} & \die{2} & \die{3} & \die{4} & \die{5} \\ 
 \die{1} & \die{1} & \die{2} & \die{3} & \die{4} & \die{5} & \die{6} \\
 \die{2} & \die{2} & \die{3} & \die{4} & \die{5} & \die{6} & \die{1} \\
 \die{3} & \die{3} & \die{4} & \die{5} & \die{6} & \die{1} & \die{2} \\
 \die{4} & \die{4} & \die{5} & \die{6} & \die{1} & \die{2} & \die{3} \\
 \die{5} & \die{5} & \die{6} & \die{1} & \die{2} & \die{3} & \die{4}
\end{array}
\qquad
\begin{array}{c | c c c c c c }
 \Z/6\Z  & 0 & 1 & 2 & 3 & 4 & 5 \\
 \hline
 0 & 0 & 1 & 2 & 3 & 4 & 5 \\ 
 1 & 1 & 2 & 3 & 4 & 5 & 0 \\
 2 & 2 & 3 & 4 & 5 & 0 & 1 \\
 3 & 3 & 4 & 5 & 0 & 1 & 2 \\
 4 & 4 & 5 & 0 & 1 & 2 & 3 \\
 5 & 5 & 0 & 1 & 2 & 3 & 4
\end{array}
\]	

\smallGap
\textbf{Conclusion:} $G/H\cong \Z/6\Z$ via the correspondence in the addition tables.
\end{frame}

% % %
%\begin{frame}
%\begin{ex}
%The elements in $\Z^2/\langle(2,-2),(2,2)\rangle$, i.e., the cosets of $H:=\langle (2,-2),(2,2)\rangle$ are 
%\end{ex}
%\centering
% \begin{tikzpicture}[scale=0.5]
%  \coordinate (Origin)   at (0,0);
%    \coordinate (XAxisMin) at (-5,0);
%    \coordinate (XAxisMax) at (5,0);
%    \coordinate (YAxisMin) at (0,-5);
%    \coordinate (YAxisMax) at (0,5);
%    \draw [thin, gray,-latex] (XAxisMin) -- (XAxisMax);% Draw x axis
%    \draw [thin, gray,-latex] (YAxisMin) -- (YAxisMax);% Draw y axis
%
%    \clip (-5,-5) rectangle (5cm,5cm); % Clips the picture...
%    %\pgftransformcm{1}{0.6}{0.7}{1}{\pgfpoint{0cm}{0cm}}
%          % This is actually the transformation matrix entries that
%          % gives the slanted unit vectors. You might check it on
%           % MATLAB etc. . I got it by guessing.
%    \coordinate (Bone) at (2,2);
%    \coordinate (Btwo) at (2,-2);
%    \draw[style=help lines,dashed] (-4,-4) grid[step=1cm] (4,4);
%          % Draws a grid in the new coordinates.
%          %\filldraw[fill=gray, fill opacity=0.3, draw=black] (0,0) rectangle (2,2);
%              % Puts the shaded rectangle
%    \foreach \x in {-7,-6,...,7}{% Two indices running over each
%      \foreach \y in {-7,-6,...,7}{% node on the grid we have drawn 
%        \node[draw,circle,inner sep=2pt,fill] at (2*\x+2*\y,2*\x-2*\y) {};
%            % Places a dot at those points
%      }
%    }
%    %\draw [ultra thick,-latex,red] (Origin)
%     %   -- (Bone) node [above left] {$b_1$};
%    %\draw [ultra thick,-latex,red] (Origin)
%     %   -- (Btwo) node [below right] {$b_2$};
%    %\draw [thin,-latex,red, fill=gray, fill opacity=0.3] (0,0)
%        % -- ($2*(0,2)+(2,-2)$)
%        % -- ($3*(0,2)+2*(2,-2)$) -- ($(0,2)+(2,-2)$) -- cycle;
%  \end{tikzpicture}`
%\end{frame}

% % %
\begin{frame}[c]
\begin{exe}[cf. Problem 72]\label{exe:prob72}
``Simplify" the following \vocab{group presentations} (a term we define in Section \ref{sec:2p10FTOFinitelyGeneratedAbelianGroups}) by exhibting an isomorphism in each case.
\begin{enumerate}
\item $\Z\times \Z/\langle(1,1)\rangle$
\item\label{exept:prob72-2} $\Z\times \Z/\langle(2,-1),(-1,2)\rangle$
\end{enumerate} 
\end{exe}
\end{frame}

% % % % %
\subsection[\subsecname]{Quotient means kill}\label{subsec:quotientMeansKill}
% % %
\begin{frame}[c]{\subsecname}
We saw in Example \ref{ex:ZmodnZ}, the (group) homomorphism
\begin{align*}
\varphi : \Z &\to \Z/n\Z \\
m &\mapsto m\mod n
\end{align*}
is surjective, and $\ker{\varphi}=n\Z$.  We say $\varphi$ \vocab{kills} $n$.  
\end{frame}

% % %
\begin{frame}[c]
Generalizing this notion:
\begin{UNABexe}[cf. Problem 73]\label{exe:prob73}
Prove that if $H\norml G$ then the natural group homomorphism 
\begin{align*}
\bar{\psi}: G &\to G/H \\
	g &\mapsto gH
\end{align*}
is surjective.  Colloquially, when we apply $\bar{\psi}$ we say we are \emph{killing} the subgroup $H$. 
\end{UNABexe}

\smallGap
Often, when defining quotient maps, we use a bar.
\end{frame}

% % % % %
\answerKey
% % %
\begin{frame}{\subsecname}
\begin{block}{Exercise*}\end{block}
\vspace{-0.75pc}
\textbf{Solution:} Let $\varphi:G\to H$ denote a group homomorphism with $K=\ker\varphi$.  We wish to show $gK=Kg$ for all $g\in G$, in fact, equivalently, $gKg\1=K$ for all $g\in G$.  Suppose $k\in K$.  Put $k'=g\1kg$.  Then $k=gk'g\1\in gKg\1$, so $K\subseteq gKg\1$.  For the other containment, say $gkg\1\in gKg\1$ for some $k\in K$.  Then 
\[
\varphi(gkg\1)=\varphi(g)\varphi(k)\varphi(g\1)=\varphi(g)\cdot 1_H\cdot \varphi(g\1)=\varphi(gg\1)=1_H.
\]
It follows that $gkg\1\in K$ and therefore $gKg\1=K$.
\qed

\smallGap
\begin{block}{Exercise*}\end{block}
\vspace{-0.75pc}
\textbf{Solution:}  Suppose $G$ is abelian and $g\in G$.  Let $H$ denote a subgroup of $G$.  Then for all $h\in H$, $gh=hg$ so $gH=Hg$.
\qed
\end{frame}

% % %
\begin{frame}
\exeSol[(cf. Problem 66)]{exe:prob66}
The elements in $G/H$ are the cosets of $H=5G$:
\[
\begin{aligned}
H &= \{0,5,10,15,20,25\} \\
1\oplus_{30}H &= \{1,6,11,16,21,26\} \\
2\oplus_{30}H &= \{2,7,12,17,22,27\}
\end{aligned} \quad
\begin{aligned}
3\oplus_{30}H &= \{3,8,13,18,23,28\} \\
4\oplus_{30}H &= \{4,9,14,19,24,29\}
\end{aligned} 
\]
By Proposition \ref{prop:cosetsPartition}, since we have already partitioned the elements in $G=\Z_{30}$, there are no other distinct cosets.

\smallGap
\exeSol[(cf. Problem 67)]{exe:prob67}
The elements in $H$ are
\[
\langle (1,1)\rangle = \{(0,0),(1,1),(0,2),(1,3)\}.
\]
There are two cosets of $H$:
\begin{align*}
H &= \{(0,0),(1,1),(0,2),(1,3)\} \\
(1,0)+H &= \{(1,0),(0,1),(1,2),(0,3)\}
\end{align*}
\end{frame}

% % %
\begin{frame}
\exeSol[(cf. Problem 68)]{exe:prob68}
\begin{itemize}
\item[(a)] Suppose $g\star H=H$.  $H$ is a subgroup and so contains the identity, $e$, and so $g=g\star e \in g\star H=H$.  

\smallGap
Conversely, suppose $g\in H$.  We know $g\star H\cap H$ contains $g\star e=g$.  By Proposition \ref{prop:cosetsPartition} the cosets must be equal.

\smallGap
\item[(b)] Suppose $g\star H=h\star H$.  Multiplying on the left, 
\begin{align*}
g\1\star(g\star H) &= g\1\star(h\star H) \\
\implies H &= (g\1h)\star H. 
\end{align*}
By part {\usebeamercolor[fg]{block title}(a)}, it follows that $g\1h\in H$.

\smallGap
On the other hand, suppose $g\1h\in H$.  Multiplying on the left,
\[
g\star g\1h=h\in g\star H,
\]
while $h=h\star e\in h\star H$.  It follows that $g\star H\cap h\star H\neq \emptyset$ and by Proposition \ref{prop:cosetsPartition}, the two cosets are equal.
\end{itemize}
\end{frame}

% % %
\begin{frame}[c]
\begin{itemize}
\item[(c)] By left multiplication by $g$, the statements $g\1h\in H$ and $h\in g\star H$ are equivalent.  Part {\usebeamercolor[fg]{block title}(b)} says $g\1h\in H$ is equivalent to $g\star H=h\star H$.  Therefore $h\in g\star H$ if and only if $g\star H=h\star H$.
\end{itemize}
\qed
\end{frame}

% % %
\begin{frame}
\begin{block}{Exercise*}\end{block}
\vspace{-0.75pc}
\textbf{Solution:}
The operation $\star_{/H}$ is constructed to be binary.  We check associativity:
\begin{align*}
\left((g_1\star H)\star_{H}(g_2\star H)\right)\star_{H}(g_3\star H) &= \left((g_1\star g_2)\star H\right)\star_H(g_3\star H) \\
	&= \left((g_1\star g_2)\star g_3\right)\star H \\
	&= \left(g_1\star (g_2\star g_3)\right)\star H \\
	&= (g_1\star H)\star_H\left((g_2\star g_3)\star H\right) \\
	&= (g_1\star H)\star_H\left((g_2\star H)\star_H(g_3\star H)\right)
\end{align*}

Let $e$ denote the identity in $G$.  The identity element in $G/H$ is $H= e\star H$.  To see why, take $g\star H\in G/H$.  Then
\begin{align*}
(g\star H)\star_H (e\star H) &= (g\star e)\star_H H = g\star H; \\
(e\star H)\star_H (g\star H) &= (e\star g)\star_H H = g\star H.
\end{align*}

\smallGap
Finally, given a coset $g\star H\in G/H$, its inverse is $g\1\star H$:
\begin{align*}
(g\star H)\star_H(g\1\star H) &= (g\star g\1)\star H = e\star H; \\
(g\1 \star H)\star_H(g\star H) &= (g\1 \star g)\star H = e\star H 
\end{align*}
\qed
\end{frame}

% % %
\begin{frame}
\exeSol[(cf. Problems 69-70)]{exe:probs69-70}
\begin{itemize}
\item[(a)] $
\begin{array}[t]{c | c c c c c }
 \Z_{30}/\langle 5\rangle  & H & 1\oplus_{30}H & 2\oplus_{30} H & 3\oplus_{30}H & 4\oplus_{30}H \\
 \hline
 H & H & 1\oplus_{30}H & 2\oplus_{30}H & 3\oplus_{30}H & 4\oplus_{30}H \\ 
 1\oplus_{30}H & 1\oplus_{30}H & 2\oplus_{30}H & 3\oplus_{30}H & 4\oplus_{30}H & H \\
 2\oplus_{30}H & 2\oplus_{30}H & 3\oplus_{30}H & 4\oplus_{30}H & H &1\oplus_{30}H \\
 3\oplus_{30}H & 3\oplus_{30}H & 4\oplus_{30}H & H & 1\oplus_{30}H & 2\oplus_{30}H \\
 4\oplus_{30}H & 4\oplus_{30}H & H & 1\oplus_{30}H & 2\oplus_{30}H & 3\oplus_{30}H 
\end{array}
$

\smallGap
\item[(b)]$
\begin{array}[t]{ c | c c }
\Z_2\times \Z_4/\langle (1,1)\rangle & H & (1,0)+H \\
\hline
H & H & (1,0)+H \\
(1,0)+H & (1,0)+H & H
\end{array}
$
\end{itemize}
\end{frame}

% % %
\begin{frame}
\exeSol[(cf. Problem 72)]{exe:prob72}
\begin{itemize}
\item[(a)] On the $\Z\times \Z$ lattice, the elements in $\langle (1,1)\rangle$ all lie on the line $y=x$.  The each coset consists of points on a line parallel to $y=x$.  Each coset crosses the $x$-axis in exactly one point, so identify each coset with the $x$-coordinate where it crosses the axis.  This gives a one-to-one correspondence with $\Z$.

\smallGap
\item[(b)] We can identify the cosets of $H=\langle (2,-1),(-1,2)\rangle$ by drawing an addition table.
\[
\begin{array}{ c | c c c }
\Z\times \Z/\langle (2,-1),(-1,2)\rangle & H & (1,0)+H & (0,1)+H \\
\hline
H & H & (1,0)+H & (0,1)+H \\
(1,0)+H & (1,0)+H & (0,1)+H & H \\
(0,1)+H & (0,1)+H & H & (1,0)+H
\end{array}
\]
Any other $\Z$-linear combination of the cosets listed above will not result in a new coset.  We see the quotient group $\Z\times \Z/\langle (2,-1),(-1,2)\rangle$ has three elements; we define the isomorphism $\Z\times\Z/\langle (2,-1),(-1,2)\rangle\xrightarrow{\cong} \Z_3$ by 
\[
H \mapsto 0 \qquad
(1,0) \mapsto 1 \qquad 
(0,1) \mapsto 2.
\]
\end{itemize}
\end{frame}

% % %
\begin{frame}
\begin{block}{Exercise*}\end{block}
\vspace{-0.75pc}
\textbf{Solution:} Take a coset $gH\in G/H$.  Then $\bar{\psi}(g)=gH$.
\qed
\end{frame}

% % % % % % % % % %
\end{document}