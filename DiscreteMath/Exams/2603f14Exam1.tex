\documentclass[11pt,letterpaper]{article}
\input{../../preamble}
\usepackage{fullpage}
\usepackage{multicol}
\everymath{\displaystyle}

\begin{document}
\flushleft
\begin{multicols}{2}

\begin{large}\textbf{Math 2603 Exam 1 \\
Wed 24 Sep 2014}\end{large}

\hfill\textbf{Name:  }\underline{\hspace{40ex}} %KEY\hspace{17ex}} 
\\
\vspace{.5in}

\end{multicols}

\pagestyle{empty}

\vspace{4pc}

\begin{center}\LARGE Discrete Math

Exam 1 (Ch. 1-2: Set theory, logic, proofs) \end{center}

\vspace{2pc}
Please provide the following data:

\vspace{2pc}
Drill Time: \underline{\hspace{40ex}}

\vspace{2pc}
Student ID: \underline{\hspace{40ex}}

\vspace{3pc}
{\bf Exam Instructions:} You have 50 minutes to complete this exam.  One $3\times 5$ inch notecard is allowed.  No graphing calculators.  No programmable calculators.  No phones, iDevices, computers, etc.  If you finish early then you may leave, UNLESS there are less than 5 minutes of class left.  To prevent disruption, if you finish with less than 5 minutes of class remaining then please stay seated and quiet.

\vspace{5pc}
Your signature below indicates that you have read this page and agree to follow the Academic Honesty Policies of the University of Arkansas.  

\vspace{3pc}
Signature: \underline{\hspace{80ex}}

\vfill
\begin{flushright}\Large Good luck!\end{flushright}

\begin{enumerate}
\newpage

\item Truth tables and valid arguments.

\begin{enumerate}

% % % % %
\vspace{1pc}
\item Fill in the truth table:

\vspace{1pc}
$\begin{array}{c | c | c| c | c|}
P & Q & R & P\wedge Q & (P\wedge Q)\to (\neg R) \\
\hline
T & T & T & & \\
\hline
T & T & F & & \\
\hline
T & F & T & & \\
\hline
T & F & F & & \\
\hline
F & T & T & & \\
\hline
F & T & F & & \\
\hline
F & F & T & & \\
\hline
F & F & F & & \\
\hline
\end{array}$

\vspace{2pc}
\item Using table (a) explain if the argument below is valid. 
\[\begin{array}{l}
\quad P\wedge Q \\
\quad Q \\
\hline
\therefore (P\wedge Q)\to (\neg R)
\end{array}\]

\vspace{5pc}
\item Determine if the following argument is valid. You may extend the table in (a) if necessary, but either way, you must justify your answer.

\[\begin{array}{l}
\quad P\wedge Q\to\neg R \\
\quad P\vee\neg Q \\
\quad \neg Q\to P \\
\hline
\therefore R
\end{array}\]
\end{enumerate}

% % % % %
\newpage
\item Let $P=$  ``If $\frac{6}{d}\in\mathbb Z$, where $d\in\mathbb Z$, then $d=3$."
\begin{enumerate}

\vspace{1pc}
\item Is $P$ a proposition?

\vspace{4pc}
\item What is $\neg P$?

\vspace{5pc}
\item Give the contrapositive of $P$.

\vspace{5pc}
\item Give the converse of $P$.

\vspace{5pc}
\item Is $P$ true or false? (Prove or give a counterexample.)
\end{enumerate}

% % % % %
\newpage
\item For a set $S$ let $\mathscr P(S)$ denote its power set.
\begin{enumerate}
\item Prove $\mathscr P(X\cap Y)=\mathscr P(X)\cap\mathscr P(Y)$.

\vspace{30pc}
\item Disprove $\mathscr P(X\cup Y)\subseteq\mathscr P(X)\cup\mathscr P(Y)$.
\end{enumerate}

% % % % %
\newpage
\item Prove by contrapositive: If $x^2\in\mathbb R\setminus\mathbb Q$, then $x\in\mathbb R\setminus\mathbb Q$.

% % % % %
\newpage
\item True/False.  If true, use induction to prove it.  If false, give a counterexample.
\begin{enumerate}
\item $1^2-2^2+3^2-4^2+\cdots+(-1)^{n+1}n^2=\frac{(-1)^{n+1}n(n+1)}{2}$

% % % % %
\newpage
\item For all $n\in\mathbb Z_{>0}$, $11^n-6$ is divisible by $5$.
\end{enumerate}

% % % % %
\newpage
\item {\bf cHaLlEnGe PrObLeM} Prove: Given any two rational numbers $r$ and $s$ with $r<s$, there is a rational number between $r$ and $s$.

% % % % %
\newpage
\item {\bf EXTRA CREDIT} Use the quotient-remainder theorem, with $d=3$, to prove that the product of any three consecutive integers is divisible by $3$.

\end{enumerate}
\end{document}