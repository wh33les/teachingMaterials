\documentclass[12pt,letterpaper]{article}
\input{../../preamble}
\usepackage{fullpage}
\usepackage{multicol}
\everymath{\displaystyle}

\begin{document}
\flushleft
\begin{multicols}{2}

\begin{large}\textbf{Math 2603 Final \\
due Wed 17 Dec 2014}\end{large}

\hfill\textbf{Name:  }\underline{\hspace{35ex}} %KEY\hspace{17ex}} 
\\
\vspace{.5in}

\end{multicols}

\pagestyle{empty}

\vspace{4pc}

\begin{center}\LARGE Discrete Math Final \end{center}

\vspace{2pc}
Please provide the following data:

\vspace{2pc}
Drill Time: \underline{\hspace{40ex}}

\vspace{2pc}
Student ID: \underline{\hspace{40ex}}

\vspace{3pc}
{\bf Exam Instructions:} %You have 50 minutes to complete this exam.  One $3\times 5$ inch notecard, two-sided, is allowed.  No graphing calculators.  No programmable calculators.  No phones, iDevices, computers, etc.  If you finish early then you may leave, UNLESS there are less than 5 minutes of class left.  To prevent disruption, if you finish with less than 5 minutes of class remaining then please stay seated and quiet.

\vspace{5pc}
Your signature below indicates that you have read this page and agree to follow the Academic Honesty Policies of the University of Arkansas.  

\vspace{3pc}
Signature: \underline{\hspace{79ex}}

\vfill
\begin{flushright}\Large Good luck!\end{flushright}

\begin{enumerate}

% % % % %
\newpage
\item \begin{enumerate}\item Prove or disprove:  for all sets $A,B,C$, 
\[(A\setminus B)\cup(C\setminus B)=(A\cup C)\setminus B.\]

\vspace{23pc}
\item Prove: For any integer $n$, if $n$ is even then $n^2$ is even.
\end{enumerate}

% % % % %
\newpage
\item Prove or disprove: For all nonnegative integers $n$ and $r$ with $r+1\leq n$, 
\[\binom{n}{r+1}=\frac{n-r}{r+1}\binom{n}{r}.\]

% % % % %
\newpage
\item Prove the following statements:
\begin{enumerate}
\item The product of any nonzero rational number and any irrational number is irrational.

\vspace{23pc}
\item Use the previous result and the fact that $\sqrt 3$ is irrational to show that $\sqrt{75}$ is irrational.

\end{enumerate}

% % % % %
\newpage
\item Use induction to prove $6\cdot 7^n-2\cdot 3^n$ is divisible by 4, for all $n\geq 1$.  

% % % % %
\newpage
\item Let I denote the following relation on $\mathbb R$:
\[\text{For all $x,y\in\mathbb R$, $x$I$y$ if and only if $x-y$ is an integer.}\]
Prove or disprove:
\begin{enumerate}
\item I is reflexive.

\vspace{6pc}
\item I is symmetric.

\vspace{6pc}
\item I is antisymmetric.

\vspace{6pc}
\item I is transitive.

\vspace{6pc}
\item I is an equivalence relation.

\vspace{6pc}
\item I is a partial order.
\end{enumerate}

% % % % %
\newpage
\item Let $P,Q,R$ denote propositions.  Prove or disprove: 
\[(P\to Q)\to R\equiv P\vee(R\to Q).\]

% % % % %
\newpage
\item Prove that for all integers $n\geq 1$,
\[\sum_{k=1}^{3n}(4k+3)=3n(6n+5).\]

% % % % %
\newpage
\item Suppose the midterm consists of 12 distinct, prewritten questions, where five are easy, three are hard, and four are medium.
\begin{enumerate}
\item How many ways can the questions be ordered so that questions 1 and 2 are both hard?

\vspace{9pc}
\item How many ways can the questions be ordered so that all the easy ones are first, then the medium ones, then the hard ones?

\vspace{9pc} 
\item How many ways can the questions be ordered so that none of the first 5 questions are hard?

\vspace{9pc}
\item How many ways can the questions be ordered so that no two consecutive questions are easy?
  
\end{enumerate}

% % % % %
\newpage
\item \begin{enumerate}\item Draw an example of a connected graph of order 6 that has both an Eulerian cycle that is not Hamiltonian, and a Hamiltonian cycle that is not Eulerian. 

\vspace{23pc}
\item Suppose $S,T$ are sets with $|S|=10$ and $|T|=8$.  Which is larger, $|\mathscr P(S\times T)|$ or $|S\times\mathscr P(T)|$ and why?
\end{enumerate}
% % % % %
\newpage
\item For each of the following functions the domain is $\mathbb Z\times\mathbb Z$ and the codomain is $\mathbb Z$.  Determine whether each function is one-to-one, onto, or both.  You must prove your answers.  
\begin{enumerate}
\item $f(m,n)=m-n$

\vspace{15pc}
\item $f(m,n)=mn$

\vspace{15pc}
\item $f(m,n)=n^2+1$
\end{enumerate}

% % % % %
%\newpage
%\item Determine whether or not each of the following graphs\footnote{Image Credits: Johnsonbaugh, Richard. \emph{Discrete Mathematics}, 7th Edition.  p. 406.} has a Hamiltonian cycle.  If it does, then exhibit it.  If it does not, then explain why.
%\begin{enumerate}
%\item 
%\begin{wrapfigure}{c}
%\includegraphics*[scale=0.6]{final7.png}
%\centering
%\end{wrapfigure}

%\vspace{10pc}
%\item 

%\vspace{-1pc}  
%\begin{figure}[h]
%\begin{center}
%\includegraphics*[scale=0.5]{final8.png}
%\caption{Example 8.3.3 in the text}\label{fig:Ham}
%\end{center}
%\end{figure}
%\end{enumerate}

\end{enumerate}
\end{document}