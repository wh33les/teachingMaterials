\input G:/Dropbox/Math/plainpreamb.tex
%%%%%%%%%%%%%%%%%%%%%%%%%%%%%%%%%%%%%%%%%%%%%%%%%%%%%%%%%
       %%% The document starts here (WYSIWYG). %%%
%%%%%%%%%%%%%%%%%%%%%%%%%%%%%%%%%%%%%%%%%%%%%%%%%%%%%%%%%

%\NoPageNumbers
\headline{\ifnum\pageno=1{}\else\centerline{\rm DISCRETE MATHEMATICS}\fi}
\rightheadline{\ifnum\pageno=1{}\else\centerline{\rm DISCRETE MATHEMATICS}\fi}
\nologo         
\parindent = 20 pt
\centerline{\bf DISCRETE MATHEMATICS }\hfill\break
\centerline{(lecture notes for Math 2603 at the University of Arkansas)}\hfill
\centerline{Fall 2014}\hfill
\centerline{by Ashley K. Wheeler}\hfill
\centerline{\it Last modified: \today}\hfill

\bigskip
\centerline{\bf Sets and Logic}
\medskip

\proclaim{Definition} A {\bf set} is a collection of objects called elements or members; order is not taken into account. \endproclaim

\ex $$A=\{a,b,c,d\}$$
is a set.  Its elements are $a,b,c,\text{ and }d$.\endex

Although the definition for set is very generic and vague, there are many less obvious examples of sets:

\ex \endex

\proclaim{Definition} Suppose $A$ is a set.  The non-negative integer 
$$|A|=\text{number of elements in $A$}$$ 
is called the {\bf cardinality} of $A$. \endproclaim

\proclaim{Definition} The set with no elements is called the {\bf empty set} (also the {\bf null set}, or the {\bf void set}) and is denoted $\emptyset$, or $\left\{\right\}$. \endproclaim

\proclaim{Definition} Two sets $X$ and $Y$ are {\bf equal} means $X$ and $Y$ have the same cardinality.  We write $X=Y$. \endproclaim

\proclaim{Definition} Suppose $X,Y$ are sets.  $X$ is a {\bf subset} of $Y$ means every element of $X$ is an element of $Y$.  We write $X\subseteq Y$. \endproclaim

\proclaim{Definition} The set of all subsets of a set $X$ is called the {\bf power set} of $X$, denoted $\Cal P(X)$, or $2^X$. \endproclaim

\proclaim{Definition} Suppose $X\subseteq Y$.  $X$ is a {\bf proper subset} of $Y$ means, in addition, that $X$ does not equal $Y$.  We write $X\subset Y$.  Note, some authors write $X\subsetneq Y$ to emphasize non-equality. \endproclaim

\proclaim{Definition} Let $X,Y$ denote sets.  The set
$$X\cup Y=\left\{x\,|\,x\in X\text{ or }x\in Y\right\}$$
is called the {\bf union} of $X$ and $Y$. \endproclaim

\proclaim{Definition} Let $X,Y$ denote sets.  The set
$$X\cap Y=\left\{x\,|\,x\in X\text{ and }x\in Y\right\}$$
is called the {\bf intersection} of $X$ and $Y$. \endproclaim

\proclaim{Definition} Let $X,Y$ denote sets.  The set
$$X\setminus Y=\left\{x\,|\,x\in X\text{ and }x\notin Y\right\}$$
is called the {\bf difference} or {\bf relative complement}.  Some authors use $X-Y$. \endproclaim

\proclaim{Definition} The set $\Bbb R\setminus\Bbb Q$ is called the set of {\bf irrational numbers}. \endproclaim

\proclaim{Definition} Sets $X$ and $Y$ are {\bf disjoint} means $X\cap Y=\emptyset$. \endproclaim

\proclaim{Definition} A collection of sets $\Cal S$ is {\bf pairwise disjoint} means any two distinct sets in $\Cal S$ are disjoint. \endproclaim

\proclaim{Definition} A {\bf universal set} or {\bf universe} is a set, usually inferred via context, whose subsets are those we are considering. \endproclaim

\proclaim{Definition} Given a universal set $U$ with $X\subseteq U$, the set 
$$\bar X=U\setminus X$$ 
is called the {\bf complement} of $X$ in $U$. \endproclaim

\proclaim{Definition} A {\bf Venn diagram} is a pictorial view of sets drawn as follows: A rectangle depicts the universal set.  Subsets of the universal set are drawn as circles.  The inside of a circle represents the members of that set. \endproclaim

\proclaim{Theorem} Let $U$ be a universal set and let $A,B,C\subseteq U$. 
\settabs 2 \columns
\+ Associative Laws: & $(A\cup B)\cup C=A\cup(B\cup C)$ \cr
\+ & $(A\cap B)\cap C=A\cap(B\cap C)$ \cr
\+ Commutative Laws: & $A\cup B=B\cup A$ \cr
\+ & $A\cap B=B\cap A$ \cr
\+ Distributive Laws: & $A\cap(B\cup C)=(A\cap B)\cup(A\cap C)$ \cr
\+ & $A\cup(B\cup C)=(A\cup B)\cap(A\cup C)$ \cr
\+ Identity Laws: & $A\cup\emptyset=A$ \cr
\+ & $A\cap U=A$ \cr
\+ Complement Laws: & $A\cup\bar A=U$ \cr
\+ & $A\cap\bar A=\emptyset$ \cr
\+ Idempotent Laws: & $A\cup A=A$ \cr
\+ & $A\cap A=A$ \cr
\+ Bound Laws: & $A\cup U=U$ \cr
\+ & $A\cap\emptyset=\emptyset$ \cr
\+ Absorption Laws: & $A\cup(A\cap B)=A$ \cr
\+ & $A\cap(A\cup B)=A$ \cr
\+ Involution Laws: & $\bar{\bar A}=A$ \cr
\+ $0/1$ Laws: & $\bar{\emptyset}=U$ \cr
\+  & $\bar U=\emptyset$ \cr
\+ De Morgan's Laws: & $\bar{(A\cup B)}=\bar A\cap\bar B$ \cr
\+ & $\bar{(A\cap B)}=\bar A\cup\bar B$ \cr
\endproclaim
 
\pf Left as an exercise. \qed 

\proclaim{Definition} A collection $\Cal S$ of non-empty sets of $X$ is a {\bf partition} of the set $X$ means every element in $X$ belongs to exactly one member of $\Cal S$. \endproclaim

\proclaim{Definition} An {\bf ordered pair} of elements, written $(a,b)$ is considered distinct from the ordered pair $(b,a)$, unless $a=b$. \endproclaim

\proclaim{Definition} Say $X,Y$ are sets.  The set of ordered pairs
$$X\times Y=\left\{(x,y)\st x\in X\text{ and }y\in Y\right\}$$
is called the {\bf Cartesian product} of $X$ and $Y$. \endproclaim

\proclaim{Definition} Ordered lists need not be restricted to two elements.  An {\bf $n$-tuple}, written $(a_1,\dots,a_n)$ takes order into account. \endproclaim

\proclaim{Definition} A \nz integer $d$ {\bf divides} an integer $m$ means there exists an integer $q$, called the {\bf quotient}, such that $m=dq$. A positive integer is {\bf prime} means the only positive integers that divide it are 1 and itself. \endproclaim

%%%%%%%%%%%%%%%%%%%%%%%%%
%%%%%%%%%%%%%%%%%%%%%%%%

\proclaim{Definition} A sentence that is either true of false, but not both, is called a {\bf proposition}. \endproclaim

\proclaim{Definition} Suppose $P,Q$ are propositions.  The {\bf conjunction} of $P$ and $Q$ is \endproclaim

%%%%%%%%%%%%%%%%%%%%%%%%%%
%%%%%%%%%%%%%%%%%%%%%%%%%

\bigskip
\noindent{\bf Conditional Propositions and Logical Equivalence}
\bigskip

\proclaim{Definition} Suppose $P,Q$ are propositions.  The statement ``if $P$ then $Q$" is called a {\bf conditional proposition}, denoted $P\to Q$. $P$ is called the {\bf hypothesis}, or {\bf antecedent}.  $Q$ is called the {\bf conclusion}, or {\bf consequent}. \endproclaim

\medskip
\centerline{Truth Tables}
\settabs 3 \columns
\+{$P$} & {$Q$} & {$P\to Q$} \cr
\+ true & true & true \cr
\+ false & true & true \cr
\+ false & false & true \cr
\+ true & false & false \cr

When a conditional has a false antecedent, its truth value is always true.  For this reason we sometimes say $P\to Q$ is {\it vacuously true} or {\it true by default}.  In the order of operations, $\rightarrow$ is evaluated last.

Truth values can be difficult to parse when propositions are expressed in everyday conversation.  The following statements mean the same thing:
\numberedlist
\li ``if $P$, then $Q$"
\li $P\to Q$
\li ``$Q$ only if $P$"
\li ``When $P$, $Q$."
\li ``If not $Q$, then not $P$." (called the {\it contrapositive} to the conditional proposition $P\to Q$)
\li $Q$ is necessary for $P$.
\li $P$ suffices for $Q$.
\endnumberedlist

\proclaim{Definition} Suppose $R=P\to Q$.  The {\bf converse} of $R$ is $Q\to P$. \endproclaim

Warning, the converse of a conditional and the contrapositive of a conditional are NOT THE SAME THING.

\proclaim{Definition} A {\bf biconditional proposition}, denoted $P\leftrightarrow Q$, is defined by the truth table
\+ $P$ & $Q$ & $P\leftrightarrow Q$ \cr
\+ true & true & true \cr
\+ true & false & false \cr
\+ false & true & false \cr
\+ false & false & true \cr
\endproclaim 

Some equivalent statements:
\numberedlist
\li ``$P\leftrightarrow Q$" 
\li ``$P$ if and only if $Q$"
\li ``$P$ is necessary and sufficient for $Q$."
\li ``$Q$ is necessary and sufficient for $P$."
\li ``$Q\leftrightarrow P$"
\li ``$P$ iff $Q$"
\endnumberedlist

Defining a proposition via truth table remedies the ambiguity that comes with trying to express a logical statement in lay-speak.  Truth tables are very effective tools in writing proofs. 

\proclaim{Definition} Two propositions are {\bf logically equivalent} means their truth tables are the same.  The symbol for logical equivalence is $\equiv$. \endproclaim

\ex De Morgan's Laws:
\numberedlist
\li $\lnot\left(P\vee Q\right)\equiv\lnot P\wedge\lnot Q$
\li $\lnot\left(P\wedge Q\right)\equiv\lnot P\vee\lnot Q$
\endnumberedlist
\pf In each case, the propositions we are trying to show are logically equivalent are the proposition on the lefthand side (LHS) of $\equiv$ and the proposition on the righthand side (RHS) or $\equiv$. 
\numberedlist
\li \settabs 4 \columns
\+ $P$ & $Q$ & $\lnot\left(P\vee Q\right)$ & $\lnot P\wedge\lnot Q$ \cr
\+ true & true & false & false \cr
\+ true & false & false & false \cr
\+ false & true & false & false \cr
\+ false & false & true & true \cr
\li
\endnumberedlist

\qed  


%\bigskip
%%% Bibliography %%%
%\Refs\nofrills{Bibliography}
%\widestnumber\key{SwMn}

%\ref\key AtyM
%\manyby M.F. Atiyah and I.G. MacDonald
%\book Introduction to Commutative Algebra
%\bookinfo Advanced Book Program
%\publ Westview Press, A Member of the Perseus Books Group
%\publaddr Boulder, Colorado
%\yr 1969
%\endref 

%\ref\key BrH
%\manyby W. Bruns and J. Herzog
%\book Cohen-Macaulay Rings
%\bookinfo Cambridge Studies in Advanced Mathematics 39
%\publ Cambridge University Press
%\publaddr Cambridge
%\yr 1993
%\endref 

%\ref\key CaMa
%\manyby G. Caviglia and D. Maclagan
%\paper Some Cases of the Eisenbud-Green-Harris Conjecture
%\yr 2007
%\jour arXiv?

%\ref\key ClLi
%\manyby G.F. Clements and B. Lindstr\"om 
%\paper A Generalization of a Combinatorial Theorem of Macaualay
%\jour Journal of Combinatorial Theory
%\vol 7
%\pages 230-238
%\yr 1969
%\endref

%\ref\key Coop
%\manyby S.M. Cooper
%\paper The Eisenbud-Green-Harris Conjecture for Ideals of Points
%\jour ???
%\yr 2008
%\endref

%\ref\key CoRo
%\manyby S.M. Cooper and L.G. Roberts
%\paper Algebraic Interpretation of a Theorem of Clements and Lindstr\"om
%\jour Journal of Commutative Algebra ???
%\yr 2008
%\endref

%\ref\key DuFo
%\manyby David S. Dummit and Richard M. Foote
%\book Abstract Algebra
%\bookinfo Third Edition
%\publ John Wiley and Sons, Inc.
%\publaddr Hoboken, NJ
%\yr 2004
%\endref

%\ref\key Eis
%\by David Eisenbud
%\book Commutative Algebra with a View Toward Algebraic Geometry
%\bookinfo Graduate Texts in Mathematics {\bf 150} 
%\publ Springer Science+Business Media, Inc.
%\publaddr New York, NY
%\yr 2004
%\endref

%\ref\key EGH1
%\manyby D. Eisenbud, M. Green, J. Harris
%\paper Higher Castelnuovo Theory
%\jour Ast\'erisque
%\vol 218
%\pages 187-202
%\yr 1993
%\endref

%\ref\key EGH2
%\manyby D. Eisenbud, M. Green, J. Harris
%\paper Cayley-Bacharach Theorems and Conjectures
%\jour Bulletin of the American Mathematical Society
%\vol 33(3)
%\pages 295-324
%\yr 1996
%\endref

%\ref\key EHar
%\manyby D. Eisenbud, J. Harris
%\book 3264 \& All That: Intersection Theory in Algebraic Geometry
%\bookinfo
%\publ preprint
%\publaddr
%\yr 2012
%\endref

%\ref\key Fult
%\by William Fulton
%\book Intersection Theory
%\bookinfo 2nd Edition
%\publ Springer-Verlaag
%\publaddr New York, NY
%\yr 1998
%\endref

%\ref\key Hart
%\by Robin Hartshorne
%\book Algebraic Geometry
%\bookinfo Graduate Texts in Mathematics {\bf 52} 
%\publ Springer Science+Business Media, LLC
%\publaddr New York, NY
%\yr 2006
%\endref

%\ref\key HD
%\by Mel Hochster
%\book $D$-modules and Lyubeznik's Finiteness Theorems for Local Cohomology
%\bookinfo Notes, see \url{http://www.math.lsa.umich.edu/~hochster/615W11/dmod.pdf}
%\publ Unpublished
%\yr 2011
%\endref

%\ref\key Hoch
%\by Mel Hochster
%\book Math 614 Lecture Notes, Fall, 2010 
%\bookinfo Course Notes, see \url{http://www.math.lsa.umich.edu/~hochster/614F10/614.html}
%\publ Unpublished
%\yr 2010
%\endref

%\ref\key HW11
%\by Mel Hochster
%\book Local Cohomology
%\bookinfo Course Notes, see \url{http://www.math.lsa.umich.edu/~hochster/615W11/loc.pdf}
%\publ Unpublished
%\yr 2011
%\endref

%\ref\key Iyen
%\manyby Srikanth B. Iyengar, Graham J. Leuschke, Anton Leykin, Claudia Miller, Ezra Miller, Anurag K. Singh, Uli Walther 
%\book Twenty-Four Hours of Local Cohomology
%\bookinfo Graduate Studies in Mathamatics, Volume 87
%\publ American Mathematical Society 
%\publaddr Providence, RI
%\yr 2007
%\endref

%\ref\key Rich
%\by B.P. Richert
%\paper A Study of the Lex Plus Powers Conjecture
%\jour Journal of Pure and Applied Algebra
%\vol 186(2)
%\pages 169-183
%\yr 2004
%\endref

%\ref\key Sm
%\by Karen Smith
%\book Math 631: Intro to Algebraic Geometry; Fall 2008 Problem Sets
%\bookinfo see \url{http://www.math.lsa.umich.edu/~kesmith/2008-631hmwk.html}
%\publ Unpublished
%\yr 2008
%\endref

%\ref\key Stan
%\by R.P. Stanley
%\paper Hilbert Functions of Graded Algebras
%\jour Advances in Mathematics
%\vol 28
%\pages 57-83
%\yr 1978
%\endref

%\ref\key Weib
%\by Charles A. Weibel
%\book An introduction to homological algebra
%\bookinfo Cambridge studies in advanced mathematics, 38
%\publ Cambridge University Press
%\publaddr Cambridge, United Kingdom
%\yr 1994
%\endref

%%% online template %%%
%\Refs : : : \endRefs list of references
%\refstyle#1 specify style A, B, or C
%A = initials, B = name, C = number
%\ref : : : \endref individual reference
%\no or \key number or key for reference
%\widestnumber\no#1 or \widestnumber\key#1
%\by author
%\bysame same as previous author
%\paper name of paper
%\vol volume
%\yr year of publication
%\jour journal
%\page or \pages page(s)
%\toappear to appear
%\inbook article in a book
%\moreref additional reference information
%\paperinfo extra information after paper title
%\procinfo information about proceedings
%\issue issue number
%\lang language
%\transl information about translated version
%\book book
%\ed or \eds editor(s)
%\publ publisher
%\publaddr publisher address
%\bookinfo extra information after book title
%\finalinfo extra information for end
%\miscnote same as \finalinfo, in parens.

%\endRefs

\bye
