\documentclass[11pt,letterpaper]{article}
\usepackage{fullpage}
\usepackage{multicol}
\usepackage{amsmath}
\usepackage{amsfonts}
\usepackage{amssymb}
%\usepackage{pstricks, pst-node, pst-plot}

\newcommand{\ds}{\displaystyle}
\newcommand{\bv}{\mathbf}
\newcommand{\lv}{\langle}
\newcommand{\rv}{\rangle}

\begin{document}
\flushleft
\begin{multicols}{2}


\begin{large}\textbf{Math 2603 Quiz 2: Chapter 1 \\
Fri 12 September 2014}\end{large}

\textbf{Name:  }\underline{\hspace{35ex}}

\vspace{.5in}

\end{multicols}

\pagestyle{empty}


\flushleft

You have 20 minutes to complete this quiz.  No calculators allowed.  Eyes on your own paper and good luck!

\begin{enumerate}
\item  \textbf{Definitions/Concepts.} Write down the definition of
\begin{enumerate} 
\item proposition 
\vspace{2pc}
\item contrapositive 
\vspace{4pc}
\end{enumerate}

\item \textbf{Questions/Problems.}

Let $K(x,y)$ be the propositional function ``$x$ knows $y$".  The domain of discourse is the Cartesian product of the set of students taking discrete math with itself (i.e., both $x$ and $y$ take on values in the set of students taking discrete math).  Represent the assertion ``someone does not know anyone" symbolically.

\vspace{5pc}
\item \textbf{Computations/Algebra.}
\begin{enumerate}
\item Write the truth table of the proposition $\neg(P\vee Q)\wedge\neg\left((\neg P\vee R)\wedge Q\right)$.
\vspace{18pc}
\item $|\left\{\emptyset\right\}|=$
\vfill
\end{enumerate}

\end{enumerate}

\end{document}


