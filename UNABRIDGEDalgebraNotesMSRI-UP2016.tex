\documentclass[t,10pt,aspectratio=2516,envcountsect]{beamer}
\usetheme{Singapore}%Warsaw}
\usecolortheme{beaver,lily}
\setbeamertemplate{section in toc}[sections numbered]
\setbeamertemplate{subsection in toc}{
	\vspace{0.2pc}
	\hspace{15pt}
	\inserttocsubsection
	
	} % don't get rid of the above line of space!
\setbeamerfont{section in toc}{size=\fontsize{8}{9}\selectfont}
\setbeamerfont{subsection in toc}{size=\fontsize{6}{7}\selectfont}
\setbeamercolor{subsection in toc}{fg=red!50!black}
\setbeamercolor{titlelike}{fg=red!50!black}

%\beamertemplatenavigationsymbolsempty 

% % % % % % % % % % 
\usepackage{ifthen}
\usepackage{subfiles}
\usepackage{multicol}
\usepackage{hyperref}
%\usepackage{enumitem}
\usepackage{amsmath,amssymb}
\usepackage{amsthm}
\usepackage{mathtools}
\usepackage[mathscr]{euscript}
\usepackage[misc,geometry]{ifsym}
\usepackage{wasysym}
\usepackage{fancyvrb}
\usepackage{cancel}
\usepackage{xcolor}
\usepackage{graphicx}
\usepackage{tkz-graph} %tkz-graph loads tikz
\usepackage{tkz-berge}
\usepackage{chngcntr}
\usetikzlibrary{matrix,arrows,shapes,cd,calc}
\usepackage{comment}

% % % % % % % % % %
\AtBeginDocument{%
 \abovedisplayskip=2pt
 \abovedisplayshortskip=2pt
 \belowdisplayskip=3pt
 \belowdisplayshortskip=2pt
}
\AtBeginSection{
	\begin{frame}[t]{\insertsection}
	\begin{multicols}{2}
	\ifthenelse{\thesection < 6}{
		\tableofcontents[sections={1-5},currentsection]
		}{
		\tableofcontents[sections={6-10},currentsection]
		}
	\end{multicols}
	\end{frame}
}	

\numberwithin{subsection}{section}
\renewcommand{\thefootnote}{\tiny $\dagger$}
\setbeamertemplate{caption}[numbered]
\graphicspath{{algebraIndividualLectures/}}
	
% % % % % % % % % % 
\setbeamertemplate{theorems}[numbered]
\numberwithin{theorem}{subsection}
\numberwithin{equation}{section}
\numberwithin{figure}{section}

\theoremstyle{plain}
\newtheorem{thm}{Theorem}
\newtheorem{prop}{Proposition}
\newtheorem{dfn}{Definition}
\newtheorem{cor}{Corollary}

\theoremstyle{definition}
\newtheorem{ex}{Example}
\newtheorem{exe}{\color{orange!70!black}\bf Exercise}
\newtheorem*{UNABexe}{\color{orange!40!black}\bf Exercise$^{\text{\textbf *}}$}
\newtheorem*{que}{Question}

\newcommand{\lilRefP}[1]{{\usebeamercolor[fg]{block title}(\ref{#1})}}
\newcommand{\lilRefPd}[1]{{\usebeamercolor[fg]{block title}\ref{#1}.}}

% % % % % % % % % % 
\DeclareMathOperator{\coker}{coker}
\DeclareMathOperator{\GL}{GL}
\DeclareMathOperator{\image}{image}
\DeclareMathOperator{\ord}{ord}
\DeclareMathOperator{\Mat}{Mat}
\DeclareMathOperator{\SL}{SL}

\newcommand{\1}{^{-1}}
\newcommand{\C}{\mathbb C}
\renewcommand{\mod}[1]{\;(\text{mod } #1)}
\newcommand{\N}{\mathbb N}
\newcommand{\norml}{\mathrel{\triangleleft}} % strict normal subgroup
%\newcommand{\normleq}{\mathrel{\trianglelefteq}} % normal subgroup
\newcommand{\Q}{\mathbb Q}
\newcommand{\R}{\mathbb R}
\newcommand{\surj}{\twoheadrightarrow}
\newcommand{\T}{^{\text{T}}}
\newcommand{\vect}[1]{\mathbf #1}
\newcommand{\Z}{\mathbb Z}

\newcommand{\smallGap}{\vspace{3pt}}
\newcommand{\pf}{\textbf{Proof: }}
\newcommand{\bigProof}{
	\begin{block}{Proof.}
	\end{block}
	\vspace{-1pc}
	}
\newcommand{\vocab}[1]{\alert{\bf #1}}
\newcommand{\die}[1]{\raisebox{-0.5ex}{\text{\Cube{#1}}}}

% The following macros are used for writing exercises to solutions:
\newcommand{\answerKey}{
	\subsection[\subsecname]{Solutions to exercises}\footnotesize
	}
\newcommand{\exeSol}[2][]{
	\begin{block}
	{Exercise \ref{#2} #1}
	\end{block}
	\vspace{-0.75pc}
	\textbf{Solution:}
	}

% % % % % % % % % % % % % % % % % % % %
\title[Group Theory]{
	Lectures on Group Theory 
	}
\subtitle{}
\author[Ashley K. Wheeler]{
	{\large Ashley K. Wheeler} \\
	\vspace{0.5pc}
	{\small
	\url{comp.uark.edu/~ashleykw} \\
	Department of Mathematical Sciences \\
	University of Arkansas}
	}
\institute[UArk]{}
\date[2016 MSRI-UP]{\small
	2016 Mathematical Sciences Research Institute \\
	Undergraduate Program \\
	\smallGap
	Berkeley, CA \\ 
	11 June -- 24 July,  2016 \\
	
	}
\logo{\tiny{\it last updated:} \today \hspace{10pt}}

% % % % % % % % % % % % % % % % % % % %
\begin{document}
% % %
{
\setbeamertemplate{headline}{\pgfuseshading{beamer@headfade}}

\begin{frame}
\titlepage
\end{frame}

% % %
\begin{frame}{Overview}
\begin{multicols}{2}
%Preface
%\smallGap
\tableofcontents[sections={1-5}]
\end{multicols}
\end{frame}

% % %
\begin{frame}{Overview}
\begin{multicols}{2}
\tableofcontents[sections={6-10}]
%Bibliography
\end{multicols}
\end{frame}
}

% % % % % % % % % %
%\section*{Preface}
% % %
%\begin{frame}{\secname}{}\footnotesize
%These slides are an unabridged version of the Group Theory Lectures given at the 2016 MSRI-UP.  Included are some finer details and invokations of advanced topics beyond the scope of the material needed for the research topic.
%{\bf INTRODUCTION, THESE ARE A COMPANION TO LUIS' NOTES...
%%CONVENTION ABOUT EXERCISES, EXAMPLES, VOCABULARY WORDS...
%%EARLIER LECTURES FOCUS ON RIGOROUS TECHNIQUES IN PROOFS...}
%\end{frame}

% % % % % % % % % %
\subfile{algebraIndividualLectures/UNAB2p1basicDefinitions}
\subfile{algebraIndividualLectures/UNAB2p2basicProperties}
\subfile{algebraIndividualLectures/UNAB2p3subgroups}
\subfile{algebraIndividualLectures/UNAB2p4ordersOfGroupsAndElements}
\subfile{algebraIndividualLectures/UNAB2p5cyclicGroups}
\subfile{algebraIndividualLectures/UNAB2p6groupHomomorphisms}
\subfile{algebraIndividualLectures/UNAB2p7quotientGroups}
\subfile{algebraIndividualLectures/UNAB2p8firstIsomorphismTheorem}
\subfile{algebraIndividualLectures/UNAB2p9diagonalizationOfIntegerMatrices}
\subfile{algebraIndividualLectures/UNAB2p10FTOFinitelyGeneratedAbelianGroups}

\begin{comment}
% % % % % % % % % %
\section*{Bibliography}
% % %

\begin{frame}[allowframebreaks]{References}
\def\newblock{}
\bibliographystyle{plain}
\bibliography{mybib}
\end{frame}

\begin{frame}{\secname}{}
{\bf PUT ONE HERE}

@book {atiyah+macdonald,
    AUTHOR = {Atiyah, M. F. and Macdonald, I. G.},
     TITLE = {Introduction to commutative algebra},
 PUBLISHER = {Addison-Wesley Publishing Co., Reading, Mass.-London-Don
              Mills, Ont.},
      YEAR = {1969},
     PAGES = {ix+128},
   MRCLASS = {13.00},
  MRNUMBER = {0242802 (39 \#4129)},
MRREVIEWER = {J. A. Johnson},
}

@book {dummit+foote,
    AUTHOR = {Dummit, David S. and Foote, Richard M.},
     TITLE = {Abstract algebra},
   EDITION = {Third},
 PUBLISHER = {John Wiley \& Sons, Inc., Hoboken, NJ},
      YEAR = {2004},
     PAGES = {xii+932},
      ISBN = {0-471-43334-9},
   MRCLASS = {00-01 (16-01 20-01)},
  MRNUMBER = {2286236 (2007h:00003)},
}

@Book{Eisenbud94,
  author =	"David Eisenbud",
  title =	"Commutative algebra with a view toward algebraic
		 geometry",
  booktitle =	"Commutative algebra with a view toward algebraic
		 geometry",
  pages =	"xvi,785",
  year = 	"1994",
  volume =	"150",
  series =	"Graduate Texts in Mathematics",
  publisher =	"Springer-Verlag",
  address =	"New York-Berlin-Heidelberg-London-Paris-Tokyo-Hong
		 Kong-Barcelona-Budapest",
  cdate =	"1970-01-01",
  mdate =	"2005-08-18"
}

@book {MR1712872,
    AUTHOR = {Mac Lane, Saunders},
     TITLE = {Categories for the working mathematician},
    SERIES = {Graduate Texts in Mathematics},
    VOLUME = {5},
   EDITION = {Second},
 PUBLISHER = {Springer-Verlag, New York},
      YEAR = {1998},
     PAGES = {xii+314},
      ISBN = {0-387-98403-8},
   MRCLASS = {18-02},
  MRNUMBER = {1712872},
}

@book {miller+sturmfels,
    AUTHOR = {Miller, Ezra and Sturmfels, Bernd},
     TITLE = {Combinatorial commutative algebra},
    SERIES = {Graduate Texts in Mathematics},
    VOLUME = {227},
 PUBLISHER = {Springer-Verlag},
   ADDRESS = {New York},
      YEAR = {2005},
     PAGES = {xiv+417},
      ISBN = {0-387-22356-8},
   MRCLASS = {13-01 (05-01 05E99 13D02 14M15 14M25)},
  MRNUMBER = {2110098 (2006d:13001)},
MRREVIEWER = {Joseph Gubeladze}
}

\end{frame}
\end{comment}

% % % % % % % % % % % % % % % % % % % %
\end{document}