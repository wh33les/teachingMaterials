\documentclass[12pt]{article}
\usepackage{fullpage}
\usepackage{lastpage}
\usepackage{fancyhdr}
\pagestyle{fancy}

\addtolength{\topmargin}{-0.25in}
\usepackage{graphicx}	
\usepackage{array, multicol}
\usepackage{amsmath}

\everymath{\displaystyle}

\fancypagestyle{plain}{
	\fancyhf{}
	\addtolength{\headheight}{2.92\baselineskip}
	\lhead{\bf Quiz 1: The Idea of Limits \\
		\S 2.1-2.2 
		}
	\rhead{\bf MATH 2554 (Calculus I) \\
		%\vspace{0.5pc}
		due Tues 26 Jan 2016}
	\rfoot{Quiz 1 p.\thepage\ (of \pageref{LastPage})}
	}
\fancyhf{}
\renewcommand{\headrulewidth}{0pt}

\title{%\flushleft\vspace{-1.5pc}\Large
	%\bf Quiz 1: The Idea of Limits ($\textstyle\oint$2.1-2.2)
	}
\author{}
\date{}

\rfoot{Quiz 1 p.\thepage\ (of \pageref{LastPage})}

% % % % %
\begin{document}
\maketitle

\vspace{-7pc}
\noindent{\bf Directions:} This quiz is due on Tuesday, 26 January, 2016 at the beginning of your drill.  You may use your brain, notes, book, or other humans to complete your work.  \textbf{Your solutions must be on a separate sheet of paper, in order, stapled, de-fringed, and legible with your name on the top right corner of the first page.}  If you fail to meet any of these requirements, you will receive a zero.  Each question is worth one point, and will be graded as correct or not correct (all or nothing).

\noindent\hrulefill

\begin{enumerate}

% % %
\item Given the function $f(x)=x-x^3$, 
\begin{enumerate}
	\item determine the slope of the secant line between the following $x$-coordinates:
	\begin{multicols}{3}
	\begin{enumerate}
		\item $\left[1,1.5\right]$
		%\vspace{8pc}
		\item $\left[1,1.05\right]$
		%\vspace{8pc}
		\item $\left[1,1.005\right]$
		%\vspace{8pc}
		\item $\left[1,h\right]$, assuming $h>1$
		%\vspace{8pc}
	\end{enumerate}
	\end{multicols}
	%\vspace{8pc}
	
	\item then use your answers from (a) to estimate the slope of the tangent line to $f(x)$ at $x=1$;
	%\vspace{5pc}

	\item using the limit symbol, how would you express your conclusion in part (b)?
	%\vspace{4pc}
\end{enumerate}

% % % 
\item Given the function $w(z)=z^3-z^2$,
\begin{enumerate}
	\item make a table of values of the function given the inputs
	\[
	z=0.9,0.99,0.999,1.1,1.01,1.001;
	\]
	\item then use your answers from (a) to ``estimate" the value for $w(1)$ \textit{(note, for this particular function you already know the value for $w(1)$ -- the point of this exercise is to make sure it is consistent with your answers in part (a))};
	\item using the limit symbol, how would you express your conclusion in part (b)?
\end{enumerate}	
% % %
\item Sketch the graph of a function satisfying all of the following:
\begin{multicols}{2}
\begin{itemize}
\item $\lim_{x\to -1}h(x)=3$
\item $h(-1)$ is undefined
\item $h(5)=2$
%\item $\lim_{x\to 5}h(x)$ does not exist
\item $\lim_{x\to 5^+}h(x)=0$
\end{itemize}
\end{multicols}
%\vspace{1pc}

% % %
\item Given the function 
\[
g(x)=\begin{cases}
	1+\sin{x} & \text{if }x<0 \\
	\cos{x} & \text{if }0\leq x\leq\pi \\
	\sin{x} & \text{if }x>\pi
\end{cases}
\]
\begin{enumerate}
	\item sketch the graph of $g(x)$, then
	\item use the graph to determine the value(s) of $a$ for which $\lim_{x\to a}g(x)$ does not exist;
	\item what are $\lim_{x\to a^+}g(x)$ and $\lim_{x\to a^-}g(x)$, where $a$ is your answer(s) from part (b)?
\end{enumerate}


% % % % %
\end{enumerate}
\end{document}