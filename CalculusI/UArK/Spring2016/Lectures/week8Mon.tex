\documentclass[cal1spr16Lectures.tex]{subfiles}

\begin{document}

%\section[Week 8]{Week 8: 7-11 Mar}

% % % 
\subsubsection{\bf Monday 7 March}
% % %

\begin{frame}[allowframebreaks]{Mon 7 Mar}
\begin{itemize}\footnotesize
\item Exam 2: Solutions posted as soon as make-ups are in.  But we will go through them today.  You'll get your test back in drill tomorrow and the curve will be posted.
\item Midterm:
	\begin{itemize}\footnotesize
	\item Covers \alert{everything} up to \alert{\S 3.9}.  All the slides are up.  We will work fast through them today, but solutions to the exercises in the slides will be posted.
	\item Morning Section: Walker room 124
	
	Afternoon Section: Walker room 218
	
	You \alert{must} take the test with your officially scheduled section.
%
\framebreak
	\item If you have questions about your exam conflicts, contact me NOW.  
	\item Study guide is in MLP.
	\item Basic scientific calculator is allowed...? Yes.
	\item Sit in every other seat.
	\item 15 Questions, 10 points each.
	\item Don't expect a curve. :(
	\end{itemize}
\end{itemize}
\end{frame}

% % %
\subsection[3.9 Derivatives of Logarithmic and Exponential Functions]{\S 3.9 Derivatives of Logarithmic and Exponential Functions}
% % %

% % %
\begin{frame}{\S 3.9 Derivatives of Logarithmic and Exponential Functions}\footnotesize
The natural exponential function $f(x)=e^x$ has an inverse function, namely $f^{-1}(x)=\ln x$.  This relationship has the following properties:
\begin{itemize}
\item[1.] $e^{\ln x}=x $ for $x>0$ and $\ln(e^x)=x$ for all $x$.
\item[2.] $y=\ln x \quad\Longleftrightarrow\quad x=e^y$
\item[3.] For real numbers $x$ and $b>0$, 
\[b^x=e^{\ln (b^x)}=e^{x \ln b}.\]
\end{itemize}
\end{frame}

% % %
\subsubsection{Derivative of $y=\ln x$}
% % % 

% % %
\begin{frame}{\small Derivative of $y=\ln x$}\footnotesize
Using 2. from the last slide, plus implicit differentiation, we can find $\textstyle\frac{d}{dx}\left(\ln x\right)$.  Write $y=\ln x$.  We wish to find $\textstyle\frac{dy}{dx}$.  From 2.,
\begin{align*}
\frac{d}{dx}\big(x=e^y\big) \Rightarrow \frac{d}{dx}x &= \frac{d}{dx}(e^y) \\[0.5pc]
1 &= e^y\left(\frac{dy}{dx}\right) \\[0.5pc]
\frac{dy}{dx} &= \frac{1}{e^y}=\frac{1}{x} 
\end{align*}
So \alert{$\frac{d}{dx}(\ln x)=\frac{1}{x}$}.
\end{frame}

% % %
\subsubsection{Derivative of $y=\ln{|x|}$}
% % %

% % %
\begin{frame}{\small Derivative of $y=\ln |x|$}\footnotesize
Recall, we can only take ``$\ln{}$" of a positive number.  However:
\begin{itemize}
\item For $x>0$, $\ln |x| = \ln x$, so 
\[\frac{d}{dx} (\ln |x|)=\frac{1}{x}.\]
\item For $x<0$, $\ln |x| = \ln(-x)$, so 
\[\frac{d}{dx} (\ln |x|)= \frac{d}{dx} (\ln(-x)) = \frac{1}{-x} \cdot (-1) = \frac{1}{x}.\]
\end{itemize}
In other words, the absolute values do not change the derivative of natural log.
%\[\frac{d}{dx}(\ln x)=\frac{d}{dx}(\ln |x|)=\frac{1}{x}.\]
\end{frame}

% % %
\begin{frame}
\begin{exe} Find the derivative of each of the following functions:
\begin{itemize}
\item $f(x)=\ln(15x)$
\item $g(x)=x \ln x$
\item $h(x)=\ln(\sin x)$
\end{itemize}
\end{exe}
\end{frame}

% % %
\subsubsection{Derivative of $y=b^x$}
% % %

% % %
\begin{frame}{\small Derivative of $y=b^x$}
What about other logs?  Say $b>0$.  Since $b^x=e^{\ln b^x}=e^{x \ln b}$ (by 3. on the earlier slide), 
\begin{align*}
\alert{\frac{d}{dx}(b^x)} &= \frac{d}{dx}(e^{x \ln b}) \\[0.75pc]
 &=e^{x \ln b} \cdot \ln b \\[0.75pc]
 &= \alert{b^x \ln b}.
\end{align*}
\end{frame}

% % %
\begin{frame}
\begin{exe} Find the derivative of each of the following functions:
\begin{itemize}
\item $f(x)=14^x$
\item $g(x)=45(3^{2x})$
\end{itemize}
\end{exe}
\begin{exe} Determine the slope of the tangent line to the graph $f(x)=4^x$ at $x=0$. \end{exe}
\end{frame}

% % %
\subsubsection{Story Problem Example}
% % %

% % %
\begin{frame}{\small Story Problem Example}\footnotesize
\begin{ex} The energy (in Joules) released by an earthquake of magnitude $M$ is given by the equation
\[E=25000 \cdot 10^{1.5 M}.\]
%\vspace{-1pc}
\begin{itemize}\footnotesize
\item[(a)] How much energy is released in a magnitude 3.0 earthquake?
\item[(b)] What size earthquake releases 8 million Joules of energy?
\item[(c)] What is $\textstyle\frac{dE}{dM}$ and what does it tell you?
\end{itemize}
\end{ex}
\end{frame}

% % %
\subsubsection{Derivatives of General Logarithmic Functions}
% % %

% % %
\begin{frame}{\small Derivatives of General Logarithmic Functions}\footnotesize
The relationship $y=\ln x \Longleftrightarrow x=e^y$ applies to logarithms of other bases:
\[y=\log_b x \quad\Longleftrightarrow\quad x=b^y.\]
Now taking $\textstyle\frac{d}{dx}\left(x=b^y\right)$ we obtain
\vspace{-0.75pc}
\begin{align*}
1 &= b^y\ln b\left(\frac{dy}{dx}\right) \\
\frac{dy}{dx} &= \frac{1}{b^y \ln b} \\[0.5pc]
\alert{\frac{d}{dx}(\log_b x)} &=\alert{\frac{1}{x \ln b}} 
\end{align*}
\end{frame}

% % %
\begin{frame}
\begin{ex}
The derivative of $f(x)=\log_2{(10x)}$ is
\begin{itemize}
\item[A. ] $\textstyle\frac{1}{10x}$
\item[B. ] $\textstyle\frac{1}{x\ln 2}$
\item[C. ] $\textstyle\frac{1}{x}$
\item[D. ] $\textstyle\frac{10}{x\ln 2}$
\end{itemize}
\end{ex}
\end{frame}

% % %
\subsubsection{Neat Trick: Logarithmic Differentiation}
% % %

% % %
\begin{frame}{\small Neat Trick: Logarithmic Differentiation}
\begin{ex}  Compute the derivative of $f(x)=\frac{x^2(x-1)^3}{(3+5x)^4}$. \end{ex}
{\bf Solution:} We can use logarithmic differentiation -- first take the natural log of both sides and then use properties of logarithms.
\end{frame}

% % %
\begin{frame}\footnotesize
\begin{alignat*}{2}
\alert{\ln(}f(x)\alert{)} &= \alert{\ln \Big(} \frac{x^2(x-1)^3}{(3+5x)^4} \alert{\Big)} \\%[0.5pc]
&= \ln{x^2} + \ln{(x-1)^3}-\ln{(3+5x)^4} \\%[0.5pc]
&= 2\ln x + 3\ln(x-1)-4\ln(3+5x)
\end{alignat*}
\alert{Now} we take $\textstyle\frac{d}{dx}$ on both sides:
\begin{alignat*}{2}
\frac{1}{f(x)}\left(\frac{df}{dx}\right) &= 2\left(\frac{1}{x}\right) + 3\left(\frac{1}{x-1}\right) - 4\left(\frac{1}{3+5x}\right)(5) \\[1pc]
\frac{f^{\prime}(x)}{f(x)} &= \frac{2}{x} + \frac{3}{x-1} - \frac{20}{3+5x}
\end{alignat*}
\end{frame}

% % %
\begin{frame}{}
Finally, solve for $f^{\prime}(x)$:
\begin{alignat*}{2}
f^{\prime}(x) &= \alert{f(x)} \left[ \frac{2}{x} + \frac{3}{x-1} - \frac{20}{3+5x} \right] \\[1pc]
&= \alert{\frac{x^2 (x-1)^3}{(3+5x)^4}} \left[ \frac{2}{x} + \frac{3}{x-1} - \frac{20}{3+5x} \right]
\end{alignat*}
\end{frame}

% % %
\begin{frame}
\begin{exe}
Use logarithmic differentiation to calculate the derivative of 
\[
f(x)=\frac{(x+1)^{\frac{3}{2}}(x-4)^{\frac{5}{2}}}{(5x+3)^{\frac{2}{3}}}.
\]
\end{exe}
\end{frame}

% % %
\subsubsection{Book Problems}
% % %

% % %
\begin{frame}
\begin{block}{3.9 Book Problems} 9-29 (odds), 55-67 (odds) \end{block}
\end{frame}


% % %
\subsection{Midterm Review}
% % %

% % %
\begin{frame}[allowframebreaks]{Midterm Review}\footnotesize
\begin{itemize}
\item \S 2.1-2.2 
	\begin{itemize}\footnotesize
	\item {\bf Material may not be explicitly tested, but the topics here are foundational to later sections.}
	\end{itemize}
%\framebreak	
\item \S 2.3 Techniques for Computing Limits
	\begin{itemize}\footnotesize
	\item {\bf Be able to do questions similar to 1-48.}
	\item Know and be able to compute limits using analytical methods (e.g., limit laws, additional techniques).
	\item Be able to evaluate one-sided and two-sided limits of functions.
	\item Know the Squeeze Theorem and be able to use this theorem to determine limits.
	\end{itemize}
\framebreak
\begin{exe}[problems from past midterm] Evaluate the following limits:
\begin{itemize}
\item $\displaystyle\lim_{x\to 3}\frac{x^2-x-6}{x^2-9}$
\item $\displaystyle\lim_{\theta\to 0}\frac{\sec\theta\tan\theta}{\theta}$
\end{itemize}
\end{exe}
%
\framebreak
\item \S 2.4 Infinite Limits
	\begin{itemize}\footnotesize
	\item {\bf Be able to do questions similar to 17-30.}
	\item Be able to use a graph, a table, or analytical methods to determine infinite limits.
	\item Be able to use analytical methods to evaluate one-sided limits.
	\item Know the \alert{definition} of a vertical asymptote and be able to determine whether a function has vertical asymptotes.
	\end{itemize}
%
\framebreak
\item \S 2.5 Limits at Infinity
	\begin{itemize}\footnotesize
	\item {\bf Be able to do questions similar to 9-30 and 38-46.}
	\item Be able to find limits at infinity and horizontal asymptotes. 
	\item Know how to compute the limits at infinity of rational functions and algebraic functions.
	\item Be able to list horizontal and/or vertical asymptotes of a function.
	\end{itemize}
\begin{exe}
Determine the horizontal asymptote(s) for the function
\[
f(x)=\frac{10x^3-3x^2+8}{\sqrt{25x^6+x^4+2}}
\]
\begin{itemize}
\item[A. ] $y=2$
\item[B. ] $y=0$
\item[C. ] $y=-2$
\item[D. ] $y=\pm 2$
\end{itemize}
\end{exe}	

\framebreak	
\item \S 2.6 Continuity
	\begin{itemize}\footnotesize
	\item {\bf Be able to do questions similar to 9-44.}
	\item Know the definition of continuity and be able to apply the continuity checklist.
	\item Be able to determine the continuity of a function (including those with roots) on an interval.
	\item Be able to apply the Intermediate Value Theorem to a function.
	\end{itemize}
\framebreak
\begin{exe}[problem from past midterm]
Determine the value of $k$ so the function is continuous on $0\leq x\leq 2$.
\[f(x)=\begin{cases}x^2+k & 0\leq x\leq 1 \\
	-2kx+4 & 1<x\leq 2\end{cases}
	\]
\end{exe}	

\framebreak
\item \S 3.1 Introducing the Derivative
	\begin{itemize}\footnotesize
	\item {\bf Be able to do questions similar to 11-32.}
	\item Know the definition of a derivative and be able to use this definition to calculate the derivative of a given function.
	\item Be able to determine the equation of a line tangent to the graph of a function at a given point.
	\item Know the 3 conditions for when a function is not differentiable at a point, and why these three conditions make a function not differentiable at the given point.
	\end{itemize}

\framebreak
\item \S 3.2 Working with Derivatives
	\begin{itemize}\footnotesize
	\item Be able to use the graph of a function to sketch the graph of its derivative, without computing derivatives
	\item Know the 3 conditions for when a function is not differentiable at a point, and why these three conditions make a function not differentiable at the given point
	\item Be able to determine where a function is not differentiable
	\end{itemize}
%
\framebreak	
\item \S 3.3 Rules for Differentiation
	\begin{itemize}\footnotesize
	\item {\bf Be able to do questions similar to 7-41.}
	\item Be able to use the various rules for differentiation (e.g., constant rule, power rule, constant multiple rule, sum and difference rule) to calculate the derivative of a function.
	\item Know the derivative of $e^x$.
	\item Be able to find slopes and/or equations of tangent lines.
	\item Be able to calculate higher-order derivatives of functions.
	\end{itemize}
%
\framebreak
\begin{exe}
Given that $y=3x+2$ is tangent to $f(x)$ at $x=1$ and that $y=-5x+6$ is tangent to $g(x)$ at $x=1$, write the equation of the tangent line to $h(x)=f(x)g(x)$ at $x=1$.
\end{exe}
%
\framebreak
\item \S 3.4 The Product and Quotient Rules
	\begin{itemize}\footnotesize
	\item {\bf Be able to do questions similar to 7-42 and 47-52.}
	\item Be able to use the product and/or quotient rules to calculate the derivative of a given function.
	\item Be able to use the product and/or quotient rules to find tangent lines and/or slopes at a given point.
	\item Know the derivative of $e^{kx}$.
	\item Be able to combine derivative rules to calculate the derivative of a function.
	\end{itemize}
	
{\bf Note:} Functions are not always given by a formula.  When faced with a problem where you don't know where to start, go through the rules first.
%
\framebreak
\begin{exe} Suppose you have the following information about the functions $f$ and $g$:
\[f(1)=6\quad f'(1)=2\quad g(1)=2\quad g'(1)=3\]
\vspace{-1pc}	
	\begin{itemize}\footnotesize
	\item Let $F=2f+3g$.  What is $F(1)$?  What is $F'(1)$?
	\item Let $G=fg$.  What is $G(1)$?  What is $G'(1)$?
	\end{itemize}
\end{exe}
%
\framebreak
\item \S 3.5 Derivatives of Trigonometric Functions
	%\vspace{-0.75pc}
	\begin{itemize}\footnotesize
	\item {\bf Be able to do questions similar to 1-55.}
	\item Know the two special trigonometric limits
	\vspace{-0.5pc}
	\[\lim_{x\to 0}\frac{\sin x}{x}=1\qquad\text{and}\qquad\lim_{x\to 0}\frac{\cos x-1}{x}=0\]
	and be able to use them to solve other similar limits.
	\item Know the derivatives of $\sin x$, $\cos x$, $\tan x$, $\cot x$, $\sec x$, $\csc x$, and be able to use the quotient rule to derive the derivatives of $\tan x$, $\cot x$, $\sec x$, and $\csc x$.
	\item Be able to calculate derivatives (including higher order) involving trig functions using the rules for differentiation.
	\end{itemize}
%
\framebreak
\begin{exe}
Calculate the derivative of the following functions:
\begin{itemize}
\item $f(x)=(1+\sec x)\sin^3x$
%
%\vspace{1pc}
\item $g(x)=\displaystyle\frac{\sin x+\cot x}{\cos x}$
\end{itemize}
\end{exe}
%
%
\begin{exe} Evaluate $\displaystyle\lim_{x\to -3}\frac{\sin{(x+3)}}{x^2+8x+15}$. \end{exe}
%
%
\framebreak
\item \S 3.6 Derivatives as Rates of Change
	\begin{itemize}\footnotesize
	\item {\bf Be able to do questions similar to 11-18.}
	\item Be able to use the derivative to answer questions about rates of change involving:
		\begin{itemize}%\tiny
		\item Position and velocity
		\item Speed and acceleration
		\item Growth rates
		\item Business applications
		\end{itemize}
\framebreak		
	\item Be able to use a position function to answer questions involving velocity, speed, acceleration, height/distance at a particular time $t$, maximum height, and time at which a given height/distance is achieved.
	\item Be able to use growth models to answer questions involving growth rate and average growth rate, and cost functions to answer questions involving average and marginal costs.
	\end{itemize}
%
\framebreak
\item \S 3.7 The Chain Rule
	\begin{itemize}\footnotesize
	\item {\bf Be able to do questions similar to 7-43.}
	\item Be able to use both versions of the Chain Rule to find the derivative of a composition function.
	\item Be able to use the Chain Rule more than once in a calculation involving more than two composed functions.
	\item Know and be able to use the Chain Rule for Powers:
	\[\frac{d}{dx}\left(f(x)\right)^n=n\left(f(x)\right)^{n-1}f'(x)\]
	\end{itemize}
\framebreak
\begin{exe} Suppose $f'(9)=10$ and $g(x)=f(x^2)$.  What is $g'(3)$? \end{exe}
%
\framebreak
\item \S 3.8 Implicit Differentiation
	\begin{itemize}\footnotesize
	\item {\bf Be able to do questions similar to 5-26 and 33-46.}
	\item Be able to use implicit differentiation to calculate $\textstyle\frac{dy}{dx}.$
	\item Be able to use the derivative found from implicit differentiation to find the slope at a given point and/or a line tangent to the curve at the given point.
	\item Be able to calculate higher-order derivatives of implicitly defined functions.
	\item Be able to calculate $\textstyle\frac{dy}{dx}$ when working with functions containing rational functions.
	\end{itemize}
%
\begin{exe} Use implicit differentiation to calculate $\textstyle\frac{dz}{dw}$ for
\[e^{2w}=\sin(wz)\]
\end{exe}
\begin{exe} If $\sin x=\sin y$, then 
\begin{itemize}\footnotesize
\item $\textstyle\frac{dy}{dx}=$ ? 
\item $\textstyle\frac{d^2y}{dx^2}=$ ? 
\end{itemize}
\end{exe}
\framebreak
\item \S 3.9 Derivatives of Logarithmic and Exponential Functions
	\vspace{-0.5pc}
	\begin{itemize}\footnotesize
	\item Be able to compute derivatives involving $\ln x$ and $\log_b x$
	\item Be able to compute derivatives of exponential functions of the form $b^x$
	\item Be able to use logarithmic differentiation to determine $f^{\prime}(x)$
	\end{itemize}

\end{itemize}
\end{frame}

% % % 
\subsubsection{Running Out of Time on the Exam Plus other Study Tips}
% % %
\begin{frame}[allowframebreaks]{\small Running Out of Time on the Exam Plus other Study Tips}\footnotesize
\begin{itemize}
\item Do practice problems completely, from beginning to end (as if it were a quiz).  You might think you understand something but when it's time to write down the details things are not so clear.  
\item Find a buddy who understands concepts a little better than you and work on problems for 2-3 hours.  Then find a buddy who is struggling and work with them 2-3 hours.  %Explaining to someone else tests how deeply you really know the material.  This strategy also helps reduce stress because it doesn't require you to devote a full day or night of studying, just 2-3 hours at a time of productive work.
\item Don't count on cookie cutter problems.  If you are doing a practice problem where you've memorized all the steps, make sure you understand why each step is needed.  The exam problems may have a small variation from homeworks and quizzes.  If you're not prepared, it'll come as a ``twist" on the exam...
%
\framebreak
\item If you encounter an unfamiliar type of problem on the exam, relax, because it's most likely not a trick!  The solutions will always rely on the information from the required reading/assignments.  Take your time and do each baby step carefully.  
\item During the exam, do the problems you are most confident with first!  %Different people will find different problems easier.
\item During the exam, budget your time.  Count the problems and divide by 50 minutes.  The easier questions will take less time so doing them first leaves extra time for the harder ones.  When studying, aim for 10 problems per hour (i.e., 6 minutes per problem).
%
\framebreak
\item Always make sure you \alert{answer the question}.  This is also a good strategy if you're not sure how to start a problem, figure out what the question wants first.
\item The exam is not a race.  If you finish early take advantage of the time to check your work.  You don't want to leave feeling smug about how quickly you finished only to find out next week you lost a letter grade's worth of points from silly mistakes.
\end{itemize}
\end{frame}

% % %
\subsubsection{Other Study Tips}
% % %

% % %
\begin{frame}{\small Other Study Tips}
\footnotesize
\begin{itemize}
\item Brush up on algebra, especially radicals, logs, common denominators, etc.  Many times knowing the right algebra will simplify the problem!
\item When in doubt, show steps.  
\item You will be punished for wrong notation.  The slides for \S 3.1 show different notations for the derivative.  Make sure whichever one you use in your work, that you are using it correctly.
\item Read the question!  
\item Do the book problems.
\item Look at the pictures in the book and the interactive applets on MLP.
\end{itemize}
\end{frame}

\end{document}