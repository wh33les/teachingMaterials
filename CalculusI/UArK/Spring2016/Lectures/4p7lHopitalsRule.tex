\documentclass[cal1spr16Lectures.tex]{subfiles}

\begin{document}

\section[]{}

% % %
\subsection[4.7 L'H\^opital's Rule]{\S 4.7 L'H\^{o}pital's Rule}
% % % 

% % %
\begin{frame}{\S 4.7 L'H\^{o}pital's Rule}\small
In Ch.\ 2, we examined limits that were computed using analytical techniques.  Some of these limits, in particular those that were indeterminate, could not be computed with simple analytical methods% (e.g., substitution)
.

\vspace{1pc}
For example, 
\vspace{-0.5pc}
\[\lim_{x \to 0} \frac{\sin x}{x}\quad\text{ and }\quad \lim_{x \to 0} \frac{1-\cos x}{x}\] 
are both limits that can't be computed by substitution, because plugging in 0 for $x$ gives $\frac{0}{0}$.
\end{frame}

% % %
\begin{frame}
\frametitle{}
\small
\begin{thm}[L'H\^{o}pital's Rule $(\frac{0}{0})$]
Suppose $f$ and $g$ are differentiable on an open interval $I$ containing $a$ with $g^{\prime}(x) \ne 0$ on $I$ when $x \ne a$.  If 
$$\lim_{x \to a} f(x)=\lim_{x \to a} g(x)=0$$
then
$$\lim_{x \to a} \frac{f(x)}{g(x)}=\lim_{x \to a} \frac{f^{\prime}(x)}{g^{\prime}(x)},$$
provided the limit on the right side exists (or is $\pm \infty$).
\end{thm}

\vspace{1pc}
(The rule also applies if $x \to a$ is replaced by $x \to \pm \infty$, $x \to a^+$ or $x \to a^-$.)
\end{frame}

% % %
\begin{frame}%[t]
\frametitle{}
\small
\begin{ex} Evaluate the following limit:  
\vspace{-0.5pc}
\[\lim_{x \to -1} \frac{x^4+x^3+2x+2}{x+1}.\]
\end{ex}

\footnotesize
{\bf Solution:}  By direct substitution, we obtain 0/0.  So we must apply l'H\^{o}pital's Rule (LR) to evaluate the limit:
\begin{alignat*}{2}
\lim_{x \to -1} \frac{x^4+x^3+2x+2}{x+1} &\overset{\text{LR}}{=} \lim_{x \to -1} \frac{ \dfrac{d}{dx} \left(x^4+x^3+2x+2 \right)}{\dfrac{d}{dx} (x+1)} \\
&= \lim_{x \to -1} \frac{4x^3+3x^2+2}{1} \\
&= -4+3+2 = 1
\end{alignat*}
\end{frame}

% % %
\begin{frame}
\frametitle{}
\small
\begin{thm}[L'H\^{o}pital's Rule $(\frac{\infty}{\infty})$]
Suppose $f$ and $g$ are differentiable on an open interval $I$ containing $a$ with $g^{\prime}(x) \ne 0$ on $I$ when $x \ne a$.  If 
$$\lim_{x \to a} f(x)=\lim_{x \to a} g(x)=\pm\infty$$
then
$$\lim_{x \to a} \frac{f(x)}{g(x)}=\lim_{x \to a} \frac{f^{\prime}(x)}{g^{\prime}(x)},$$
provided the limit on the right side exists (or is $\pm \infty$).
\end{thm}

\vspace{1pc}
(The rule also applies if $x \to a$ is replaced by $x \to \pm \infty$, $x \to a^+$ or $x \to a^-$.)
\end{frame}

% % %
\begin{frame}%[t]
\frametitle{}
\begin{exe} Evaluate the following limits using l'H\^{o}pital's Rule:

\begin{itemize}
\item $\displaystyle\lim_{x \to \infty} \frac{4x^3-2x^2+6}{\pi x^3+4}$

\vspace{1pc}
\item $\displaystyle\lim_{x \to 0} \frac{\tan 4x}{\tan 7x}$
\end{itemize}
\end{exe}
\end{frame}

% % %
\subsubsection{L'H\^opital's Rule in disguise}
% % %

% % %
\begin{frame}{\small L'H\^opital's Rule in disguise}
\small
Other indeterminate limits in the form $0 \cdot \infty$ or $\infty - \infty$ cannot be evaluated directly using l'H\^{o}pital's Rule.

\vspace{1pc}
For $0 \cdot \infty$ cases, we must rewrite the limit in the form $\frac{0}{0}$ or $\frac{\infty}{\infty}$.  A common technique is to divide by the reciprocal:
\[\lim_{x \to \infty} \alert{x^2} \sin \left( \frac{1}{5x^2} \right) = \lim_{x \to \infty} \frac{\sin \left( \dfrac{1}{5x^2} \right)}{\alert{\dfrac{1}{x^2}}}\]
\end{frame}

% % %
\begin{frame}%[t]
\frametitle{}
\small
\begin{exe} Compute $\displaystyle\lim_{x \to \infty} x \sin \left( \dfrac{1}{x} \right).$ \end{exe}
\end{frame}

% % %
\begin{frame}
\footnotesize
For $\infty - \infty$, we can divide by the reciprocal as well as use a change of variables:
\begin{ex} Find $\displaystyle\lim_{x \to \infty} x-\sqrt{x^2+2x}$. \end{ex}
{\bf Solution:}
\vspace{-1.5pc}
\begin{alignat*}{2}
\lim_{x \to \infty} x-\sqrt{x^2+2x} &= \lim_{x \to \infty} x-\sqrt{\alert{x^2}(1+\frac{2}{x})} \\%[0.5pc]
&=\lim_{x \to \infty} x-\alert{x}\sqrt{1+\frac{2}{x}} \\
&= \lim_{x \to \infty} \alert{x} \left( 1-\sqrt{1+\frac{2}{x}} \right) \\
&= \lim_{x \to \infty} \frac{ 1-\sqrt{1+\frac{2}{x}} }{\alert{\frac{1}{x}} }
\end{alignat*}
\end{frame}

% % %
\begin{frame}
\small
This is now in the form $\frac{0}{0}$, so we can apply l'H\^{o}pital's Rule and evaluate the limit.  

\vspace{1pc} 
In this case, it may even help to change variables.  Let $t=\frac{1}{x}$:
\[\lim_{x \to \infty} \frac{ 1-\sqrt{1+\frac{2}{x}} }{\dfrac{1}{x} } = \lim_{t \to 0^+} \frac{1-\sqrt{1+2t}}{t}.\]
\end{frame}

% % %
\subsubsection{Other Indeterminate Forms}
% % %

% % %
\begin{frame}{\small Other Indeterminate Forms}
\footnotesize
Limits in the form $1^{\infty}$, $0^0$, and $\infty^0$ are also considered indeterminate forms, and to use l'H\^{o}pital's Rule, we must rewrite them in the form $\frac{0}{0}$ or $\frac{\infty}{\infty}$.  Here's how:

\vspace{1pc}
Assume $\displaystyle\lim_{x \to a} f(x)^{g(x)}$ has the indeterminate form $1^{\infty}$, $0^0$, or $\infty^0$.
\begin{itemize}
\item[1.] Evaluate $L=\displaystyle\lim_{x \to a} g(x) \ln f(x)$.  This limit can often be put in the form $\frac{0}{0}$ or $\frac{\infty}{\infty}$, which can be handled by l'H\^{o}pital's Rule.
\item[2.] Then $\displaystyle\lim_{x \to a} f(x)^{g(x)}=e^L$. \alert{Don't forget this step!}
\end{itemize}
\end{frame}

% % %
\begin{frame}
\frametitle{}
\small
\begin{ex} Evaluate $\displaystyle\lim_{x \to \infty} \left( 1+\dfrac{1}{x} \right)^x$. \end{ex}
\footnotesize
{\bf Solution:}  This is in the form $1^{\infty}$, so we need to examine
\begin{alignat*}{2}
L &= \lim_{x \to \infty} x \ln \left( 1 + \frac{1}{x} \right) \\
 &= \lim_{x \to \infty} \frac{\ln \left( 1 + \frac{1}{x} \right)}{ \frac{1}{x}} \\
 &\overset{\text{LR}}{=} \lim_{x \to \infty} \frac{ \dfrac{1}{1+ \frac{1}{x}}\left(-\frac{1}{x^2}\right) }{- \frac{1}{x^2}} \\ 
&= \lim_{x \to \infty} \dfrac{1}{1+ \frac{1}{x}} = 1.
\end{alignat*}
\end{frame}

% % %
\begin{frame}
NOT DONE!  Write
\[\lim_{x \to \infty} \left( 1+\dfrac{1}{x} \right)^x = e^L = e^1 = e.\]
\end{frame}

% % %
\subsubsection{Examining Growth Rates}
% % %

% % %
\begin{frame}{\small Examining Growth Rates}
\footnotesize
We can use l'H\^{o}pital's Rule to examine the rate at which functions grow in comparison to one another.
\begin{dfn} Suppose $f$ and $g$ are functions with $\displaystyle\lim_{x \to \infty} f(x) = \displaystyle\lim_{x \to \infty} g(x) = \infty$.  Then $\boldsymbol{f}$ {\bf grows faster than} $\boldsymbol{g}$ as $x \to \infty$ if 
$$\lim_{x \to \infty} \frac{g(x)}{f(x)} = 0\ \text{or}\ \lim_{x \to \infty} \frac{f(x)}{g(x)} = \infty.$$
$g \ll f$ means that $f$ grows faster than $g$ as $x \to \infty$.
\end{dfn}

\begin{dfn} The functions $f$ and $g$ have {\bf comparable growth rates} if 
$$\lim_{x \to \infty} \frac{f(x)}{g(x)}=M,\ \text{where}\ 0<M<\infty.$$
\end{dfn}
\end{frame}

% % %
\subsubsection{Pitfalls in Using L\^opital's Rule}
% % %

% % %
\begin{frame}{\small Pitfalls in Using l'H\^{o}pital's Rule}
\footnotesize
\begin{itemize}
\item[1.] L'H\^{o}pital's Rule says that $\displaystyle\lim_{x \to a}\frac{f(x)}{g(x)} = 
\displaystyle\lim_{x \to a}\frac{f^{\prime}(x)}{g^{\prime}(x)}$.  \alert{NOT} 
\[\lim_{x \to a}\frac{f(x)}{g(x)} = \lim_{x \to a}\left[ \frac{f(x)}{g(x)} \right]^{\prime}\ \text{or}\
\lim_{x \to a}\frac{f(x)}{g(x)} = \lim_{x \to a} \left[ \frac{1}{g(x)} \right]^{\prime} f^{\prime}(x)\]
(i.e., don't confuse this rule with the Quotient Rule).
\item[2.] Be sure that the limit with which you are working is in the form $\frac{0}{0}$ or $\frac{\infty}{\infty}$.
\item[3.] When using l'H\^{o}pital's Rule more than once, simplify as much as possible before repeating the rule.
\item[4.] If you continue to use l'H\^{o}pital's Rule in an unending cycle, another method must be used.
\end{itemize}
\end{frame}

% % %
\subsubsection{Book Problems}
% % %

% % % 
\begin{frame}
\begin{block}{4.7 Book Problems}
13-59 (odds), 69-79 (odds)
\end{block}
\end{frame}

\end{document}