\documentclass[cal1spr16Lectures.tex]{subfiles}
%\AtBeginSubsection{
%	\begin{frame}[allowframebreaks]{}
%	\begin{multicols}{2}
%	\tableofcontents[currentsubsection]
%	\end{multicols}
%	\end{frame}
%	}
	
\begin{document}

%\section[Week 5]{Week 5: 15-19 February}

% % %
\subsubsection{\bf Wednesday 17 February}
\begin{frame}[allowframebreaks]{Wed 17 Feb}
\begin{itemize}\footnotesize
\item Expect Exam back on Thursday.  Feedback on Friday.  Scores $\to$ MLP?
\item \alert{Instructions for when you get your exam back:}
\begin{itemize}\footnotesize
	\item Look over your test, but don't write on it.
	\item If you find discrepancies on points or grading, write your grievances on a separate sheet of paper.
	\item Return that paper with your exam to your drill instructor by the end of drill.
	\item Once you leave the room with your exam you lose this opportunity.
	\item This is the only way you can get points back on the exam.
\end{itemize}
\framebreak
\item MIDTERM in less than three weeks.
\begin{itemize}
	\item Tuesday 8 March 6-7:30p
	\item If you have legitimate conflict, i.e., anything that is also scheduled in ISIS, I need to know now.  If you are not sure if it conflicts with a course, please have that instructor contact me ASAP.
	\item Morning Section: Walker rm 124
	
	Afternoon Section: Walker rm 218
\end{itemize}
\item Later this month: Sub on Friday 26 Feb and Monday 29 Feb.
\end{itemize}
\end{frame}

% % %
\begin{frame}{}
\begin{exe} 
	\begin{itemize}
	\item[(a)] Find the slope of the line tangent to the curve $f(x)=x^3-4x-4$ at the point $(2,-4)$. 
	\item[(b)] Where does this curve have a horizontal tangent?
	\end{itemize}
\end{exe}	
\end{frame}

% % %
\subsubsection{Higher-Order Derivatives}
% % %

% % %
\begin{frame}{\small Higher-Order Derivatives}
If we can write the derivative of $f$ as a function of $x$, then we can take \emph{its} derivative, too.  The derivative of the derivative is called the {\bf second derivative} of $f$, and is denoted $f^{\prime\prime}$.  

\vspace{1pc}
In general, we can differentiate $f$ as often as needed.  If we do it $n$ times, the $n$th derivative of $f$ is 
\[\alert{f^{(n)}}(x)=\frac{\alert{d^n} f}{\alert{dx^n}}=\alert{\frac{d}{dx}}[\alert{f^{(n-1)}}(x)].\]
\end{frame}

% % %
\subsubsection{Book Problems}
% % %

% % %
\begin{frame}{}
\begin{block}{3.3 Book Problems} 9-48 (every 3rd problem), 51-53, 58-60 \end{block} 
\begin{itemize}
\item For these problems, use only the rules we have derived so far.
\end{itemize}
\end{frame}

% % %
\subsection[3.4 The Product and Quotient Rules]{\S 3.4 The Product and Quotient Rules}
% % %

% % %
\begin{frame}{\S 3.4 The Product and Quotient Rules}
Issue: Derivatives of products and quotients do \alert{NOT} behave like they do for limits.  
\end{frame}

% % %
\begin{frame}\footnotesize
As an example, consider $f(x)=x^2$ and $g(x)=x^3$.  We can try to differentiate their product in two ways:
\begin{itemize}\footnotesize
\item $\begin{aligned}[t]
	\frac{d}{dx}[f(x)g(x)] &= \frac{d}{dx}\left(x^5 \right) \\[0.25pc]
	 &= 5x^4
	\end{aligned}$
\item $\begin{aligned}[t]
	f^{\prime}(x)g^{\prime}(x) &= (2x)(3x^2) \\
	 &= 6x^3
	 \end{aligned}$
\end{itemize}
\begin{que}Which answer is the correct one? \end{que}
\end{frame}

% % %
\subsubsection{Product Rule}
% % %

% % %
\begin{frame}{\small Product Rule}\footnotesize
If $f$ and $g$ are any two functions that are differentiable at $x$, then
\[\alert{\frac{d}{dx}[f(x) g(x)] = f^{\prime}(x) g(x) + g^{\prime}(x) f(x)}.\]
In the example from the previous slide, we have
\begin{align*}
\frac{d}{dx}[x^2\cdot x^3] &= \frac{d}{dx}(x^2)\cdot (x^3)+x^2\cdot\frac{d}{dx}(x^3) \\
 &= (2x)\cdot (x^3)+x^2\cdot (3x^2) \\[0.25pc]
 &= 2x^4+3x^4 \\[0.25pc]
 &= 5x^4
\end{align*}
\end{frame}

% % %
\subsubsection{Derivation of the Product Rule}
% % %

% % %
\begin{frame}[allowframebreaks]{\small Derivation of the Product Rule}\footnotesize
\begin{align*}
&\frac{d}{dx}[f(x)g(x)] = \lim_{h \to 0} \frac{f(x+h)g(x+h)-f(x)g(x)}{h} \\[1pc]
 =\lim_{h \to 0} &\left(\frac{f(x+h)g(x+h)+\alert{[-f(x)g(x+h)+f(x)g(x+h)]}-f(x)g(x)}{h}\right) \\[0.5pc] 
 =\lim_{h \to 0} &\left(\frac{f(x+h)g(x+h)\alert{-f(x)g(x+h)}}{h}\right) \\
&\hspace{5pc} + \left(\lim_{h \to 0}\frac{\alert{f(x)g(x+h)}-f(x)g(x)}{h}\right) \\[0.5pc]
\end{align*} 

\framebreak
\begin{align*} 
 =\lim_{h \to 0} &\left(g(x+h) \frac{f(x+h)-f(x)}{h}\right) + \left(\lim_{h \to 0} f(x) \frac{g(x+h)-g(x)}{h}\right) \\[0.5pc]
 &=g(x)f^{\prime}(x)+f(x)g^{\prime}(x)
\end{align*}
\end{frame}

% % % 
\begin{frame}
\begin{exe} 
Use the product rule to find the derivative of the function $(x^2+3x)(2x-1)$.
\begin{itemize}
\item[A. ] $2(2x+3)$
\item[B. ] $6x^2+10x-3$
\item[C. ] $2x^3+5x^2-3x$
\item[D. ] $2x(x+3)+x(2x-1)$
\end{itemize}
\end{exe}
\end{frame}

% % %
\subsubsection{Derivation of the Quotient Rule}
% % %

% % %
\begin{frame}{\small Derivation of Quotient Rule}\footnotesize
\begin{que} Let $q(x)=\frac{f(x)}{g(x)}$.  What is $\frac{d}{dx}q(x)$? \end{que}
{\bf Answer:} We can write $f(x)=q(x) g(x)$ and then use the Product Rule:
\[f^{\prime}(x) = q^{\prime}(x) g(x) + g^{\prime}(x) q(x)\] 
and now solve for $q^{\prime}(x)$: 
\[q^{\prime}(x)=\frac{f^{\prime}(x)-q(x)g^{\prime}(x)}{g(x)}.\]
\end{frame} 

% % %
\begin{frame}{}\footnotesize
Then, to get rid of $q(x)$, plug in $\frac{f(x)}{g(x)}$:
\begin{align*}
q^{\prime}(x) &= \frac{f^{\prime}(x)-g^{\prime}(x)\alert{\frac{f(x)}{g(x)}}}{g(x)} \\[0.5pc]
 &= \frac{\alert{g(x)} \left( f^{\prime}(x)-g^{\prime}(x)\frac{f(x)}{g(x)} \right)}{\alert{g(x)}\cdot g(x)} \\[1pc]
\alert{\frac{d}{dx}\left(\frac{f(x)}{g(x)}\right)} & \alert{=\frac{f^{\prime}(x) g(x)-g^{\prime}(x)f(x)}{g(x)^2}}
\end{align*}
``LO-D-HI minus HI-D-LO  over LO squared"
\end{frame}

% % %
\subsubsection{Quotient Rule}
% % %

% % % 
\begin{frame}{\small Quotient Rule}
Just as with the product rule, the derivative of a quotient is not a quotient of derivatives, i.e.
\[\frac{d}{dx} \left[ \frac{f(x)}{g(x)}\right] \ne \frac{f^{\prime}(x)}{g^{\prime}(x)}.\]
Here is the correct rule, the Quotient Rule:
\[\frac{d}{dx} \left[ \frac{f(x)}{g(x)}\right] = \frac{f^{\prime}(x) g(x)-g^{\prime}(x) f(x)}{[g(x)]^2}.\]
\end{frame}

% % %
\begin{frame}
\begin{exe} Use the Quotient Rule to find the derivative of 
\[\frac{4x^3+2x-3}{x+1}.\]
\end{exe}
\begin{exe} Find the slope of the tangent line to the curve 
\[f(x)=\frac{2x-3}{x+1}\text{ at the point }(4,1).\] 
\end{exe}
\end{frame}

\end{document}