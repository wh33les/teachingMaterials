\documentclass[cal1spr16Lectures.tex]{subfiles}

\begin{document}

%\section[Week 10]{Week 10: 28 Mar - 1 Apr}

% % % 
\subsubsection{\bf Monday 28 March}
% % %

\begin{frame}[allowframebreaks]{Mon 28 Mar}
\begin{itemize}%\footnotesize
\item Exam 3: next week, probably Friday.  Covers \S 3.10-4.6 
\end{itemize}
\end{frame}

% % %
\subsection[4.3 Graphing Functions]{\S 4.3 Graphing Functions}
% % %

% % %
\begin{frame}{\S 4.3 Graphing Functions}
\footnotesize
{\bf Graphing Guidelines:}
\begin{itemize}
\item[1.] Identify the domain or interval of interest.
\item[2.] Exploit symmetry.
\item[3.] Find the first and second derivatives.
\item[4.] Find critical points and possible inflection points.
\item[5.] Find intervals on which the function is increasing or decreasing, and concave up/down.
\item[6.] Identify extreme values and inflection points.
\item[7.] Locate vertical/horizontal asymptotes and determine end behavior.
\item[8.] Find the intercepts.
\item[9.] Choose an appropriate graphing window and make a graph.
\end{itemize}
\end{frame}

% % %
\begin{frame}
\begin{exe} According to the graphing guidelines, sketch a graph of 
\[f(x)=\frac{x^2}{x^2-4}.\]
\end{exe}
\end{frame}

% % % 
\subsubsection{Book Problems}

% % %
\begin{frame}
\begin{block}{4.3 Book Problems}
7,8, 15-35 (odds), 45-53 
\end{block}
\end{frame}

\end{document}