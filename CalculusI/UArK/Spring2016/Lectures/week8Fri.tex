\documentclass[cal1spr16Lectures.tex]{subfiles}

\begin{document}

%\section[Week 8]{Week 8: 7-11 Mar}

% % % 
\subsubsection{\bf Friday 11 March}
% % %

\begin{frame}[allowframebreaks]{Fri 11 Mar}
\begin{itemize}\footnotesize
\item Exam 2: Curve, etc. is posted. 

\begin{center}
\includegraphics[scale=0.6]{..//Exam2Spread}
\end{center}
\framebreak
\item Midterm: expect it back next week in drill.  Don't expect a curve. :(
\item ``Fast Track Calculus": Dr. Kathleen Morris will be teaching a second 8 weeks Calculus One class.  ``If you have a student who is maybe doing poorly because of illness or a tragic event in their life during the beginning of the semester, this might be an opportunity for a new start for them."  The class requires departmental consent so the student will need to contact Kathleen to get permission to enroll.
\end{itemize}
\end{frame}

% % %
\subsection[3.10 Derivatives of Inverse Trigonometric Functions]{\S 3.10 Derivatives of Inverse Trigonometric Functions}
% % %

% % %
\begin{frame}{\S 3.10 Derivatives of Inverse Trigonometric Functions}{}\small
{\bf Recall:} If $y=f(x)$, then $f^{-1}(x)$ is the value of $y$ such that $x=f(y)$.
\begin{ex} If $f(x)=3x+2$, then what is $f^{-1}(x)$? \end{ex}
{\bf NOTE:}  $f^{-1}(x)\alert{\neq} f(x)^{-1}\textstyle\left(=\frac{1}{f(x)}\right)$

\vspace{0.5pc}
To avoid this confusion, we use $\arcsin{x},\,\arccos{x}\,\arctan{x},\dots$ to denote inverse trig functions.
\end{frame}

% % %
\subsubsection{Derivative of Inverse Sine}
% % %

% % %
\begin{frame}{\small Derivative of Inverse Sine}
Trig functions are functions, too.  Just like with ``$f\,$", there has to be something to ``plug in".  It makes no sense to just say $\sin$, without having $\sin(\text{\alert{\it something}})$.
\[y=\sin^{-1}x \Longleftrightarrow x=\sin y\]  
\end{frame}

% % %
\begin{frame}\footnotesize
The derivative of $y=\sin^{-1}x$ can be found using implicit differentiation: 
\begin{alignat*}{2}
x &= \sin y \\
\frac{d}{dx}(x) &= \frac{d}{dx}(\sin y) \\
1 &= (\cos y) \frac{dy}{dx} \\
\frac{dy}{dx} &= \frac{1}{\cos y}
\end{alignat*}
\end{frame}

% % %
\begin{frame}\footnotesize
We still need to replace $\cos y$ with an expression in terms of $x$.  We use the trig identity $\sin^2 y + \cos^2 y = 1$ (careful with notation: in this case we mean $\left(\sin y\right)^2+\left(\cos y\right)^2=1$).  Then 
\[\cos y= \pm \sqrt{1-\sin^2 y}=\pm\sqrt{1-x^2}.\]
The range of $y=\sin^{-1}x$ is $-\textstyle\frac{\pi}{2} \leq y \leq \frac{\pi}{2}$.  In this range, cosine is never negative, so we can just take the positive portion of the square root.
Therefore,
\[\frac{dy}{dx}=\frac{1}{\cos y}=\frac{1}{\sqrt{1-x^2}} \implies \alert{\frac{d}{dx}(\sin^{-1}x)=\frac{1}{\sqrt{1-x^2}}}.\]
\end{frame}

% % %
\begin{frame}{}
\begin{exe} Compute the following:
\begin{itemize}
\item[1.] $\frac{d}{dx} \left( \sin^{-1}(4x^2-3) \right)$
\item[2.] $\frac{d}{dx} \left( \cos(\sin^{-1}x) \right)$
\end{itemize}
\end{exe}
\end{frame}

% % %
\subsubsection{Derivative of Inverse Tangent}
% % %

% % %
\begin{frame}{\small Derivative of Inverse Tangent}\footnotesize
Similarly to inverse sine, we can let $y=\tan^{-1}x$ and use implicit differentiation:
\vspace{-0.5pc}
\begin{alignat*}{2}
x &= \tan y \\
\frac{d}{dx}(x) &= \frac{d}{dx} (\tan y) \\
1 &= (\sec^2 y) \frac{dy}{dx} \\
\frac{dy}{dx} &= \frac{1}{\sec^2 y}
\end{alignat*}
\end{frame}

% % %
\begin{frame}
Use the trig identity $\sec^2 y-\alert{\tan^2 y}=1$ to replace $\sec^2 y$ with $1+\alert{x^2}$:
\[\alert{\frac{d}{dx}(\tan^{-1} x)=\frac{1}{1+x^2}}\]
\end{frame}

% % %
\subsubsection{Derivative of Inverse Secant}
% % %

% % %
\begin{frame}{\small Derivative of Inverse Secant}\footnotesize
Again, use the same method as with inverse sine:
\begin{alignat*}{2}
y &= \sec^{-1}x \\
x &= \sec y \\
\frac{d}{dx}(x) &= \frac{d}{dx} (\sec y) \\
1 &= \sec y \tan y \frac{dy}{dx} \\
\frac{dy}{dx} &= \frac{1}{\sec y \tan y}
\end{alignat*}
\end{frame}

% % %
\begin{frame}\footnotesize
Use the trig identity $\sec^2 y-\tan^2 y=1$ again to get 
\[\tan y=\pm\sqrt{\sec^2 y-1}=\pm\sqrt{x^2-1}.\]
This time, the $\pm$ matters:
\vspace{-1.1pc}
\begin{columns}
\begin{column}{0.3\textwidth}
	\begin{center}\includegraphics[scale=0.49]{pictures/invSecpic}\end{center}
\end{column}
\begin{column}{0.6\textwidth}
	\begin{itemize}
	\item If $x \ge 1$, then $0\leq y<\textstyle\frac{\pi}{2}$ and so $\tan y >0.$  
	\item If $x \le -1$, then $\textstyle\frac{\pi}{2} < y \leq \pi$ and so  $\tan y <0$.  
	\end{itemize}
\end{column}
\end{columns}	
\end{frame}

% % %
\begin{frame}\footnotesize
Therefore,
\[\alert{\frac{d}{dx}(\sec^{-1} x)=\frac{1}{|x|\sqrt{x^2-1}}}.\]
\hrulefill

Using other trig identities (which you do not need to prove)
\[\cos^{-1}x+\sin^{-1}x=\frac{\pi}{2}\quad \cot^{-1}x+\tan^{-1}x=\frac{\pi}{2}\quad \csc^{-1}x+\sec^{-1}x=\frac{\pi}{2}\] 
we can get the rest of the inverse trig derivatives. 
\end{frame}

% % %
\subsubsection{All Other Inverse Trig Derivatives}
% % %

% % %
\begin{frame}{\small All Other Inverse Trig Derivatives}
To summarize:
\begin{columns}
\begin{column}{0.45\textwidth}
	\begin{itemize}
	\item[]$\textstyle\frac{d}{dx}(\sin^{-1}x)=\frac{1}{\sqrt{1-x^2}}$
	\item[]$\textstyle\frac{d}{dx}(\tan^{-1}x)=\frac{1}{1+x^2}$ 
	\item[]$\textstyle\frac{d}{dx}(\sec^{-1}x)=\frac{1}{|x|\sqrt{x^2-1}}$ 
	\end{itemize}
\end{column}
\begin{column}{0.6\textwidth}
	\begin{itemize}
	\item[] \alert{$\textstyle\frac{d}{dx}(\cos^{-1}x)=-\frac{1}{\sqrt{1-x^2}}$} \\
	$\quad (-1<x<1)$ 
	\item[] \alert{$\textstyle\frac{d}{dx}(\cot^{-1}x)=-\frac{1}{1+x^2}$} \\
	$\quad (-\infty<x<\infty) $ 
	\item[] \alert{$\textstyle\frac{d}{dx}(\csc^{-1}x)=-\frac{1}{|x|\sqrt{x^2-1}}$} \\
	$\quad (|x|>1)$ 
	\end{itemize}
\end{column}
\end{columns}
\end{frame}

% % %
\begin{frame}
\begin{ex} Compute the derivatives of $f(x)=\tan^{-1}\left(\textstyle\frac{1}{x}\right)$ and $g(x)=\sin \left(\sec^{-1}(2x) \right)$. \end{ex}
\end{frame}

% % %
\subsubsection{Derivatives of Inverse Functions in General}
% % %

% % %
\begin{frame}{\small Derivatives of Inverse Functions in General}\footnotesize
Let $f$ be differentiable and have an inverse on an interval $I$.  Let $x_0$ be a point in $I$ at which $f^{\prime}(x_0)\ne0$.  Then $f^{-1}$ is differentiable at $y_0=f(x_0)$ and 
\[\left(f^{-1}\right)^{\prime}(y_0)=\frac{1}{f^{\prime}(x_0)}\]
where $y_0=f(x_0)$.
\begin{ex} Let $f(x)=3x+4$.  Find $f^{-1}(x)$ and $\left(f^{-1}\right)^{\prime}\left(\textstyle\frac{1}{3}\right)$. \end{ex}
\end{frame}

% % %
\subsubsection{Book Problems}
% % %

% % %
\begin{frame}
\begin{block}{3.10 Book Problems} 7-33 (odds), 37-41 (odds) \end{block}
\end{frame}

% % %
\subsection[3.11 Related Rates]{\S 3.11 Related Rates}
% % %

% % %
\begin{frame}{\S 3.11 Related Rates}
\small 
In this section, we use our knowledge of derivatives to examine how variables change with respect to time.

\vspace{1pc}
The prime feature of these problems is that two or more variables, which are related in a known way, are themselves changing in time.

\vspace{1pc}
The goal of these types of problems is to determine the rate of change (i.e., the derivative) of one or more variables at a specific moment in time.
\end{frame}

% % %
\begin{frame}
\frametitle{}
\begin{block}{Problem}
The edges of a cube increase at a rate of 2 cm/sec.  How fast is the volume changing when the length of each edge is 50 cm?
\end{block}
\small
\begin{itemize}
\item {\bf Variables:}  $V$ (Volume of the cube) and $x$ (length of edge)
\item {\bf How Variables are related:}  $V=x^3$
\item {\bf Rates Known:}  $\dfrac{dx}{dt}=2\ \text{cm/sec}$
\item {\bf Rate We Seek:}  $\dfrac{dV}{dt}$ when $x=50\ \text{cm}$
\end{itemize}
\end{frame}

% % % 
\begin{frame}
\small
Note that both $V$ and $x$ are functions of $t$ (their respective sizes are dependent upon how much time has passed).

\vspace{1pc}
So we can write $V(t)=x(t)^3$ and then differentiate this with respect to $t$:
$$V^{\prime}(t) = 3 x(t)^2 \cdot x^{\prime}(t).$$

\vspace{1pc}
{\bf Note that $x(t)$ is the length of the cube's edges at time $t$, and $x^{\prime}(t)$ is the rate at which the edges are changing at time $t$.}
\end{frame}

% % %
\begin{frame}

We can rewrite the previous equation as

\[\frac{dV}{dt}=3x^2 \cdot \frac{dx}{dt}.\]

\vspace{1pc}
So the rate of change of the volume when $x=50\ \text{cm}$ is 
\[\left.\frac{dV}{dt}\right|_{x=50}=3\cdot 50^2 \cdot 2 = 15000\ \text{cm}^3/\text{sec}.\]
\end{frame}

\end{document}