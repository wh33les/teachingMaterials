\documentclass[cal1spr16Lectures.tex]{subfiles}

\begin{document}

%\section[Week 13]{Week 13: 18-22 Apr}

% % % 
\subsubsection{\bf Monday 18 April}
% % %

\begin{frame}[allowframebreaks]{Mon 18 Apr}
\begin{itemize}
\item April 22: Last day to drop with a "W".
\item Exam 4 next week, probably Friday.  Covers \S 4.7-5.4
\end{itemize}
\end{frame}

% % %
\subsubsection{Sigma Notation}
% % %

% % %
\begin{frame}{\small Sigma Notation}\footnotesize
Riemann sums become more accurate when we make $n$ (the number of rectangles) bigger, but obviously writing it all down is no fun!  Sigma notation gives a shorthand.  Here is how sigma notation works, through an example: 
\begin{ex} 
$\sum_{n=1}^5 n^2$ is the sum all integer values from the lowest limit ($n=1$) to the highest limit ($n=5$) in the summand $n^2$ (in this case $n$ is the indexing variable).  
\[\sum_{n=1}^5 \alert n^2 = \alert 1^2+\alert 2^2+\alert 3^2+\alert 4^2+\alert 5^2=55.\] 
\end{ex}
\end{frame}

% % %
\begin{frame}\small
\begin{ex} 
Evaluate $\sum_{k=0}^3 (2k-1)$. 
\end{ex}
\footnotesize
{\bf Solution:} In this example, $k$ is the indexing variable.  It starts at $0$ and goes to $3$, which means we write down the expression in the parentheses for each of the integers from $0$ to $3$, then add the results together:
\begin{align*}
\sum_{k=0}^3(2\alert k-1) &= (2(\alert 0)-1)+(2(\alert 1)-1)+(2(\alert 2)-1)+(2(\alert 3)-1) \\
	&= -1+1+3+5 = 8.
\end{align*}
\end{frame}

% % %
\subsubsection{$\Sigma$-Shortcuts}
% % %

% % %
\begin{frame}{\small $\Sigma$-Shortcuts}\footnotesize
($n$ is always a positive integer)
\vspace{-1pc}
\begin{align*}
\sum_{k=1}^n c & = cn \text{ (where $c$ is a constant)} \\%[0.25pc]
\sum_{k=1}^n k &= \dfrac{n(n+1)}{2} \\%[0.25pc]
\sum_{k=1}^n k^2 &= \dfrac{n(n+1)(2n+1)}{6} \\%[0.25pc]
\sum_{k=1}^n k^3 &= \dfrac{n^2(n+1)^2}{4}
\end{align*}
\vspace{-1.5pc}
\begin{que}
What is the indexing variable in these formulas?
\end{que}
\end{frame}

% % %
\subsubsection{Riemann Sums Using Sigma Notation}
% % %

% % %
\begin{frame}{\small Riemann Sums Using Sigma Notation}\footnotesize
Suppose $f$ is defined on a closed interval $[a,b]$ which is divided into $n$ subintervals of equal length $\Delta x$.  As before, $\overline{x}_k$ denotes a point in the $k$th subinterval $[x_{k-1},x_k]$, for $k=1,2,\dots,n$.  Recall that $x_0=a$ and $x_n=b$. 

\vspace{1pc}
Here is how we can write the Riemann sum in a much more compact form:
\vspace{-1pc}
\[
R = f(\overline{x}_1)\Delta x + f(\overline{x}_2)\Delta x + \cdots + f(\overline{x}_n)\Delta x 
 = \alert{\sum_{k=1}^n f(\overline{x}_k) \Delta x}.
\]
\end{frame}

% % %
\begin{frame}
With sigma-notation we can even derive explicit formulas for the basic Riemann sums (the expression in \alert{red} is $\overline x_k$ for each case:

\begin{itemize}
\item[1.] $\displaystyle\sum_{k=1}^n f(\alert{a+(k-1)\Delta x}) \Delta x=$ left Riemann sum
\item[2.] $\displaystyle\sum_{k=1}^n f(\alert{a+k\Delta x}) \Delta x=$  right Riemann sum
\item[3.] $\displaystyle\sum_{k=1}^n$\hspace{-1pt} $f(\alert{a+\left(k-\frac{1}{2}\right)\Delta x}) \Delta x=$ midpoint Riemann sum
\end{itemize}
\end{frame}

\begin{frame}\small
\begin{exe} 
\begin{itemize}
\item[(a)] Use sigma notation to write the left, right, and midpoint Riemann sums for the function $f(x)=x^2$ on the interval $[1,5]$ given that $n=4$.
\item[(b)] Based on these approximations, estimate the area bounded by the graph of $f(x)$ over $[1,5]$.
\end{itemize}
\end{exe}
{\bf Suggestion:} As $n$ gets very big, Riemann sums, along with the $\Sigma$-shortcuts plus algebra, often make the problem way more manageable. 
\end{frame}

% % %
\subsubsection{Book Problems}
% % %

% % %
\begin{frame}
\begin{block}{5.1 Book Problems}
9-37
\end{block}
\end{frame}

% % %
\subfile{5p2definiteIntegrals}

\end{document}