\documentclass[cal1spr16Lectures.tex]{subfiles}
%\AtBeginSubsection{
%	\begin{frame}[allowframebreaks]{}
%	\begin{multicols}{2}
%	\tableofcontents[currentsubsection]
%	\end{multicols}
%	\end{frame}
%	}
	
\begin{document}

%\section[Week 1]{Week 1: 19-22 January}

% % %
\subsubsection{\bf Wednesday 20 January}
\begin{frame}[allowframebreaks]{Wed 20 Jan}
Welcome to Cal I!
\begin{itemize}\footnotesize
\item \href{http://comp.uark.edu/~ashleykw/Cal1Spring2016/cal1spr16.html}{\alert{\url{comp.uark.edu/~ashleykw/Cal1Spring2016/cal1spr16.html}}} \\
	Course website.  All information is here, including a link to MLP, lecture slides, administrative information, etc.  You should have already seen the 
	\href{http://comp.uark.edu/~ashleykw/Cal1Spring2016/syllabusCal1Spring2016.pdf}{\alert{\textbf{syllabus}}}
	by now.
\item MyLabsPlus (MLP) has the graded homework.   Solutions to Quizzes and Drill exercises will be posted there, under ``Menu $\to$ Course Tools $\to$ Document Sharing".  
\framebreak 
\item Lecture slides are available on the course website.  I'll try to have the week's slides posted in advance but the individual lectures might not be posted until right before class.  \textbf{Don't try to take notes from the slides.}  Instead, print out the slides beforehand or else follow along on your tablet/phone/laptop.  You should, however, take notes when we do exercises during lecture.% (which is frequent).  We will always review those solutions on the document camera.  Document camera notes are reserved only for those who attended lecture that day.  :p
\item For old Calculus materials, see the parent page \url{comp.uark.edu/~ashleykw} and look for links under ``Previous Semesters".  
\end{itemize}
\end{frame}

% % %
\subsection[2.1 The Idea of Limits]{$\S$2.1 The Idea of Limits}
% % %

% % % 
\begin{frame}{\S 2.1 The Idea of Limits}
\begin{que} How would you define, and then differentiate between, the following pairs of terms? 
\begin{itemize}
\item instantaneous velocity vs. average velocity?
\item tangent line vs. secant line? 
\end{itemize}
\end{que}

(Recall: What is a tangent line and what is a secant line?)
\end{frame}

% % %
\begin{frame}\footnotesize
\begin{ex} An object is launched into the air. Its position $s$ (in feet) at any time $t$ (in seconds) is given by the equation:
\[s(t)=-4.9t^2+30t+20.\]
\hrulefill
\begin{itemize}
\item[(a)] Compute the average velocity of the object over the following time intervals:  $[1,3],\,[1,2],\,[1,1.5]$
\item[(b)] As your interval gets shorter, what do you notice about the average velocities?  What do you think would happen if we computed the average velocity of the object over the interval $[1,1.2]$? $[1,1.1]$? $[1,1.05]$?
\end{itemize}
\end{ex}
\end{frame}

% % %
\begin{frame}\footnotesize
\begin{block}{Example, cont.} An object is launched into the air. Its position $s$ (in feet) at any time $t$ (in seconds) is given by the equation:
\[s(t)=-4.9t^2+30t+20.\]
\hrulefill
\begin{itemize}
\item[(c)] How could you use the average velocities to estimate the instantaneous velocity at $t=1$?
\item[(d)] What do the average velocities you computed in 1.\ represent on the graph of $s(t)$?
\end{itemize}
\end{block}
\end{frame}

% % %
\begin{frame}
\begin{que} What happens to the relationship between \alert{instantaneous} velocity and \alert{average} velocity as the time interval gets shorter? \end{que}

{\bf Answer:} The instantaneous velocity at $t=1$ is the limit of the average velocities as $t$ approaches 1.
\end{frame}

% % %
\begin{frame}
\begin{que} What about the relationship between the \alert{secant} lines and the \alert{tangent} lines as the time interval gets shorter? \end{que}

{\bf Answer:} The slope of the tangent line at $(1, 45.1=s(1))$ is the limit of the slopes of the secant lines as $t$ approaches 1.
\end{frame}

% % %
\subsubsection{Book Problems}
\begin{frame}
\begin{block}{2.1 Book Problems}1-3, 7-13, 15, 21, 25, 27, 29\end{block}
%\begin{itemize}
%\item 
Even though book problems aren't turned in, they're a very good way to study for quizzes and tests (wink wink wink).  
%\end{itemize}
\end{frame}

% % %
\subsection[2.2 Definition of Limits]{\S 2.2 Definition of Limits}
% % %

% % %
\begin{frame}{\S 2.2 Definition of Limits}
\begin{que} 
\begin{itemize}
	\item Based on your everyday experiences, how would you define a ``limit"? 
	\item Based on your mathematical experiences, how would you define a ``limit"?
	\item How do your definitions above compare or differ?
\end{itemize}
\end{que}
\end{frame}

\end{document}