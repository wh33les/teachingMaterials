\documentclass[cal1spr16Lectures.tex]{subfiles}
%\AtBeginSubsection{
%	\begin{frame}[allowframebreaks]{}
%	\begin{multicols}{2}
%	\tableofcontents[currentsubsection]
%	\end{multicols}
%	\end{frame}
%	}
	
\begin{document}

%\section[Week 2]{Week 2: 25-29 Jan}

% % %
\subsubsection{\bf Wednesday 27 January}
\begin{frame}[allowframebreaks]{Wed 27 Jan}
\begin{itemize}\footnotesize
\item GET YOUR CLICKER.  Starting next week, no attendance sheet, clickers only.	
\item There is no Blackboard for this course.
\item Stay on top of the MLP!  First deadline is coming up.  Don't wait till the last minute.
\end{itemize}
\end{frame}

% % %
\subsubsection{Definition of a Vertical Asymptote}
\begin{frame}{\small Definition of Vertical Asymptote}
\begin{dfn} Suppose a function $f$ satisfies at least one of the following: 

\begin{itemize}
\item $\displaystyle\lim_{x \to a} f(x) = \pm\infty$,
\item $\displaystyle\lim_{x \to a^+} f(x) = \pm\infty$
\item $\displaystyle\lim_{x \to a^-} f(x) = \pm\infty$
\end{itemize}

Then the line $x=a$ is called a {\bf vertical asymptote} of $f$. \end{dfn}
\end{frame}

% % %
\begin{frame}
\begin{exe} Given $f(x)=\frac{3x-4}{x+1}$, determine, analytically (\alert{meaning using ``number sense" and without a table or a graph}), 

\begin{itemize}
\item[(a)\;] $\displaystyle\lim_{x \to -1^+} f(x)$ 
\item[(b)\;] $\displaystyle\lim_{x \to -1^-} f(x)$
\end{itemize}
\end{exe}
\end{frame}

% % %
\subsubsection{Summary Statements}
% % %
\begin{frame}{\small Summary Statements}
Here is a common way you can summarize your solutions involving limits:

\vspace{1pc}
``Since the numerator approaches \alert{(\#)} and the denominator approaches $0$, and is \alert{(positive/negative)}, and since \alert{(analyze signs here)}, \alert{(insert limit problem)}=\alert{($+\infty\,/-\infty$)}."
\end{frame}

% % %
\begin{frame}
Remember to check for factoring -- 

%\vspace{1.5pc}
\begin{exe} \begin{itemize}
\item[(a)] What is/are the vertical asymptotes of 
\[f(x)=\frac{3x^2-48}{x+4} ?\]
\item[(b)] What is $\displaystyle\lim_{x \to -4} f(x)$?  Does that correspond to your earlier answer?
\end{itemize}
\end{exe}
\end{frame}

% % %
\subsubsection{Book Problems}
\begin{frame}
\begin{block}{2.4 Book Problems} 7-10, 15, 17-23, 31-34, 44-45 \end{block}
\end{frame}

% % % 
\subsection[2.5 Limits at Infinity]{\S 2.5 Limits at Infinity}
% % %

% % %
\begin{frame}{\S 2.5 Limits at Infinity}\footnotesize
Limits at infinity determine what is called the {\bf end behavior} of a function.
\begin{exe}
\begin{itemize}
\item[(a) ]Evaluate the following functions at the points $x=\pm100,\pm1000,\pm10000$;
\[f(x)=\frac{4x^2+3x-2}{x^2+2}\qquad g(x)=-2+\frac{\cos{x}}{\sqrt[3]{x}}
\]
\item[(b) ] What is your conjecture about $\lim_{x\to\infty}f(x)$?  $\lim_{x\to-\infty}f(x)$?  $\lim_{x\to-\infty}g(x)$?  $\lim_{x\to\infty}g(x)$?
\end{itemize}
\end{exe}
\end{frame}

% % %
\subsubsection{Horiztonal Asymptotes}
\begin{frame}{\small Horizontal Asymptotes}\footnotesize
\begin{dfn} If $f(x)$ becomes arbitrarily close to a finite number $L$ for all sufficiently large and positive $x$, then we write 
\[\lim_{x \to \infty}f(x)=L.\]
The line $y=L$ is a {\bf horizontal asymptote} of $f$. \end{dfn}  
The limit at negative infinity, $\displaystyle\lim_{x \to -\infty}f(x)=M$, is defined analogously and in this case, the horizontal asymptote is $y=M$.
\end{frame}

% % %
\subsubsection{Infinite Limits at Infinity}
\begin{frame}{\small Infinite Limits at Infinity}\footnotesize
\begin{que} Is it possible for a limit to be both an infinite limit and a limit at infinity? (Yes.) \end{que}

\vspace{0.5pc}
If $f(x)$ becomes arbitrarily large as $x$ becomes arbitrarily large, then we write 
\[\displaystyle\lim_{x \to \infty}f(x)=\infty.\]  

\vspace{0.5pc}
(The limits 
$\displaystyle\lim_{x \to \infty}f(x)=-\infty$, $\displaystyle\lim_{x \to -\infty}f(x)=\infty$, and $\displaystyle\lim_{x \to -\infty} f(x)=-\infty$ are defined similarly.)
\end{frame}

% % %
\begin{frame}{}{}%{\small Infinite Limits at Infinity, cont.}{}
{\bf Powers and Polynomials:}  Let $n$ be a positive integer and let $p(x)$ be a polynomial.

\vspace{1pc}
\begin{itemize}
\item $n=$ even number: $\displaystyle\lim_{x\to\pm\infty}x^n=\infty$ 

\vspace{1pc}
\item $n=$ odd number: $\displaystyle\lim_{x \to \infty} x^n = \infty$ and $\displaystyle\lim_{x \to -\infty} x^n = -\infty$
\end{itemize}
\end{frame}

% % %
\begin{frame}\footnotesize
\begin{itemize}
\item (again, assuming $n$ is positive)
\[\lim_{x\to\pm\infty}\frac{1}{x^n}=\lim_{x\to\pm\infty}x^{-n}=0\]

%\vspace{0.5pc}
\item For a polynomial, only look at the term with the highest exponent:
\[\lim_{x\to\pm\infty}p(x)=\lim_{x\to\pm\infty}\text{(constant)}\cdot x^n\] 
The constant is called the {\bf leading coefficient}, lc$(p)$.  The highest exponent that appears in the polynomial is called the {\bf degree}, $\deg(p)$.
\end{itemize}
\end{frame}

% % %
\begin{frame}\footnotesize
{\bf Rational Functions:}  Suppose $f(x)=\dfrac{p(x)}{q(x)}$ is a rational function.

\begin{itemize}
\item If $\deg(p)<\deg(q)$, i.e., \alert{the numerator has the smaller degree}, then 

\vspace{-1pc}
\[\lim_{x\to\pm\infty}f(x)=0\] 

\vspace{-0.5pc}
and $y=0$ is a horizontal asymptote of $f$.

\vspace{0.5pc}
\item If $\deg(p)=\deg(q)$, i.e., \alert{numerator and denominator have the same degree}, then 

\vspace{-1.5pc}
\[\lim_{x\to\pm\infty}f(x)=\dfrac{\text{lc}(p)}{\text{lc}(q)}\] 

\vspace{-0.5pc}
and $y=\textstyle\frac{\text{lc}(p)}{\text{lc}(q)}$ is a horizontal asymptote of $f$.
\end{itemize}
\end{frame}

% % %
\begin{frame}\footnotesize
\begin{itemize}
\item If $\deg(p)>\deg(q)$, \alert{(numerator has the bigger degree)} then 
\[\lim_{x\to\pm\infty}f(x)=\infty\quad\text{or}\quad -\infty\] 
and $f$ has no horizontal asymptote.

\vspace{0.5pc}
\item Assuming that $f(x)$ is in \alert{reduced form} ($p$ and $q$ share no common factors), vertical asymptotes occur at the zeroes of $q$.  

\vspace{0.5pc}
(This is why it is a good idea to check for factoring and cancelling first!)
\end{itemize}
\end{frame}

% % %
\begin{frame}%\footnotesize
When evaluating limits at infinity for rational functions, it is not enough to use the previous rule to show the limit analytically.

\vspace{1pc}
To evaluate these limits, we divide both numerator and denominator by $x^n$, where $n$ is the degree of the polynomial in the denominator.
\end{frame}

\end{document}