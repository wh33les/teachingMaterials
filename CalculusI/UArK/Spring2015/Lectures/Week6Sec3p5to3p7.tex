\documentclass[14pt]{beamer}
\usetheme{Warsaw}
\usecolortheme{beaver}
\usefonttheme{professionalfonts}

\input{../../preamble}
\usepackage{amscd,amsmath,amssymb,amsthm,graphicx}
\usepackage[mathscr]{eucal}
\usepackage{paralist}
\usepackage{tabto}
\usepackage[normalem]{ulem}

% % % % % % % % % %
\title[Cal I S2015]{MATH 2554 (Calculus I)}
\subtitle{}
\author[Wheeler]{Dr. Ashley K. Wheeler}
\institute{University of Arkansas}
\date{\today}
\logo{}

% % %
\begin{document}
\maketitle

% % %
\begin{frame}
\frametitle{Table of Contents}
\tableofcontents
\end{frame}

% % % % % % % % % % Mon 16 Feb 2015

% % %
\begin{frame}
\section[Week 6]{Week 6: 16-20 February}
\frametitle{Monday 16 February (Week 6)}
\begin{itemize}
\item SNOW DAY (no class)
\item Read $\oint 3.5$.  We begin $\oint 3.6$ on Wednesday.
\end{itemize}
\end{frame}

% % %
\begin{frame}
\subsection[3.5 Derivatives as Rates of Change]{$\oint$ 3.5 Derivatives as Rates of Change}
\frametitle{$\oint$ 3.5 Derivatives as Rates of Change}
\small
{\bf Position and Velocity}
Suppose an object moves along a straight line and its location at time $t$ is given by the position function $s=f(t).$  

\bigskip 

The {\bf displacement} of the object between $t=a$ and $t=a+\Delta t$ is 
$$\Delta s = f(a+\Delta t)-f(a).$$

\bigskip

Here $\Delta t$ represents how much time has elapsed.
\end{frame}

% % %
\begin{frame}

We now define average velocity as 
$$\frac{\Delta s}{\Delta t}=\frac{f(a+\Delta t)-f(a)}{\Delta t}.$$

\bigskip

Recall that the limit of the average velocities as the time interval approaches 0 was the instantaneous velocity (which we denote here by $v$).  Therefore, the instantaneous velocity at $a$ is 
$$v(a)=\lim_{\Delta t \to 0} \frac{f(a+\Delta t)-f(a)}{\Delta t} = f^{\prime}(a).$$
\end{frame}

% % %
\begin{frame}

In mathematics, speed and velocity are related but not the same:

\bigskip

If the {\it velocity} of an object at any time $t$ is given by $v(t)$, then the {\it speed} of the object at any time $t$ is given by 
$$|v(t)|=|f^{\prime}(t)|.$$
\end{frame}

% % %
\begin{frame}
\small
By definition, acceleration (denoted by $a$) is the instantaneous rate of change of the velocity of an object at time $t$.

\bigskip

Therefore,
$$a(t)=v^{\prime}(t)$$
and since velocity was the derivative of the position function $s=f(t)$, then 
$$a(t)=v^{\prime}(t)=f^{\prime\prime}(t).$$

\bigskip

Summary:  Given the position function $s=f(t)$, the velocity at time $t$ is the first derivative, the speed at time $t$ is the absolute value of the first derivative, and the acceleration at time $t$ is the second derivative.
\end{frame}

% % %
\begin{frame}

Question:  Given the position function $s=f(t)$ of an object launched into the air, how would you know:

\begin{itemize}

\item[1.] The highest point the object reaches?

\item[2.] How long it takes to hit the ground?

\item[3.] The speed at which the object hits the ground?

\end{itemize}
\end{frame}

% % %
\begin{frame}
\frametitle{Growth Models}
Suppose $p=f(t)$ is a function of the growth of some quantity of interest.  The average growth rate of $p$ between times $t=a$ and a later time $t=a+\Delta t$ is the change in $p$ divided by the elapsed time $\Delta t$:
$$\frac{\Delta p}{\Delta t}=\frac{f(a+\Delta t)-f(a)}{\Delta t}.$$
\end{frame}

% % %
\begin{frame}

As $\Delta t$ approaches 0, the average growth rate approaches the derivative $\dfrac{dp}{dt}$, which is the instantaneous growth rate (or just simply the growth rate).  Therefore,
$$\frac{dp}{dt}=\lim_{\Delta t \to 0} \frac{f(a+\Delta t)-f(a)}{\Delta t} = \lim_{\Delta t \to 0} \frac{\Delta p}{\Delta t}.$$
\end{frame}

% % %
\begin{frame}
\frametitle{Exercise}
The population of the state of Georgia (in thousands) from 1995 ($t=0$) to 2005 ($t=10$) is modeled by the polynomial 
$$p(t)=-0.27t^2+101t+7055.$$

\begin{itemize}

\item[1.] What was the average growth rate from 1995 to 2005?
\item[2.] What was the growth rate for Georgia in 1997?
\item[3.] What can you say about the population growth rate in Georgia between 1995 and 2005?

\end{itemize}
\end{frame}

% % %
\begin{frame}
\frametitle{Average and Marginal Cost}
Suppose a company produces a large amount of a particular quantity.  Associated with manufacturing the quantity is a {\bf cost function} $C(x)$ that gives the cost of manufacturing $x$ items.  This cost may include a {\bf fixed cost} to get started as well as a {\bf unit cost} (or {\bf variable cost}) in producing one item.
\end{frame}

% % %
\begin{frame}

If a company produces $x$ items at a cost of $C(x)$, then the average cost is $\dfrac{C(x)}{x}.$

\bigskip

This average cost indicates the cost of items already produced.  Having produced $x$ items, the cost of producing another $\Delta x$ items is 
$C(x+\Delta x)-C(x)$.  So the average cost of producing these extra $\Delta x$ items is 
$$\frac{\Delta C}{\Delta x}=\frac{C(x+\Delta x)-C(x)}{\Delta x}.$$
\end{frame}

% % %
\begin{frame}
If we let $\Delta x$ approach 0, we have
$$\lim_{\Delta x \to 0}\frac{\Delta C}{\Delta x}=C^{\prime}(x)$$
which is called the {\bf marginal cost}.  

\bigskip 

The marginal cost is the approximate cost to produce one additional item after producing $x$ items.

\bigskip

{\bf Note:}  In reality, we can't let $\Delta x$ approach 0 because $\Delta x$ represents whole numbers of items.
\end{frame}

% % %
\begin{frame}%[t]
\frametitle{Exercise}
If the cost of producing $x$ items is given by 
$$C(x)=-0.04x^2+100x+800$$
for $0 \le x \le 1000$, find the average cost and marginal cost functions.  Also, determine the average and marginal cost when $x=500.$
\end{frame}

% % %
\begin{frame}
\frametitle{HW from Section 3.5}
Do problems 9--12, 17--18, 22--23, 27--37 odd (pp.\ 171--175 in textbook).
\end{frame}

% % % % % % % % % % Wed 18 Feb 2015

% % %
\begin{frame}
\frametitle{Wednesday 18 February (Week 6)}
\begin{itemize}
\item Quiz 5 due tomorrow in drill.
\item Exam 2 Friday 27 Feb (next week!)
\item Midterm the following Tues.
\end{itemize}
\end{frame}

% % %
\begin{frame}
\subsection[3.6 The Chain Rule]{$\oint$ 3.6 The Chain Rule}
\frametitle{$\oint$ 3.6 The Chain Rule}
\small
Suppose that Yvonne ($y$) can run twice as fast as Uma ($u$). Therefore $$\dfrac{dy}{du}=2.$$  Suppose that Uma can run four times as fast as Xavier ($x$).  So 
$$\dfrac{du}{dx}=4.$$

\bigskip

How much faster can Yvonne run than Xavier?

\bigskip

In this case, we would take both our rates and multiply them together:
$$\frac{dy}{du} \cdot \frac{du}{dx}=2 \cdot 4 = 8.$$
\end{frame}

% % %
\begin{frame}
\frametitle{Version 1 of the Chain Rule}
If $g$ is differentiable at $x$, and $y=f(u)$ is differentiable at $u=g(x)$, then the composite function $y=f(g(x))$ is differentiable at $x$, and its derivative can be expressed as 
$$\frac{dy}{dx}=\frac{dy}{du} \cdot \frac{du}{dx}$$
\end{frame}

% % %
\begin{frame}
\frametitle{Guidelines for Using the Chain Rule}
\small
Assume the differentiable function $y=f(g(x))$ is given.
\begin{itemize}
\item[1.] Identify the outer function $f$, the inner function $g$, and let $u=g(x).$
\item[2.] Replace $g(x)$ by $u$ to express $y$ in terms of $u$:
$$y=f(g(x)) \implies y=f(u)$$
\item[3.]  Calculate the product $\dfrac{dy}{du} \cdot \dfrac{du}{dx}$
\item[4.] Replace $u$ by $g(x)$ in $\dfrac{dy}{du}$ to obtain $\dfrac{dy}{dx}.$
\end{itemize}
\end{frame}

% % %
\begin{frame}
\footnotesize
{\bf Example:} Use Version 1 of the Chain Rule to calculate $\dfrac{dy}{dx}$ for $y=(5x^2 +11x)^{20}$.

\begin{itemize}
\item inner function: $u=5x^2+11x$ 
\item outer function: $y=u^{20}$
\end{itemize}

\vspace{1pc}
We have $y = f(g(x)) = (5x^2 +11x)^{20}$.  Differentiate:
\begin{align*} 
\dfrac{dy}{dx}= \dfrac{dy}{du}\cdot\dfrac{du}{dx} &= 20u^{19} \cdot (10x+11) \\
 &=20(5x^2 +11x)^{19} \cdot (10x+11)
\end{align*}
\end{frame}

% % %
\begin{frame}%[t]
Use the first version of the Chain Rule to calculate $\dfrac{dy}{dx}$ for $$y=\left( \dfrac{3x}{4x+2} \right)^5.$$
\end{frame}

% % %
\begin{frame}
\frametitle{Version 2 of the Chain Rule}
\small
Notice if $y=f(u)$ and $u=g(x)$, then $y=f(u)=f(g(x))$, so we can also write:
\begin{align*}
\dfrac{dy}{dx} &= \dfrac{dy}{du} \cdot \dfrac{du}{dx} \\[0.75pc]
 &= f^{\prime}(u) \cdot g^{\prime}(x) \\[0.75pc]
 &= f^{\prime}(g(x)) \cdot g^{\prime}(x).
\end{align*}
\end{frame}

% % %
\begin{frame}
\footnotesize
Use Version 2 of the Chain Rule to calculate $\dfrac{dy}{dx}$ for $y=(7x^4+2x+5)^9.$
\begin{itemize}
\item inner function: $g(x)=7x^4+2x+5$ 
\item outer function: $f(u)=u^9$
\end{itemize}
Then
\begin{align*}
f^{\prime}(u) &= 9u^8 \implies f^{\prime}(g(x))=9(7x^4+2x+5)^8 \\
g^{\prime}(x) &=28x^3+2.
\end{align*}

Putting it together,
$$\frac{dy}{dx}=f^{\prime}(g(x)) \cdot g^{\prime}(x) = 9(7x^4+2x+5)^8 \cdot (28x^3+2)$$
\end{frame}

% % % 
\begin{frame}
\frametitle{Chain Rule for Powers}
\small
If $g$ is differentiable for all $x$ in the domain and $n$ is an integer, then
$$\frac{d}{dx} \bigg[\left(g(x)\right)^n \bigg]=n(g(x))^{n-1} \cdot g^{\prime}(x).$$

\vspace{1pc}
Example:
\begin{align*}
\frac{d}{dx} \bigg[ (1-e^x)^4 \bigg] &= 4(1-e^x)^3 \cdot (-e^x) \\
 &= -4e^x (1-e^x)^3
\end{align*}
\end{frame}

% % %
\begin{frame}[t]
\frametitle{Composition of 3 or more functions}
\footnotesize
Compute $\dfrac{d}{dx} \bigg[ \sqrt{(3x-4)^2 + 3x} \bigg]$.
\begin{alignat*}{2}
\dfrac{d}{dx} \bigg[ \sqrt{(3x-4)^2 + 3x} \bigg] &= \dfrac{1}{2} \big( (3x-4)^2 + 3x \big)^{-\frac{1}{2}} \cdot \dfrac{d}{dx} \big[ (3x-4)^2 + 3x \big] \\
&= \dfrac{1}{2 \sqrt{ \big( (3x-4)^2 + 3x \big)}} \cdot \bigg[ 2(3x-4) \dfrac{d}{dx}(3x-4)  + 3 \bigg] \\
&= \dfrac{1}{2 \sqrt{ \big( (3x-4)^2 + 3x \big)}} \cdot \big[ 2(3x-4) \cdot 3 + 3 \big] \\
&= \dfrac{18x-21}{2 \sqrt{ \big( (3x-4)^2 + 3x \big)}} 
\end{alignat*}
\end{frame}

% % %
\begin{frame}
\frametitle{HW from Section 3.6}
Do problems 7--29 odd, 30, 33--43 odd, 49 (pp.\ 180--181 in textbook)
\end{frame}

% % % % % % % % % % Fri 20 Feb 2015

% % %
\begin{frame}
\frametitle{Friday 20 February (Week 6)}
\begin{itemize}
\item Exam 2 Friday 27 Feb (next week!)
\item Midterm the following Tues.
\end{itemize}
\end{frame}

% % %
\begin{frame}
\subsection[3.7 Implicit Differentiation]{$\oint$ 3.7 Implicit Differentiation}
\frametitle{$\oint$ 3.7 Implicit Differentiation}
\small
Up to now, we have calculated derivatives of functions of the form $y=f(x)$, where $y$ is defined {\bf explicitly} in terms of $x$.

\bigskip

In this section, we examine relationships between variables that are {\bf implicit} in nature, meaning that $y$ either is not defined explicitly in terms of $x$ or cannot be easily manipulated to solve for $y$ in terms of $x$.
\end{frame}

% % %
\begin{frame}
\frametitle{Examples of functions implicitly defined}
$$x^2 + y^2 = 9$$

\bigskip

$$x+y^3-xy=4$$

\bigskip

$$\cos(x-y)+\sin y = \sqrt{2}$$
\end{frame}

% % %
\begin{frame}

The goal of {\bf implicit differentiation} is to find a single expression for the derivative directly from an equation of the form $F(x,y)=0$ without first solving for $y$.
\end{frame}

% % %
\begin{frame}

Calculate $\dfrac{dy}{dx}$ directly from the equation for the circle 
$$x^2 + y^2 = 9.$$

\bigskip

{\bf Solution:}  To note that $x$ is our independent variable and that we are differentiating with respect to $x$, we replace $y$ with $y(x)$:
$$x^2 + (y(x))^2 = 9.$$
\end{frame}

% % %
\begin{frame}
\small
Now differentiate each term with respect to $x$:
$$\frac{d}{dx} (x^2) + \frac{d}{dx} ((y(x))^2) = \frac{d}{dx}(9).$$

\bigskip

By the Chain Rule, $\dfrac{d}{dx}((y(x))^2)=2y(x) y^{\prime}(x)$, or $\dfrac{d}{dx}(y^2)=2y \dfrac{dy}{dx}.$

\bigskip

So
$$2x+2y \dfrac{dy}{dx}=0 \implies \dfrac{dy}{dx}=\dfrac{-2x}{2y}=-\dfrac{x}{y}.$$
\end{frame}

% % %
\begin{frame}
\small
Now $\dfrac{dy}{dx}=-\dfrac{x}{y}$, so we can find slopes of tangent lines at various points along the circle.

\bigskip

The slope of the tangent line at (0,3) is 
\[\left. \dfrac{dy}{dx} \right|_{(x,y)=(0,3)} = -\dfrac{0}{3}=0.\]

\bigskip

The slope of the tangent line at $(1,2\sqrt{2})$ is
\[\left. \dfrac{dy}{dx} \right|_{(x,y)=(1,2\sqrt{2})} = -\dfrac{1}{2\sqrt{2}}.\]
\end{frame}

% % %
\begin{frame}%[t]
\frametitle{Example}
Find $\dfrac{dy}{dx}$ for $xy+y^3=1.$
\end{frame}

% % %
\begin{frame}%[t]
\frametitle{Finding tangent lines}
Find an equation of the line tangent to the curve $x^4-x^2 y+y^4=1$ at the point $(-1,1).$
\end{frame}

% % %
\begin{frame}%[t]
\frametitle{Higher Order Derivatives}
Find $\dfrac{d^2 y}{dx^2}$ if $xy+y^3=1.$
\end{frame}

% % %
\begin{frame}
\frametitle{Power Rule for Rational Exponents}
Implicit differentiation also allows us to extend the power rule to rational exponents:

\bigskip

Assume $p$ and $q$ are integers with $q \ne 0$.  Then 
$$\frac{d}{dx} (x^{p/q})=\frac{p}{q} x^{p/q-1}$$
provided $x \ge 0$ when $q$ is even.
\end{frame}

% % %
\begin{frame}
\frametitle{HW from Section 3.7}

Do problems 5--21 odd, 27--45 odd (pp.\ 188--189 in textbook)
\end{frame}

\begin{comment}
\end{comment}

\end{document}