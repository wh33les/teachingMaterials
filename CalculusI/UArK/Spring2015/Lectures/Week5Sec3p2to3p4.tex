\documentclass[14pt]{beamer}
\usetheme{Warsaw}
\usecolortheme{beaver}
\usefonttheme{professionalfonts}

\input{../../preamble}
\usepackage{amscd,amsmath,amssymb,amsthm,graphicx}
\usepackage[mathscr]{eucal}
\usepackage{paralist}
\usepackage{tabto}
\usepackage[normalem]{ulem}

% % % % % % % % % %
\title[Cal I S2015]{MATH 2554 (Calculus I)}
\subtitle{}
\author[Wheeler]{Dr. Ashley K. Wheeler}
\institute{University of Arkansas}
\date{\today}
\logo{}

% % %
\begin{document}
\maketitle

% % %
\begin{frame}
\frametitle{Table of Contents}
\tableofcontents
\end{frame}

% % %
\begin{frame}
\section[Week 5]{Week 5: 9-13 February}
\frametitle{Monday 9 February (Week 5)}
\begin{itemize}
\item Quiz \#4 tomorrow DURING drill
\item Quiz \#5 distributed Thurs
\item Exam \#1 results: 
	\begin{itemize}
	\item Exams returned tomorrow during drill.
	\item Median raw scores for each problem are on the webpage, along with the curved grading scale.
	\item Out of 75 points, including your signature on the cover page.
	\end{itemize}
\end{itemize}
\end{frame}

% % %
\begin{frame}
\begin{itemize}
\small
\item Points back: 
	\begin{itemize}
	\footnotesize
	\item The number written on the cover page is your raw score.  Make sure it is correct.
	\item If you have a dispute with the grading, write down in words, on a separate sheet of paper, exactly what your reasoning in solving the problem was.  It is not enough to say ``I deserve partial credit", or ``The grading should have been worth $x$ points."  Remember, you are trying to convince me you understand the material.  What you write on that separate sheet of paper should reflect that.
	\item This must be done by the end of drill.  Return the exam, along with your list of appeals, to your drill instructor.
	\end{itemize}
\end{itemize}
\end{frame}

% % %
\begin{frame}
\subsection[3.2 Rules of Differentiation]{$\oint$ 3.2 Rules of Differentiation}
\frametitle{$\oint$ 3.2 Rules of Differentiation}
\small
Recall the definition of the derivative:
\vspace{-0.25pc}
\[f^{\prime}(x)=\lim_{h \to 0} \frac{f(x+h)-f(x)}{h}\]

\vspace{-0.25pc}
(as a function of $x$, i.e., a formula).

\vspace{1pc}
And, for any particular point $a$, we have 
\vspace{-0.25pc}
\[f^{\prime}(a)=\lim_{x \to a} \frac{f(x)-f(a)}{x-a}.\]
\end{frame}

% % %
\begin{frame}
\frametitle{Constant Functions}
\small
The constant function $f(x)=c$ is a horizontal line with a slope of 0 at every point.  This is consistent with the definition of the derivative:
\begin{align*}
f^{\prime}(x)&=\lim_{h \to 0} \frac{f(x+h)-f(x)}{h} \\
 &=\lim_{h \to 0} \frac{c-c}{h} \\[0.5pc]
 &=\lim_{h \to 0} 0 = 0.
\end{align*}

Therefore, for constant functions, $f^{\prime}(x)=0.$
\end{frame}

% % %
\begin{frame}
\frametitle{}
{\bf Fact:} For any positive integer $n$, we can factor
\[x^n-a^n=(x-a)(x^{n-1}+x^{n-2}a+\cdots+xa^{n-2}+a^{n-1}).\]
For example, when $n=2$, we get
\[x^2-a^2=(x-a)(x+a),\]
which is the difference of squares formula.
\end{frame}

% % %
\begin{frame}
\frametitle{Power Rule}
\small
Suppose $f(x)=x^n$ where $n$ is a positive integer.  Then at a point $a$,
\begin{align*}
f^{\prime}(a) &= \lim_{x \to a} \frac{f(x)-f(a)}{x-a} = \lim_{x \to a} \frac{x^n - a^n}{x-a} \\[0.25pc]
&= \lim_{x \to a} \frac{(x-a)(x^{n-1}+x^{n-2}a + \dots + xa^{n-2}+a^{n-1})}{x-a} \\[0.25pc]
&= (a^{n-1}+a^{n-2}\cdot a + \dots + a \cdot a^{n-2} + a^{n-1}) = n a^{n-1}. 
\end{align*}

Using the formula for the derivative as a function of $x$, one can show $\dfrac{d}{dx} (x^n)= nx^{n-1}.$
\end{frame}

% % %
\begin{frame}
\frametitle{Constant Multiple Rule}
\small
Consider a function of the form $cf(x)$, where $c$ is a constant.

Just like with limits, we can factor out the constant: 
\begin{align*}
\frac{d}{dx}[cf(x)] &= \lim_{h \to 0} \frac{cf(x+h)-cf(x)}{h} \\
&= \lim_{h \to 0} \frac{c[f(x+h)-f(x)]}{h} = c\lim_{h \to 0} \frac{f(x+h)-f(x)}{h} \\[0.25pc]
&= cf^{\prime}(x)
\end{align*}

Therefore, $\dfrac{d}{dx}[cf(x)]=cf^{\prime}(x).$
\end{frame}

% % %
\begin{frame}
\frametitle{Sum Rule}
\small
Sums of functions also behave under the same limit laws when we differentiate:
\begin{align*}
\frac{d}{dx}[f(x)+g(x)] &= \lim_{h \to 0} \frac{[f(x+h)+g(x+h)]-[f(x)+g(x)]}{h} \\
&= \lim_{h \to 0} \left[\frac{[f(x+h)-f(x)]}{h}+\frac{[g(x+h)-g(x)]}{h}\right] \\[0.25pc]
&= \lim_{h \to 0} \frac{f(x+h)-f(x)}{h}+ \lim_{h \to 0}\frac{g(x+h)-g(x)}{h} \\[0.25pc]
&= f^{\prime}(x)+g^{\prime}(x)
\end{align*}
\end{frame}

% % %
%\begin{frame}
%So if $f$ and $g$ are differentiable at $x$,
%$$\frac{d}{dx}[f(x)+g(x)]=f^{\prime}(x)+g^{\prime}(x).$$
%\bigskip
%The Sum Rule can be generalized for more than two functions to include $n$ functions.
%\bigskip
%Note:  Using the Sum Rule and the Constant Multiple Rule produces the Difference Rule:
%$$\frac{d}{dx}[f(x)-g(x)]=f^{\prime}(x)-g^{\prime}(x).$$
%\end{frame}

% % %
\begin{frame}
\frametitle{}
\begin{ex}Using the differentiation rules we have discussed, calculate the derivatives of the following functions.  Note which rule(s) you are using.
\begin{itemize}
\item[1.] $y=x^5$
\item[2.] $y=4x^3-2x^2$
\item[3.] $y=-1500$
\item[4.] $y=3x^3-2x+4$
\end{itemize}
\end{ex}
\end{frame}

% % %
\begin{frame}
\frametitle{Exponential Functions}
\small
Let $f(x)=b^x$, where $b>0$, $b \neq 1$.  To differentiate at $0$, we write

\vspace{-2pc}
\[f^{\prime}(0)=\lim_{x \to 0}\dfrac{f(x)-f(0)}{x-0}=\lim_{x \to 0} \dfrac{b^x-b^0}{x}=
\lim_{x \to 0} \dfrac{b^x-1}{x}.\]

\vspace{1pc}
It is not obvious what this limit should be.  However, consider the cases $b=2$ and $b=3$.  By constructing a table of values, we can see that 
\[\lim_{x \to 0} \frac{2^x-1}{x} \approx 0.693 \quad \text{and}\quad \lim_{x \to 0} \frac{3^x-1}{x} \approx 1.099.\]
\end{frame}

% % %
\begin{frame}
\small
So, $f^{\prime}(0)<1$ when $b=2$ and $f^{\prime}(0)>1$ when $b=3$.  As it turns out, there is a particular number $b$, with $2<b<3$, whose graph has a tangent line with slope 1 at $x=0$.  

In other words, such a number $b$ has the property that 
\[\lim_{x \to 0} \dfrac{b^x-1}{x}=1.\]
\begin{que}What number is it? \end{que}
{\bf Ans:} This number is $e=2.718281828459 \dots$ (known as the Euler number).  The function $f(x)=e^x$ is called the \emph{\alert{natural exponential function}}.

\end{frame}

% % % 
\begin{frame}
\footnotesize
Now, using  $\displaystyle\lim_{x \to 0} \dfrac{e^x-1}{x}=1$, we can find the formula for $\dfrac{d}{dx}(e^x)$:
\begin{align*}
\frac{d}{dx}(e^x) &= \lim_{h \to 0} \frac{e^{x+h}-e^x}{h} \\[0.25pc]
 &= \lim_{h \to 0}\frac{e^x \cdot e^h -e^x}{h} \\[0.25pc]
 &= \lim_{h\to 0}\frac{e^x(e^h-1)}{h} \\[0.25pc]
 &=e^x \left(\lim_{h \to 0} \frac{e^h-1}{h}\right) \\[0.25pc]
  &= e^x \cdot 1 = e^x
\end{align*}
\end{frame}

% % %
\begin{frame}
\frametitle{}
\begin{exe} 
	\begin{itemize}
	\item Find the slope of the line tangent to the curve $f(x)=x^3-4x-4$ at the point $(2,-4)$. 
	\item Where does this curve have a horizontal tangent?
	\end{itemize}
\end{exe}	
\end{frame}

% % %
\begin{frame}
\frametitle{Higher-Order Derivatives}
If we can write the derivative of $f$ as a function of $x$, then we can take its derivative, too.  The derivative of the derivative is called the \emph{\alert{second derivative}} of $f$, and is denoted $f^{\prime\prime}$.  

In general, we can differentiate $f$ as often as needed.  If we do it $n$ times, the $n$th derivative of $f$ is 
\[\alert{f^{(n)}}(x)=\dfrac{\alert{d^n} f}{\alert{dx^n}}=\alert{\dfrac{d}{dx}}[\alert{f^{(n-1)}}(x)].\]
\end{frame}

% % %
\begin{frame}
\frametitle{HW from Section 3.2}
Do problems 3-45 (x3) (pp.\ 142--145 in textbook)

\vspace{1pc}
For these problems, use only the rules we have derived so far.
\end{frame}

% % % % % % % % % % Wed 11 Feb 2015

% % %
\begin{frame}
\frametitle{Wednesday 11 February (Week 5)}
\begin{itemize}
\item Exam curve: How to adjust your score to fit the syllabus points
\item Thurs 12 Feb usual weekly take-home quiz (Quiz 5)
\end{itemize}
\end{frame}

% % %
\begin{frame}
\subsection[3.3 The Product and Quotient Rules]{$\oint$ 3.3 The Product and Quotient Rules}
\frametitle{$\oint$ 3.3 The Product and Quotient Rules}
\footnotesize
Issue: Derivatives of products and quotients do \alert{NOT} behave like they do for limits.  As an example, consider 
\[f(x)=x^2\quad\text{ and }\quad g(x)=x^3.\]
We can try to differentiate their product in two ways:
\begin{itemize}
\item $\begin{aligned}[t]
	\dfrac{d}{dx}[f(x)g(x)] &= \dfrac{d}{dx}\left(x^5 \right) \\[0.25pc]
	 &= 5x^4
	\end{aligned}$
\item $\begin{aligned}[t]
	f^{\prime}(x)g^{\prime}(x) &= (2x)(3x^2) \\
	 &= 6x^3
	 \end{aligned}$
\end{itemize}
\begin{que}Which answer is the correct one? \end{que}
\end{frame}

% % %
\begin{frame}
\frametitle{Product Rule}
\small
If $f$ and $g$ are any two functions that are differentiable at $x$, then
\[\alert{\frac{d}{dx}[f(x) g(x)] = f^{\prime}(x) g(x) + g^{\prime}(x) f(x)}.\]
In the example from the previous slide, we have
\begin{align*}
\dfrac{d}{dx}[x^2\cdot x^3] &= \frac{d}{dx}(x^2)\cdot (x^3)+x^2\cdot\dfrac{d}{dx}(x^3) \\
 &= (2x)\cdot (x^3)+x^2\cdot (3x^2) \\[0.25pc]
 &= 2x^4+3x^4 \\[0.25pc]
 &= 5x^4
\end{align*}
\end{frame}

% % %
\begin{frame}
\frametitle{Derivation of the Product Rule}
\footnotesize
\begin{align*}
&\dfrac{d}{dx}[f(x)g(x)] = \lim_{h \to 0} \frac{f(x+h)g(x+h)-f(x)g(x)}{h} \\[1pc]
 =\lim_{h \to 0} &\left(\frac{f(x+h)g(x+h)+\alert{[-f(x)g(x+h)+f(x)g(x+h)]}-f(x)g(x)}{h}\right) \\[0.5pc] 
 =\lim_{h \to 0} &\left(\frac{f(x+h)g(x+h)\alert{-f(x)g(x+h)}}{h}\right) \\
 &\hspace{5pc} + \left(\lim_{h \to 0}\frac{\alert{f(x)g(x+h)}-f(x)g(x)}{h}\right) \\[0.5pc]
 =\lim_{h \to 0} &\left(g(x+h) \frac{f(x+h)-f(x)}{h}\right) + \left(\lim_{h \to 0} f(x) \frac{g(x+h)-g(x)}{h}\right) \\[0.5pc]
 &=g(x)f^{\prime}(x)+f(x)g^{\prime}(x)
\end{align*}
\end{frame}

% % %
\begin{frame}
\frametitle{Derivation of Quotient Rule}
\small
\begin{que} Let $q(x)=\dfrac{f(x)}{g(x)}$.  What is $\dfrac{d}{dx}q(x)$? \end{que}
We can write $f(x)=q(x) g(x)$ and then use the Product Rule:
\[f^{\prime}(x) = q^{\prime}(x) g(x) + g^{\prime}(x) q(x)\] 
and now solve for $q^{\prime}(x)$: 
\[q^{\prime}(x)=\frac{f^{\prime}(x)-q(x)g^{\prime}(x)}{g(x)}.\]
\end{frame} 

% % %
\begin{frame}
\frametitle{}
\small
Then, to get rid of $q(x)$, plug in $\frac{f(x)}{g(x)}$:
\begin{align*}
q^{\prime}(x) &= \dfrac{f^{\prime}(x)-g^{\prime}(x)\alert{\dfrac{f(x)}{g(x)}}}{g(x)} \\[0.5pc]
 &= \frac{g(x) \left( f^{\prime}(x)-g^{\prime}(x) \alert{\dfrac{f(x)}{g(x)}} \right)}{g(x)\cdot\alert{g(x)}} \\[1pc]
 &=\frac{f^{\prime}(x) g(x)-g^{\prime}(x)f(x)}{g(x)^2}
\end{align*}
\alert{``LO-D-HI minus HI-D-LO  over LO squared"}
\end{frame}

% % % 
%\begin{frame}
%\frametitle{Quotient Rule}
%Just as with the product rule, the derivative of a quotient is not a quotient of derivatives, i.e.
%$$\frac{d}{dx} \left[ \frac{f(x)}{g(x)}\right] \ne \frac{f^{\prime}(x)}{g^{\prime}(x)}.$$
%
%\bigskip
%
%Here is the correct rule, the Quotient Rule:
%$$\frac{d}{dx} \left[ \frac{f(x)}{g(x)}\right] = \frac{f^{\prime}(x) g(x)-g^{\prime}(x) f(x)}{[g(x)]^2}.$$
%\end{frame}

% % %
\begin{frame}
\frametitle{}
\begin{exe} Use the Quotient Rule to find the derivative of 
\[\frac{4x^3+2x-3}{x+1}.\]
\end{exe}
\begin{exe} Find the slope of the tangent line to the curve 
\[f(x)=\dfrac{2x-3}{x+1}\text{ at the point }(4,1).\] 
\end{exe}
\end{frame}

% % %
\begin{frame}
\frametitle{}
The Quotient Rule also allows us to extend the Power Rule to negative numbers:  

If $n$ is any integer, then $\dfrac{d}{dx}\left[ x^n \right] = nx^{n-1}.$

\begin{que} How? \end{que}
\end{frame}

% % % % % % % % % % Fri 13 Feb 2015

% % %
\begin{frame}
\frametitle{Friday 13 February (Week 5)}
\begin{itemize}
\item possible snow day on Monday, so stay caught up and read $\oint 3.5$
\item Wednesday: $\oint 3.6$: The Chain Rule
\end{itemize}
\end{frame}

% % %
\begin{frame}%[t]
\begin{exe} If $f(x)=\dfrac{x(3-x)}{2x^2}$, find $f^{\prime}(x).$ \end{exe}
\end{frame}

% % %
\begin{frame}%[t]
\frametitle{Derivative of $e^{kx}$}
For any real number $k$,
\[\frac{d}{dx} \left( e^{kx} \right) = ke^{kx}.\]
\begin{exe}What is the derivative of $x^2 e^{3x}$? \end{exe}
\end{frame}

% % %
\begin{frame}
\frametitle{Rates of Change}
\small
The derivative provides information about the instantaneous rate of change of the function being differentiated (compare to the limit of the slopes of the secant lines from $\oint 2.1$).

\vspace{1pc}
For example, suppose that the population of a culture can be modeled by the function $p(t)$.  We can find the instantaneous growth rate of the population at any time $t \ge 0$ by computing $p^{\prime}(t)$ as well as the \alert{\emph{steady-state population}} (also called the \emph{carrying capacity} of the population).  The steady-state population equals 
$$\lim_{t \to \infty} p(t).$$
\end{frame}

% % %
\begin{frame}
\frametitle{HW from Section 3.3}
Do problems 6--51 (x3) (pp.\ 152--154 in textbook).
\end{frame}

% % %
\begin{frame}
\subsection[3.4 Derivatives of Trigonometric Functions]{$\oint$ 3.4 Derivatives of Trigonometric Functions}
\frametitle{$\oint$ 3.4 Derivatives of Trigonometric Functions}
Derivative formulas for sine and cosine can be derived using the following limits:
$$\lim_{x \to 0} \frac{\sin x}{x}=1$$
\bigskip
$$\lim_{x \to 0} \frac{\cos x -1}{x}=0$$
\end{frame}

% % %
\begin{frame}%[t]
\begin{exe} Evaluate $\displaystyle\lim_{x \to 0} \frac{\sin 9x}{x}$ and $\displaystyle\lim_{x \to 0} \frac{\sin 9x}{\sin 5x}.$ \end{exe}
\end{frame}

% % %
\begin{frame}

\frametitle{Derivatives of sine and cosine functions}

Using the previous limits and the definition of the derivative, we obtain
\begin{align*}
\frac{d}{dx} (\sin x) &= \cos x \\
\frac{d}{dx} (\cos x) &= -\sin x
\end{align*}

\end{frame}

% % %
\begin{frame}

\frametitle{Trig Identities you should know}
\small
\begin{columns}[T]

\begin{column}{.5\textwidth}

\begin{block}

\begin{itemize}
\item[]$\sin^2 x + \cos^2 x = 1$ \vspace{0.2cm}
\item[]$\tan^2 x + 1 = \sec^2 x$ \vspace{0.2cm}
\item[]$\sin 2x =2\sin x \cos x$ \vspace{0.2cm}
\item[]$\cos 2x = 1-2\sin^2 x$ \vspace{0.2cm}
\item[]$\cos^2 x = \dfrac{1+\cos 2x}{2}$ \vspace{0.1cm}
\item[]$\sin^2 x = \dfrac{1-\cos 2x}{2}$ \vspace{0.2cm}

\end{itemize}

\end{block}

\end{column}

\begin{column}{.5\textwidth}

\begin{block}

\begin{itemize}

\item[] $\tan x = \dfrac{\sin x}{\cos x}$ \vspace{0.2cm}
\item[] $\cot x = \dfrac{\cos x}{\sin x}$ \vspace{0.2cm}
\item[] $\cot x = \dfrac{1}{\tan x}$ \vspace{0.2cm}
\item[] $\sec x = \dfrac{1}{\cos x}$ \vspace{0.1cm}
\item[] $\csc x = \dfrac{1}{\sin x}$ \vspace{0.2cm}

\end{itemize}

\end{block}

\end{column}

\end{columns}

\end{frame}


% % %
\begin{frame}

\frametitle{Derivatives of other Trig functions}
\small
\begin{align*}
\dfrac{d}{dx}(\tan x)&=\dfrac{d}{dx} \left( \dfrac{\sin x}{\cos x}\right) \\
 &= \dfrac{\cos x \cos x - (-\sin x)\sin x }{\cos^2 x} \\
&= \dfrac{\cos^2 x + \sin^2 x}{\cos^2 x} \\
&= \dfrac{1}{\cos^2 x} = \sec^2 x
\end{align*}

So $\dfrac{d}{dx} (\tan x)=\sec^2 x.$

\end{frame}

% % %
\begin{frame}

By using trig identities and the Quotient Rule, we obtain

\begin{align*}
\dfrac{d}{dx} (\csc x) &= \dfrac{d}{dx} \left( \dfrac{1}{\sin x}\right) = -\csc x \cot x \\
\dfrac{d}{dx} (\sec x) &= \dfrac{d}{dx} \left( \dfrac{1}{\cos x}\right) = \sec x \tan x \\
\dfrac{d}{dx} (\cot x) &= \dfrac{d}{dx} \left( \dfrac{1}{\tan x}\right) = -\csc^2 x  
\end{align*}

\end{frame}

% % %
\begin{frame}%[t]
\begin{exe} Compute the derivative of the following functions:

$$f(x)=\frac{\tan x}{1+\tan x} \qquad g(x)=\sin x \cos x$$

\end{exe}
\end{frame}

% % %
\begin{frame}

\frametitle{Higher-order trig derivatives}

There is a cyclic relationship between the higher order derivatives of $\sin x$ and $\cos x$:

\begin{columns}

\begin{column}{.5\textwidth}

\begin{block}

\begin{itemize}

\item[] $f(x)=\sin x$  
\item[] $f^{\prime}(x)=\cos x$ 
\item[] $f^{\prime\prime}(x)=-\sin x $
\item[] $f^{(3)}(x)=-\cos x $
\item[] $f^{(4)}(x)=\sin x $
\end{itemize}

\end{block}

\end{column}

\begin{column}{.5\textwidth}

\begin{block}

\begin{itemize}

\item[] $g(x)=\cos x$
\item[] $g^{\prime}(x)=-\sin x$
\item[] $g^{\prime\prime}(x)=-\cos x$
\item[] $g^{(3)}(x)=\sin x$
\item[] $g^{(4)}(x)=\cos x$

\end{itemize}

\end{block}

\end{column}

\end{columns}

\end{frame}

% % %
\begin{frame}
\frametitle{HW from Section 3.4}
Do problems 7, 13, 17, 21--27, 33, 35, 44--46, 53--55 (pp.\ 161--162 in textbook)
\end{frame}

\begin{comment}
\end{comment}

\end{document}