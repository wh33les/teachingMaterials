\documentclass[14pt]{beamer}
\usetheme{Warsaw}
\usecolortheme{beaver}
\usefonttheme{professionalfonts}

\input{../../preamble}
\usepackage{amscd,amsmath,amssymb,amsthm,graphicx}
\usepackage{paralist}
\usepackage{tabto}
\usepackage[normalem]{ulem}
\usepackage{tikz}
\usepackage{tkz-euclide}
\usetkzobj{all}
\newcommand{\dint}{\displaystyle\int}
\newcommand{\dlim}{\displaystyle\lim}
\newcommand{\dsum}{\displaystyle\sum}

% % % % % % % % % %
\title[Cal I S2015]{MATH 2554 (Calculus I)}
\subtitle{}
\author[Wheeler]{Dr. Ashley K. Wheeler}
\institute{University of Arkansas}
\date{\today}
\logo{}

% % %
\begin{document}
\maketitle

% % %
\begin{frame}
\frametitle{Table of Contents}
\tableofcontents
\end{frame}

% % % % % % % % % % Mon 27 Apr 2015

% % %
\begin{frame}
\section[Week 15]{Week 15: 27-30 April}
\frametitle{Monday 27 April (Week 15)}
\footnotesize
\begin{itemize}
\item Computer HW this week: $\oint 5.5$
\item Quiz \#15 tomorrow (Tues) in drill -- take-home due Thurs
\item FINAL! Monday 4 May 6-8p OZAR 026.  Study the Coordinator's Review Questions.
\item dead day review SCEN 101 3-5p bring the Coordinator's Review Questions.
\item Thurs 30 April -- don't skip drill
\item your top 10 quizzes go into your final grade
\item BONUS REVIEW see website and MLP -- worth up to 2\% of final grade
\end{itemize}
\end{frame}

% % %
\subsection[$\oint 5.5$ Substitution Rule]{$\oint 5.5$ Substitution Rule}

% % %
\begin{frame}{$\oint 5.5$ Substitution Rule}
\footnotesize
{\bf Idea:}  Suppose we have $F(g(x))$, where $F$ is an antiderivative of $f$.  Then
\begin{align*}
\frac{d}{dx} \bigg[F(g(x)) \bigg] &= F^{\prime}(g(x)) \cdot g^{\prime}(x) = f(g(x)) \cdot g^{\prime}(x) \\
\text{ and }\dint f(g(x)) \cdot g^{\prime}(x)\ dx &= F(g(x))+C
\end{align*}

\vspace{1pc}
If we let $u=g(x)$, then $du=g^{\prime}(x)\ dx$.  The integral becomes
\[\dint f(g(x)) \cdot g^{\prime}(x)\ dx = \dint f(u)\ du.\]
\end{frame}

% % %
\begin{frame}{\small Substitution Rule for Indefinite Integrals}
\small
Let $u=g(x)$, where $g^{\prime}$ is continuous on an interval, and let $f$ be continuous on the corresponding range of $g$.  On that interval,
\[\dint f(g(x)) g^{\prime}(x)\ dx = \dint f(u)\ du.\]

\vspace{1pc}
\alert{$u$-Substitution is the Chain Rule, backwards.}
\end{frame}

% % %
\begin{frame}{}
\small
\begin{ex} Evaluate $\dint 8x \cos(4x^2 + 3)\ dx.$ \end{ex}

\vspace{1pc}
{\bf Solution:} Look for a function whose derivative also appears.
\begin{align*}
u(x) &=4x^2+3 \\
\text{ and }u'(x) &= \dfrac{du}{dx} = 8x \\[0.5pc]
\implies du &= 8x\ dx.
\end{align*}
\end{frame}

% % %
\begin{frame}{}
\footnotesize
Now rewrite the integral and evaluate.  Replace $u$ at the end with its expression in terms of $x$. 
\begin{alignat*}{2}
\dint 8x \cos(4x^2 + 3)\ dx &= \dint \cos(\underbrace{4x^2 + 3}_{u})\underbrace{8x\ dx}_{du} \\
&= \dint \cos u\ du \\[0.25pc]
&= \sin u + C \\[0.5pc]
&= \sin(4x^2 + 3) + C
\end{alignat*}

We can even check our answer.  By the Chain Rule,
\[\dfrac{d}{dx}\left(\sin{(4x^2+3)}+C\right)=8x\cos{(4x^2+3)}.\]
\end{frame}

% % %
\begin{frame}{\small Procedure for Substitution Rule (Change of Variables)}
\small
\begin{itemize}
\item[1.] Given an indefinite integral involving a composite function $f(g(x))$, identify an inner function $u=g(x)$ such that a constant multiple of $g^{\prime}(x)$ appears in the integrand.
\item[2.] Substitute $u=g(x)$ and $du=g^{\prime}(x)\ dx$ in the integral.
\item[3.] Evaluate the new indefinite integral with respect to $u$.
\item[4.] Write the result in terms of $x$ using $u=g(x)$.
\end{itemize}

\vspace{1pc}
{\it Warning:  Not all integrals yield to the Substitution Rule.}
\end{frame}

% % % % % % % % % % Wed 29 Apr 2015

% % %
\begin{frame}
\frametitle{Wednesday 29 April (Week 15)}
\small
\begin{itemize}
\item Computer HW this week: $\oint 5.5$
\item Quiz \#15 due Thurs
\item FINAL! Monday 4 May 6-8p OZAR 026. 
	\begin{itemize}\footnotesize
	\item 2 hours, 20 questions.
	\item Study the Coordinator's Review Questions.
	\item Study the slides and your class notes BEFORE visiting outside resources.
	\item CEA: Email me if you wish to take the exam in SCEN 407.   Exam starts at 4p.  Reduced distractions in SCEN are not guaranteed.   
	\end{itemize}
\end{itemize}
\end{frame}

% % %
\begin{frame}
\small
\begin{itemize}	
\item Dead Day review SCEN 101 3-5p bring your answers to the Coordinator's Review Questions for feedback
\item Thurs 30 April -- don't skip drill
\item your top 10 quizzes go into your final grade
\item BONUS REVIEW see website -- worth up to 2\% of final grade.  Bring to the Final Exam to turn in.
\end{itemize}
\end{frame}

% % %
\begin{frame}%[t]{Exercise}
\small
\begin{exe} Evaluate the following integrals.  Check your work by differentiating each of your answers.
\begin{itemize}
\item $\dint \sin^{10} x \cos x \ dx$
\item $-\dint \dfrac{\csc x \cot x}{1+\csc x}\ dx$
\item $\dint \dfrac{1}{(10x-3)^2}\ dx$
\item $\dint (3x^2 + 8x + 5)^8 (3x+4)\ dx$
\end{itemize}
\end{exe}
\end{frame}

% % %
\begin{frame}{\small Variations on Substitution Rule}
\footnotesize
There are times when the $u$-substitution is not obvious or that more work must be done.
\begin{ex} Evaluate $\dint \dfrac{x^2}{(x+1)^4}\ dx.$ \end{ex}

{\bf Solution:}  Let $u=x+1$.  Then $\alert{x=u-1}$ and $du=dx$.  Hence,
\begin{alignat*}{2}
\dint \dfrac{\alert{x}^2}{(x+1)^4}\ dx &= \dint \dfrac{(\alert{u-1})^2}{u^4}\ du \\
&= \dint \dfrac{u^2-2u+1}{u^4}\ du 
\end{alignat*}
\end{frame}

% % %
\begin{frame}
\footnotesize
\begin{alignat*}{2}
&= \dint \left(u^{-2}-2u^{-3}+u^{-4} \right) \ du \\
&= \dfrac{-1}{u} + \dfrac{1}{u^2} + \dfrac{-1}{3u^3} + C \\
&= \dfrac{-1}{x+1} + \dfrac{1}{(x+1)^2} - \dfrac{1}{3(x+1)^3} + C
\end{alignat*}

\begin{exe}Check it. \end{exe}

This type of strategy works, usually, on problems where you can write $u$ as a linear function of $x$.
\end{frame}

% % %
\begin{frame}{\small Substitution Rule for Definite Integrals}
\footnotesize
We can use the Substitution Rule for Definite Integrals in two different ways:
\begin{itemize}
\item[1.] Use the Substitution Rule to find an antiderivative $F$, and then use the Fundamental Theorem of Calculus to evaluate $F(b)-F(a)$.
\item[2.] Alternatively, once you have changed variables from $x$ to $u$, you may also change the limits of integration and complete the integration with respect to $u$.  Specifically, if $u=g(x)$, the lower limit $x=a$ is replaced by $u=g(a)$ and the upper limit $x=b$ is replaced by $u=g(b)$.
\end{itemize}

\vspace{1pc}
\alert{The second option is typically more efficient and should be used whenever possible.}
\end{frame}

% % %
\begin{frame}{}
\footnotesize
\begin{ex} Evaluate $\dint_0^4 \dfrac{x}{\sqrt{9+x^2}}\ dx.$ \end{ex}

{\bf Solution:} Let $u=9+x^2$.  Then $du=2x\ dx$.  Because we have changed the variable of integration from $x$ to $u$, the limits of integration must also be expressed in terms of $u$.  Recall, $u$ is a function of $x$ (the $g(x)$ in the Chain Rule). For this example,
\begin{alignat*}{2}
x=0\ &\implies\ u(0) = 9+0^2 = 9 \\
x=4\ &\implies\ u(4) = 9+4^2 = 25 
\end{alignat*}
\end{frame}

% % %
\begin{frame}
\small
We had $u=9+x^2$ and $du=2x\ dx \implies \frac{1}{2}du=x\ dx$. So:

\begin{align*}
\dint_0^4 \dfrac{x}{\sqrt{9+x^2}}\ dx &= \frac{1}{2}\dint_{9}^{25} \dfrac{du}{\sqrt{u}} \\[0.5pc]
 &= \dfrac{1}{2} \left. \left(\frac{u^{\frac{1}{2}}}{\frac{1}{2}} \right) \right\vert_{9}^{25} \\[0.5pc]
 &= \sqrt{25}-\sqrt{9} \\[0.5pc]
 &= 5-3=2.
\end{align*}
\end{frame}

% % %
\begin{frame}%[t]
\small
\begin{exe} Evaluate $\dint_0^2 \dfrac{2x}{(x^2+1)^2}\ dx.$ \end{exe}
\end{frame}

% % %
\begin{frame}{HW from Section 5.5}
Do problems 9--39 odd, 53--63 odd (pp.\ 363--364 in textbook)
\end{frame}

% % %
\subsection[Advice for the FINAL]{Advice for the FINAL}

% % %
\begin{frame}{Advice for the FINAL}
\footnotesize
\begin{itemize}
\item Review your notes and the slides first, particularly problems we did in class, then review Quizzes, before visiting outside resources.
\item Review the Midterm for an idea of questions the coordinator likes to ask and how they are graded.
\item $+C$s, $dx$s, $\lim$, units, etc. should be included in your answers {\it or else}.  Don't try to round answers unless it is for a story problem, in which case, you should say ``approximately".
\item ``Definition of Derivative" = the definition with limits
\end{itemize}
\end{frame}

% % %
\begin{frame}
\footnotesize
\begin{itemize}
\item Practice limits and l'H\^{o}pital's Rule so you know which is the quickest technique.
\item ``Mean Value Theorem for Derivatives" = MVT from $\oint$4.6.
\item $\arctan=\tan^{-1}$, etc.
\item Use the Continuity Checklist for questions about continuity.
\item Use limits for questions about vertical asyptotes and end behavior.
\item Know the difference between 1st and 2nd Derivative Tests.
\end{itemize}
\end{frame}

% % %
\begin{frame}{\small Easter Egg-xercises}
\small
\begin{exe} \begin{itemize}
\item Find the 101st derivative of $y=\cos{7x}$ at $x=0$.
\item For what values of the constants $a$ and $b$ is $(-1,2)$ a point of inflection on the curve $y=ax^3+bx^2-8x+2$?
\end{itemize}\end{exe}
\end{frame}

\begin{comment}
\end{comment}

\end{document}