\documentclass[12pt,letterpaper]{article}
\usepackage{fullpage}
\usepackage{multicol}
\usepackage{amsmath}
\usepackage{amsfonts}
\usepackage{amssymb}
%\usepackage{pstricks, pst-node, pst-plot}

\newcommand{\ds}{\displaystyle}
\newcommand{\bv}{\mathbf}
\newcommand{\lv}{\langle}
\newcommand{\rv}{\rangle}

\setlength{\textheight}{55pc}
\setlength{\textwidth}{38pc}

\begin{document}
\flushleft
\begin{multicols}{2}


\textbf{Math 2554 Quiz 10: $\oint 4.4$ \\
due: Tues 31 Mar 2015}

\textbf{Name:  }\underline{\hspace{45ex}}

\vspace{.5in}

\end{multicols}

\pagestyle{empty}

\flushleft
This quiz is due at the end of drill today.  You may use your brain, notes, book, other humans and any pet of your choice. Your solutions must be legible, in order, stapled, de-fringed, and with your name on the top right corner of each page. If you fail to meet any of these requirements you will receive a zero. 

\vspace{1pc}
Remember, {\bf answer the question} and include correct units.  Your answer should include showing it is a maximum or minimum.  

\begin{enumerate}
% % %
\item {\bf (1 point)} Suppose an airline policy states that all baggage must be box-shaped with a sum of length, width, and height not exceeding $108$ in.  What are the dimensions and volume of a square-based box with the greatest volume under these conditions?  

% % %
\item {\bf (1 pt each)} A simple model for travel costs involves the cost of gasoline and the cost of a driver.  Specifically, assume that gasoline costs $p$ dollars per gallon, the vehicle gets $g$ miles per gallon, and the driver charges $w$ dollars per hour.  Let $v$ equal the speed of the vehicle.
	\begin{enumerate}
	\item A plausible function to describe how gas mileage varies with speed is 
	\[g(v)=\frac{v(85-v)}{60}\quad\text{ miles per gallon.}\]
	Evaluate $g(0)$, $g(40)$, and $g(60)$ and explain why these values are reasonable.
	\item At what speed does te gas mileage function have its maximum?
	\item Explain why the cost of a trip of length $L$ miles is 
	\[C(v)=\frac{Lp}{g(v)}+\frac{Lw}{v}\quad\text{ dollars}\]
	(where $g(v)$ is from part (a)).
	\item Suppose $L=400$ miles, $p=4$ dollars per gallon, and $w=20$ dollars per hour.  At what (constant) speed should the vehicle be driven to minimize the cost of the trip?
	\item Should the optimal speed (your answer from (d)) be increased or decreased if $L$ is increased from $400$ to $500$ miles?  Explain.
	\item Should the optimal speed (your answer from (d)) be increased or decreased if $p$ is increased from $4$ dollars per gallon to $4.20$ dollars per gallon?
	\item Should the optimal speed (your answer from (d)) be increased or decreased if $w$ is decreased from $20$ dollars per hour to $15$ dollar per hour?
	\end{enumerate}

% % %	
\item {\bf (1 point)} Find the radius and height of a cylindrical soda can with a volume of $354$ cm$^3$ that minimizes the surface area.
		%\item {\bf (3 points)} Suppose that a blood test for a disease must be given to a population of $N$ people, where $N$ is large.  At most $N$ individual blood tests must be done.  The following strategy reduces the number of tests.  Suppose $100$ people are selected from the population and their blood samples are pooled.  One test determines whether any of the $100$ people are tested individually, making $101$ tests necessary.  However, if the pooled sample tests negative, then $100$ people have been tested with one test.  This procedure is then repeated.  Probability theory shows that if the group size is $x$ (for example, $x=100$, as described here), then te average number of blood tests required to test $N$ people is 
		%\[N(1-q^x+\frac{1}{x}),\] 
		%where $q$ is the probability that any one person tests negative.  What group size $x$ minimizes the average number of tests in the case that $N=10,000$ and $q=0.95$?  Assume that $x$ is a nonnegative real number. 
\item {\bf (1 point)} A rectangle is constructed with one side on the positive $x$-asix, one side on the positive $y$-axis, and one vertex on the line $y=10-2x$.  What dimensions maximize the area of the rectangle?  What is the maximum area?
		


\end{enumerate}

\end{document}


