\documentclass[11pt,letterpaper]{article}
\input{../preamble}
\usepackage{fullpage}
\usepackage{multicol}
\everymath{\displaystyle}

\begin{document}
\flushleft
\begin{multicols}{2}

\begin{large}\textbf{Math 2554 Exam 2: Sections 3.9-4.5 \\
Fri 7 Nov 2014}\end{large}

\hfill\textbf{Name:  }\underline{\hspace{40ex}} %KEY\hspace{17ex}} 
\\
\vspace{.5in}

\end{multicols}

\pagestyle{empty}

\flushleft

\begin{center}\Large Calculus I 

Exam 2 \end{center}

\vspace{2pc}
Please provide the following data:

\vspace{2pc}
Drill Instructor: \underline{\hspace{40ex}}

\vspace{2pc}
Drill Time: \underline{\hspace{40ex}}

\vspace{2pc}
Student ID or clicker \#: \underline{\hspace{40ex}}

\vspace{3pc}
{\bf Exam Instructions:} Sit in every other chair.  You have 50 minutes to complete this exam.  One $3\times 5$ inch notecard, 2-sided, is allowed.  No graphing calculators.  No programmable calculators.  No electronic devices except for the approved calculators (so no phones, iDevices, computers, etc).  If you finish early then you may leave, UNLESS there are less than 5 minutes of class left.  To prevent disruption, if you finish with less than 5 minutes of class remaining then please stay seated and quiet.

\vspace{5pc}
Your signature below indicates that you have read this page and agree to follow the Academic Honesty Policies of the University of Arkansas.  

\vspace{3pc}
Signature: (1 pt) \underline{\hspace{80ex}}

\vfill
\begin{flushright}\Large Good luck!\end{flushright}

\begin{enumerate}
\newpage
\item Write down the following derivatives:
\begin{enumerate}

\vspace{1pc}
\item $\frac{d}{dx}\arcsin(x)$

\vspace{6pc}
\item $\frac{d}{dx}\arccos(x)$

\vspace{6pc}
\item $\frac{d}{dx}\arctan(x)$

\vspace{6pc}
\item $\frac{d}{dx}\arccsc(x)$

\vspace{6pc}
\item $\frac{d}{dx}\arcsec(x)$

\vspace{6pc}
\item $\frac{d}{dx}\arccot(x)$

\vspace{5pc}
\end{enumerate}

\newpage
\item Let $f(x)=\frac{x^2}{x-2}$.  Go through the following Graphing Guidelines to produce a well-labelled graph of $f$:  
\begin{enumerate}
\item Identify the domain or interval of interest.

\vspace{3pc}
\item Is $f$ even, odd, or neither?

\vspace{5pc}
\item Find the first and second derivatives.

\vspace{20pc}
\item Find critical points and possible inflection points.

\newpage
\item Find intervals on which the function is increasing/decreasing and concave up/down.

\vspace{20pc}
\item Identify extreme values and inflection points by using 1st and 2nd derivative tests.

\vspace{5pc}
\item Locate vertical/horizontal asymptotes and determine end behavior.

\newpage
\vspace{5pc}
\item Find the intercepts.

\vspace{10pc}
\item Use the information from (a)-(h) to draw a well-labelled graph of $f$.

\end{enumerate}

% % %
\newpage
\item Suppose $f$ is differentiable on an interval $I$ containing the point $a$.  The {\bf linear approximation} to $f$ at $a$ is the linear function
\vspace{2pc}
\[L(x)=\hspace{15pc}\text{for \underline{\hspace{20ex}}}\]

\vspace{5pc}
\item Let $f(x)=\frac{x}{x+1}$ and $a=1$.

\begin{enumerate}
\item Write the equation of the line that represents the linear approximation to $f(x)$ at the given point $a$.

\vspace{10pc}
\item Use the linear approximation to estimate the value $f(1.1)$.

\vspace{10pc}
\item Compute the percent error in your approximation.

\end{enumerate}

\newpage
\item A rectangular bathtub that is 3 ft wide and 6 ft long is being filled with water.

\begin{enumerate}
\item How fast is the water level rising if water is filling the tub at a rate of 0.7$\text{ ft}^3$/min?

\vspace{20pc}
\item At what rate is water pouring into the tub if the water level rises at a rate of 0.8 ft/min?
\end{enumerate}

\newpage
\item Suppose you are standing on the shore of a circular pond with radius 1 mile and you want to get to a point on the shore directly opposite your position (on the other end of a diameter).  You plan to swim at 2 miles per hour from your current position to another point $P$ on the shore and then walk at 3 miles per hour along the shore to the terminal point.  How should you choose $P$ to minimize the total time for the trip?

{\bf Fact: For a circle of radius $r$ and a chord on the circle with central angle $\theta$, the length of the chord is given by $2r\sin\frac{\pi}{2}$.  Given an arc with central angle $\phi$, the arc length is $r\phi$. }

\end{enumerate}

\end{document}


