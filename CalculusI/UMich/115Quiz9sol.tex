\documentclass[11pt,letterpaper]{article}
\usepackage{fullpage}
\usepackage{multicol}
\usepackage{amsmath}
\usepackage{amsfonts}
\usepackage{amssymb}
\usepackage{graphicx}
%\usepackage{pstricks, pst-node, pst-plot}

\newcommand{\ds}{\displaystyle}
\newcommand{\bv}{\mathbf}
\newcommand{\lv}{\langle}
\newcommand{\rv}{\rangle}

\begin{document}
\flushleft
\begin{multicols}{2}


\begin{large}\textbf{Math 115 Quiz 10: $\oint $ 5.1-3 Summing Rectangles \\
Wed 8 December 2010}\end{large}

\textbf{Name:  }\underline{\hspace{35ex}}

\vspace{.5in}

\end{multicols}

\pagestyle{empty}

\flushleft

You have 30 minutes to complete this quiz.  Make your variables clear and
consistent (so if you want to say, for example, $\frac{dy}{dx}$, you should also
mention $y=f(x)$, or ``$y$ is a function of $x$'').  Calculators are OK.  

\emph{Part 2.(b) is worth 2 points.  All other lines are worth 1 point apiece.}

\begin{enumerate}
\item \textbf{Definitions/Concepts.} \emph{-- none this week --}

\vspace{1pc}
\item \textbf{Questions/Problems.}  

\begin{enumerate}
\item Suppose that a company (called All
Things Food) has hired you as a consultant. You are to
help them
save their failing product, ``Big J's Bar-B-Q Ice
Cream.'' You
have discovered that their cost and revenue functions
(in dollars)
are:
$$C(q)= 100 + 2q \qquad{\rm and}\qquad
R(q)= 15 q^{0.75},$$ where $q$ is the
number of ice cream containers produced. 

\vspace{0.3in}
\noindent {\bf a)} What is the product's fixed cost?
\[\$100\]

\vspace{3pc}
\noindent {\bf b)} Last year, All Things Food produced
$2400$ containers of Big J's Bar-B-Q Ice Cream. What
was their
profit?

\begin{eqnarray*}
 \pi (2400) &=& -(100+2(2400))+15(2400)^{.75} \\
&=& \$243.39
\end{eqnarray*}

\vspace{0.5pc}
\noindent {\bf c)} Find formulas for marginal cost and
marginal revenue, and evaluate at
$q=2400$.

\vspace{.4in}
$MC(q) = 2$

\vspace{0.4in}
$MC(2400) = \$2$/container of ice cream

\vspace{0.4in}
$MR(q) = (.75)15q^{-.25}=11.25q^{-.25}$

\vspace{0.4in}
$MR(2400) = 11.25(2400)^{-.25}=\$ 1.61$/container of ice cream

\vspace{0.4in}
\noindent {\bf d)} Big J wants to increase production
to
do better this year. Based on the marginal revenue and
marginal
cost {\it at this point} ($q=2400$), explain whether
Big J's
strategy is sound. 

\vspace{0.5pc}  At $q=2400$ the marginal cost is greater than the marginal revenue.  This means further increasing production would reduce profit.  Therefore Big J's strategy is not sound. 

\vspace{3pc}
\noindent {\bf e)} What production level will maximize
the profit available to the company?

To maximize profit, set
\[\pi '(q)=11.25q^{-.25}-2=0.\]
The only critical point is at $q\approx 1001$.  Notice $\pi '(q)$ is monotonically decreasing on the domain $q\geq 0$.  In particular, $\pi '$ must go from positive to negative at the critical point.  There for the critical point is a maximum.  The production level to maximize profit is $q=1001$ containers of ice cream.

\vspace{2pc}
\item The metal frame of a rectangular box has a square base.  The horizontal rods in the base are made out of one metal and the vertical rods are made out of a different metal.  If the horizontal rods expand at a rate of 0.001 cm/hr and the vertical rods expand at a rate of 0.002 cm/hr, at what rate is the volume of the box expanding when the base has an area of 9 cm$^2$ and the volume is 180 cm$^3$?

\vspace{0.5pc}
Let $x$ denote the length of a horizontal rod in cm, $y$ denote the length of a vertical rod in cm, and $t$ denote time in hrs.  Then the given data is
\[\frac{dx}{dt}=0.001\text{ cm/hr, }\frac{dy}{dt}=0.002\text{ cm/hr.}\]
Let $V=x^2y$ denote the volume.  The volume of the box is expanding at the rate
\[\frac{dV}{dt}=2x\frac{dx}{dt}y+x^2\frac{dy}{dt}.\]
When the base has area 9 cm$^2$, $x=3$.  Then since $V=180$, this implies $y=20$.  Substituting all the numerical values gives
\begin{eqnarray*}
 \frac{dV}{dt} &=& 2(3)(0.001)(20)+(3)^2(0.002) \\
&\approx & 0.138 \text{ cm}^3\text{/hr.}
\end{eqnarray*}


\end{enumerate}

\vfill
\item \textbf{Computations/Algebra.}  \emph{-- none this week --}

\end{enumerate}
%----------------------------------------------------------------------------------------

%\vspace{1pc}
%\noindent \textbf{ChAlLeNgE PrObLeM:}  
\end{document}


