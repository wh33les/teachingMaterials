%%This is a standard LaTeX2e article document template. personal version 12/5/200%%
\documentclass[11pt,twoside]{article}
%%%%%%%%%%%%%%%%%%%%%%%%%%%%%%%packages%%%%%%%%%%%%%%%%%%%%%%%%%%%%%%%%%%%%%%%%%%%%%%%%%%%%%%%%%%
\pagestyle{empty}

\usepackage{latexsym}
\usepackage{amssymb}
\usepackage{amsfonts}
\usepackage{amstext}
\usepackage{multicol}
%%%%%%%%%%%%%%%%%%%%%%%%%%%%%%%formatting%%%%%%%%%%%%%%%%%%%%%%%%%%%%%%%%%%%%%%%%%%%%%%%%%%%%%%%
\setlength{\topmargin}{-.1in}        %%%  This sets all the spacing stuff to use the page more
\setlength{\oddsidemargin}{0in}    %%%  efficiently than the normal "article" setup would.
\setlength{\evensidemargin}{0in}   %%%  It's OK to play with these some.
\setlength{\textheight}{8.5in}     %%%
\setlength{\textwidth}{6.25in}     %%%
\setlength{\headsep}{0in}          %%%
\setlength{\headheight}{0in}       %%%
%\setlength{\footskip}{0in}         %%%

%%%%%%%%%%%%%%%%%%%%%%%%%%%%%%%%%%%%%%%%%%%%%%%%%%%%%%%%%%%%%%%%%%%%%%%%%%%%%%%%%%%%%%%%%%%%%%%

\begin{document}

\begin{center}
{\bf \Large \underline{Math 115-036 -- Fall 2010} \\
last updated: \today}
\end{center}

\vspace{.05in}

\begin{description}
\item[\bf Instructor:] Ashley Wheeler

\item[\bf Email:]  wheeles@umich.edu

\item[\bf Office:] 4848 East Hall

\item[\bf Website:] www-personal.umich.edu/$\sim $wheeles
\end{description}

\begin{description}
\item[\bf Office Hours:] Mon 11a-12p (Math Lab), Tue 1-2p, Wed 12-1p

\vspace{.1in}
\item[\bf Text:] {\it Calculus} by Hughes-Hallet, Gleason, et al.,
5th Edition, published by John Wiley and Sons

Bring the book to class everyday.  READ THE BOOK!  I teach to supplement the book, not regurgitate it.  You gain a lot more when you read the material on your own, at your own pace.

\vspace{.1in}

\item[\bf Calculator:] TI-84 or equivalent. If you have another
model, you will be responsible for knowing how to use it.  Bring your calculator to class each day and to the Uniform Exams.

\vspace{.1in}

\item[\bf Course Website:] \emph{http://www.math.lsa.umich.edu/courses/115/}\\
Everything is here.  In order to keep the 53 sections of Math 115 uniform, we all refer to the same guidelines.  You can find information about grading and course policies (Student Guide) as well as the webwork, practice problems, assignments, and exam information.  

\vspace{.1in}

\item[\bf Homework:]  Daily homework will be assigned from each section that we cover. These assignments will be worked on the web, and the web homework will count for 5\% of the Exam Component of your final grade.  In order to do well in class, you must keep up with the daily assignments. In addition, you will be given regular team homework assignments, and a large portion of your in-class grade will be based on these group assignments.  The first team homework assignment will be given Mon 13 September.  From then on, these will be due at the beginning of class on Mondays.

\vspace{.1in}
\item[\bf Quizzes:]  There will be short weekly quizzes.  Quizzes are on Mondays, and are designed to check you are up to speed on reading.  No make-up quizzes will be given, but I will drop your lowest quiz score when I compute your final grade.

\vspace{.1in}
\item[\bf Uniform Exams:]\mbox{\ }

\begin{tabular}{lll}
First Exam &  Tue 12 October & 6-7:30p \\
Second Exam & Tue 16 November & 6-7:30p \\
Final Exam &  Fri 17 December & 10:30a-12:30p
\end{tabular}

Dates for the exams are fixed.  Generally, only students with a regularly scheduled
class are accommodated at an alternate time. Anyone with a
regularly scheduled class during these exam times should let me
know as soon as possible. Make plans now to be certain these dates are in your calendar.  \textbf{Note that travel is \emph{not} a sufficient excuse to have an exam scheduled on a different day.}

\vspace{.1in}
\item[\bf Grading Policy:]  All sections of Math 115 use the same grading guidelines to standardize the evaluation process.  A complete explanation of the grading policy is given in the Student Guide on the course web site.  Look under the heading ``Grading System".  

\vspace{.1in}
\item[\bf In-Class Grade:]  The section portion based on my judgement.

\begin{tabular}{rl}
 Team Homework & 70\% \\
 Quizzes & 20\% \\
 Class Participation & 10\%
\end{tabular}

Since attendance is not mandatory, though \emph{highly} encouraged, I will measure your class participation by how many completed in-class exercises you turn in.  Please let me know if you would like a copy of the problems you turn in.

\vspace{.1in}
\item[\bf Math Lab:]  Free tutoring from the Mathematics
Department.

\begin{tabular}{ll}
{\bf Hours:} & Mon through Thu 11a-4p, 7-10p \\
             & Fri 11a-4p \\
             & Sun 7-10p \\
{\bf Location:} & B860 East Hall
\end{tabular}

\vspace{.1in}
\item [\bf Other Important dates:]

\smallskip
\begin{tabular}{ll}
Last day to drop without a W & Mon 27 September \\
Fall Study Break & Mon-Tue 18-19 October \\
LSA Drop Deadline (with a W) & Fri 12 November \\
Thanksgiving Recess & 25-26 November \\
Last Day of Classes & Mon 13 December
\end{tabular}

\vspace{.4in}
\item Any student with a documented disability should contact me as soon as possible so that we can discuss arrangements to fit your needs.
\end{description}

\end{document}
