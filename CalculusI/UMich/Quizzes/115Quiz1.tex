\documentclass[11pt,letterpaper]{article}
\usepackage{fullpage}
\usepackage{multicol}
\usepackage{amsmath}
\usepackage{amsfonts}
\usepackage{amssymb}
%\usepackage{pstricks, pst-node, pst-plot}

\newcommand{\ds}{\displaystyle}
\newcommand{\bv}{\mathbf}
\newcommand{\lv}{\langle}
\newcommand{\rv}{\rangle}

\begin{document}
\flushleft
\begin{multicols}{2}


\begin{large}\textbf{Math 115 Quiz 1: Up thru $\oint $ 1.5 \\
Mon 20 September 2010}\end{large}

\textbf{Name:  }\underline{\hspace{35ex}}

\vspace{.5in}

\end{multicols}

\pagestyle{empty}


\flushleft

You have 15 minutes to complete this quiz.  No calculators allowed.  Eyes on your own paper and good luck!

\begin{enumerate}
\item  \textbf{Definitions/Concepts.} (1 pt each) Write down the definition of
\begin{enumerate} 
\item function 
\vspace{1.5pc}
\item linear function
\vspace{1.5pc}
\end{enumerate}

\item \textbf{Questions/Problems.} (1 pt each) 

Let $x$ be the number of months
that a shotput\footnote{A {\it shotput} is a dense
metal ball
thrown for distance by men and women in athletic
competition.}
thrower has practiced her sport. Let $f(x)$ be the
resulting
distance (in meters) she can throw the shotput. (We
assume that
this distance is a function only of $x$, and ignore
factors like
innate ability.)
For each of these expressions, translate its meaning
into
nonmathematical terms:
\begin{enumerate}
\item $f(3)$ 
\vspace{1.5pc}
\item $f(20) = 12$ 
\vspace{1.5pc}
\item $f^{-1}(16)$ 
\vspace{1.5pc}
\end{enumerate}
\noindent Translate each of these wise sayings by Coach Ironarm
into a
mathematical equation or expression: 
\begin{enumerate}
\item ``\ldots twice as far as I can throw the shot, and I've been doing this for ten years!'' (how far is this distance, in terms of $f$?) 
\vspace{1.5pc}
\item ``A rookie with no practice can usually throw a good 4 meters.'' 
\vspace{1.5pc}
\end{enumerate}

\item \textbf{Computations/Algebra.} (1 pt each) 
\begin{enumerate}
\item If $m(z)=z^2$, simplify $m(z+h)-m(z)$.
\vspace{1.5pc}
\item Convert the angle $\frac{\pi }{6}$ to degrees.
\vspace{1.5pc}
\end{enumerate}

\end{enumerate}

\end{document}


