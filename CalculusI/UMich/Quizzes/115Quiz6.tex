\documentclass[11pt,letterpaper]{article}
\usepackage{fullpage}
\usepackage{multicol}
\usepackage{amsmath}
\usepackage{amsfonts}
\usepackage{amssymb}
%\usepackage{pstricks, pst-node, pst-plot}

\newcommand{\ds}{\displaystyle}
\newcommand{\bv}{\mathbf}
\newcommand{\lv}{\langle}
\newcommand{\rv}{\rangle}

\begin{document}
\flushleft
\begin{multicols}{2}


\begin{large}\textbf{Math 115 Quiz 6: $\oint $ 3.5-7 Implicit Differentiation \\
Mon 1 November 2010}\end{large}

\textbf{Name:  }\underline{\hspace{35ex}}

\vspace{.5in}

\end{multicols}

\pagestyle{empty}

\flushleft

You have 20 minutes to complete this quiz.  Make your variables clear and
consistent (so if you want to say, for example, $\frac{dy}{dx}$, you should also
mention $y=f(x)$, or ``$y$ is a function of $x$'').  Calculators are OK.  

\begin{enumerate}
\item  \textbf{Definitions/Concepts.} \emph{none this week}

\vspace{1pc}
\item \textbf{Questions/Problems.}  \noindent A bungee jumper's height above a river ($h$ in meters) and velocity ($v$ in meters per
second --- positive $v$ is upward motion) are related. (``Bungee Jumping'' is the sport of jumping usually head-first from a tall
bridge while securely fastened by an elastic cord. A bungee jumper will bob up and down for some time after being caught by the
cord.)  The algebraic relationship between $v$ and $h$ turns out to be:
\[5v^2 + h^2 - 102h = 500\]

\begin{enumerate}
\item (1 pt)  The jumper later exclaims: ``Dude, I was like 36 meters above the river and bouncing up at like 24 meters per second! Rock on!'' but his mother suspects he was exaggerating. Demonstrate that his claim is indeed approximately correct. 

\vspace{3pc}
\item (2 pts)  Using implicit differentiation, calculate $\frac{dv}{dh}$ in terms of $v$ and $h$.

\vspace{5pc}
\item (1 pt)  Calculate $\frac{dv}{dh}$ at the moment described by the jumper in part (a).

\end{enumerate}

\vspace{5pc}
\begin{flushright}
 turn over $\rightarrow $
\end{flushright}

\pagebreak
\item \textbf{Computations/Algebra.} (1 pt each)  For each function $g(x)$, find the value of $g'(3)$ using the data given below. You must show your work to receive credit.
\begin{center}
\begin{tabular}{l l l l}
$f(1)=6$ & $f(3)=2$ & $f(6)=5$ & $f(9)=-3$ \\
$f'(1)=-2$ & $f'(3)=4$ & $f'(6)=-1$ & $f'(9)=1$ \\
\end{tabular}
\end{center}
\vskip 3pt
\begin{tabular}{l c c c c c c c c c l}
(a) $g(x) = f(3) + 10f(2x)$ & & & & & & & & & & (d)
$g(x) = \frac{f(x^2)}{x}$ \\
& \\
& \\
& \\
& \\
& \\
& \\
& \\
(b) $g(x) = 2^x f(x)$ & & & & & & & & & & (e) $g(x) =
\ln (f(x))$ \\
& \\
& \\
& \\
& \\
& \\
& \\
& \\
(c) $g(x) = [f(x)]^3$ & & & & & & & & & & (f) $g(x) =
f(\sqrt{2} \sin \frac{\pi}{4}x)$ \\
\end{tabular}

\end{enumerate}

%----------------------------------------------------------------------------------------

\vspace{1pc}
%\noindent \textbf{ChAlLeNgE PrObLeM:}  

\end{document}


