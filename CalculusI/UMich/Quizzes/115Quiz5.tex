\documentclass[11pt,letterpaper]{article}
\usepackage{fullpage}
\usepackage{multicol}
\usepackage{amsmath}
\usepackage{amsfonts}
\usepackage{amssymb}
%\usepackage{pstricks, pst-node, pst-plot}

\newcommand{\ds}{\displaystyle}
\newcommand{\bv}{\mathbf}
\newcommand{\lv}{\langle}
\newcommand{\rv}{\rangle}

\begin{document}
\flushleft
\begin{multicols}{2}


\begin{large}\textbf{Math 115 Quiz 5: $\oint $ 3.1-4 Basic Shortcuts \\
Mon 25 October 2010}\end{large}

\textbf{Name:  }\underline{\hspace{35ex}}

\vspace{.5in}

\end{multicols}

\pagestyle{empty}

\flushleft

You have 30 minutes to complete this quiz.  Make your variables clear and
consistent (so if you want to say, for example, $\frac{dy}{dx}$, you should also
mention $y=f(x)$, or ``$y$ is a function of $x$'').  Calculators are OK.  

\begin{enumerate}
\item  \textbf{Definitions/Concepts.} (1 pt each)  State the following:
\begin{enumerate}
 \item Product Rule:

\vspace{4pc}
 \item Quotient Rule:

\vspace{4pc}
 \item Chain Rule:

\end{enumerate}

\vspace{3pc}
\item \textbf{Questions/Problems.}  The acceleration due to gravity, $g$, at a
distance $r$ from the center of the earth is given by
\[g=\frac{GM}{r^2},\]
where $M$ is the mass of the earth and $G$ is a constant.

\begin{enumerate}
\item (1 pt) Find $\frac{dg}{dr}$.

\vspace{3pc}
\item (2 pts) What is the practical interpretation (in terms of acceleration) of
$\frac{dg}{dr}$?  Why would you expect it to be negative?

\vspace{3pc}
\item (1 pt) You are told that $M=6\cdot M^{24}$ and $G=6.67\cdot 10^{-20}$
where $M$ is in kilograms and $r$ in kilometers.  What is the value of
$\frac{dg}{dr}$ at the surface of the earth ($r=6400$ km)?

\vspace{3pc}
\item (1pt) What does this tell you about whether or not it is reasonable to
assume $g$ is constant near the surface of the earth?
 
\end{enumerate}
 
\vspace{2pc}
\item \textbf{Computations/Algebra.} (1 pt each) Differentiate with respect to
$x$.  You must show work to get credit.
\begin{enumerate}
\item $f(x)=\frac{x^2 +3x+2}{x+1}$

\vspace{4pc}
\item $g(x)=x^k +k^x$

\end{enumerate}

\end{enumerate}

\vspace{4pc}
----------------------------------------------------------------------------------------

\vspace{1pc}
\noindent \textbf{ChAlLeNgE PrObLeM:}  Use the identity 
\[\ln{(a^x)}=x\ln{a}\]
and the chain rule to write an alternate justification of the formula 
\[\frac{d}{dx}a^x=(\ln{a})a^x.\]

\end{document}


