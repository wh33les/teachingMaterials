\documentclass[11pt,letterpaper]{article}
\usepackage{fullpage}
\usepackage{multicol}
\usepackage{amsmath}
\usepackage{amsfonts}
\usepackage{amssymb}
%\usepackage{pstricks, pst-node, pst-plot}

\newcommand{\ds}{\displaystyle}
\newcommand{\bv}{\mathbf}
\newcommand{\lv}{\langle}
\newcommand{\rv}{\rangle}

\begin{document}
\flushleft
\begin{multicols}{2}


\begin{large}\textbf{Math 115 Quiz 3: $\oint $ 2.2-4 (Practice Probs for Exam 1) \\
Mon 4 October 2010}\end{large}

\textbf{Name:  }\underline{\hspace{35ex}}

\vspace{.5in}

\end{multicols}

\pagestyle{empty}


\flushleft

You have 15 minutes to complete this quiz.  No calculators allowed.  

\begin{enumerate}
\item  \textbf{Definitions/Concepts.} (1 pt each) 
You are teaching an introductory calculus course at
Michigan
State, and your
students have just turned in a homework assignment on
interpretations of
the derivative. You must read all their explanations
and decide whether
they make sense. Some excerpts from their solutions are
given below.
Mark each correct or sensible statement with a '+' and
each incorrect or
nonsensical statement with a '-.'
\vskip 12pt
Let $g(v)$ denote the fuel efficiency (in miles per
gallon) of a car
travelling at $v$ miles per hour.
\vskip 5pt
\underline {\hskip 8pt + \hskip 5pt} (a) The statement $g'(55) =
-0.5$ means tha t
when you use one more gallon of gasoline, the time you
can drive at $55$ mph
decreases by about a half hour.
\vskip 5pt
\underline {\hskip 8pt + \hskip 5pt} (b) If $g'(v) < 0$ for $v
\geq 45$, then the
car's fuel
efficiency will be lower at 65 mph than at 45 mph.
\vskip 5pt
\underline {\hskip 10pt -- \hskip 5pt} (c) If $g'(65) = g'(55) =
0$, then the car's fuel
efficiency is constant on the interval $55 \leq v \leq
65$.

\vspace{.5pc}
No information is given about the fuel efficiency between 55 and 65 mph.

\vskip 5pt
\underline {\hskip 10pt -- \hskip 5pt} (d) Since the units of
$g'(v)$ are hours per gallon,
the units of $g''(v)$ are hours per gallon squared.

\vspace{.5pc}
The units are hours per gal per mph.

\vspace{1pc}
\item \textbf{Questions/Problems.} (2 pts each)  Suppose that the line tangent to the graph of $f(x)$ at $x=3$ passes through the points $(1,2)$ and $(5,-4)$.
\begin{enumerate}
 \item Find $f'(3)$.

\vspace{.5pc}
This is just the slope of the line tangent to the graph at $x=3$, which can be computed using the two points given:
\begin{eqnarray*}
   f'(3) &=& \frac{-4-2}{5-1} \\ 
       &=& \frac{-6}{4} \\
       &=& \frac{-3}{2}.
  \end{eqnarray*}

\vspace{1pc}
 \item Find $f(3)$.

\vspace{.5pc}
When $x=3$ the function and the tangent line intersect.  So, to find $f(3)$, use the equation for the tangent line. Using one of the given points and the slope computed above:
\begin{eqnarray*}
 y-y_0 &=& \frac{-3}{2}(x-x_0) \\
 y-5 &=& \frac{-3}{2}(x-1) \\
 y &=& \frac{-3}{2}x+\frac{3}{2}+5 \\
   &=& \frac{-3}{2}x+\frac{7}{2}.
\end{eqnarray*}
Therefore,
\begin{eqnarray*}
 f(3) &=& \frac{-3}{2}(3)+\frac{7}{2} \\
   &=& \frac{-9}{2}+\frac{7}{2} \\
   &=& \frac{-2}{2} \\
   &=& -1.
\end{eqnarray*}

\vspace{1pc}
 \item Estimate the value of $f(2.9)$.

\vspace{.5pc}
The derivative at $x=3$ gives a linear approximation to the function $f(x)$ near $x=3$.  So use the tangent line to approximate $f(2.9)$.
\begin{eqnarray*}
 f(2.9) &\approx & \frac{-3}{2}(2.9)+\frac{7}{2} \\
  &=& \frac{-1.7}{2} \\
  &=& -0.85
\end{eqnarray*}
So $f(2.9)\approx -0.85$.

\end{enumerate}

\vspace{1pc}
\item \textbf{Computations/Algebra.}  \emph{-- none this week --}

\end{enumerate}

\end{document}


