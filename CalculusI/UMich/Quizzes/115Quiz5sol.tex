\documentclass[11pt,letterpaper]{article}
\usepackage{fullpage}
\usepackage{multicol}
\usepackage{amsmath}
\usepackage{amsfonts}
\usepackage{amssymb}
%\usepackage{pstricks, pst-node, pst-plot}

\newcommand{\ds}{\displaystyle}
\newcommand{\bv}{\mathbf}
\newcommand{\lv}{\langle}
\newcommand{\rv}{\rangle}

\begin{document}
\flushleft
\begin{multicols}{2}


\begin{large}\textbf{Math 115 Quiz 5: $\oint $ 3.1-4 Basic Shortcuts \\
Mon 25 October 2010}\end{large}

\textbf{Name:  }\underline{\hspace{35ex}}

\vspace{.5in}

\end{multicols}

\pagestyle{empty}

\flushleft

You have 30 minutes to complete this quiz.  Make your variables clear and
consistent (so if you want to say, for example, $\frac{dy}{dx}$, you should also
mention $y=f(x)$, or ``$y$ is a function of $x$'').  Calculators are OK.  

\begin{enumerate}
\item  \textbf{Definitions/Concepts.} (1 pt each)  State the following:
\begin{enumerate}
 \item Product Rule: 

\vspace{.5pc}
If $u=f(x)$ and $v=g(x)$ are differentiable, then
\[(fg)'=f'g+g'f.\]
The product rule can also be written 
\[\frac{d(uv)}{dx}=\frac{du}{dx}\cdot v+u\cdot \frac{dv}{dx}.\]
In words: 
\linebreak
The derivative of a product is the derivative of the first times the second plus the first times the derivative of the second.

\vspace{.5pc}
 \item Quotient Rule:

\vspace{.5pc}
If $u=f(x)$ and $v=g(x)$ are differentiable, then
\[\left(\frac{f}{g}\right) '=\frac{f'g-fg'}{g^2},\]
or equivalently,
\[\frac{d}{dx}\left(\frac{u}{v}\right)=\frac{\frac{du}{dx}\cdot v-u\cdot \frac{dv}{dx}}{v^2}.\]
In words:
\linebreak
The derivative of a quotient is the derivative of the numerator times the denominator minus the numerator times the derivative of the denominator, all over the denominator squared.

\vspace{.5pc}
 \item Chain Rule:

\vspace{.5pc}
If $f$ and $g$ are differentiable functions, then
\[\frac{d}{dx}f\left(g(x)\right)=f'\left(g(x)\right)\cdot g'(x).\]
In words:
\linebreak
The derivative of a composite function is the product of the derivatives of the outside and inside functions.  The derivative of the outside function must be evaluated at the inside function.

\end{enumerate}

\vspace{1pc}
\item \textbf{Questions/Problems.}  The acceleration due to gravity, $g$, at a
distance $r$ from the center of the earth is given by
\[g=\frac{GM}{r^2},\]
where $M$ is the mass of the earth and $G$ is a constant.

\begin{enumerate}
\item (1 pt) Find $\frac{dg}{dr}$.

\vspace{.5pc}
Use the power rule applied to $r^{-2}$ and multiply by the constant $GM$.
\begin{eqnarray*}
\frac{dg}{dr} &=& -2\cdot r^{-3}\cdot GM \\
&=& \frac{-2GM}{r^3}
\end{eqnarray*}

\vspace{.5pc}
\item (2 pts) What is the practical interpretation (in terms of acceleration) of
$\frac{dg}{dr}$?  Why would you expect it to be negative?

\vspace{.5pc}
The derivative $\frac{dg}{dr}$ is the rate of change of acceleration due to gravity, as the distance, $r$, from the center of the earth increases.  Gravitational pull toward the earth decreases at distances far from the earth, so $g$ decreasing implies $\frac{dg}{dr}<0$.

\vspace{.5pc}
\item (1 pt) You are told that $M=6\cdot 10^{24}$ and $G=6.67\cdot 10^{-20}$
where $M$ is in kilograms and $r$ in kilometers.  What is the value of
$\frac{dg}{dr}$ at the surface of the earth ($r=6400$ km)?

\vspace{.5pc}\textit{The typo in this problem is fixed.}
\linebreak
Replace the constants with the given values and evaluate the derivative at $r=6400$.
\begin{eqnarray*}
 g'(6400) &=& -2\frac{(6.67\cdot 10^{-20})(6\cdot 10^{24})}{6400^3} \\
&\approx & -3.053\cdot 10^{-6} \text{ kg/km}^3
\end{eqnarray*}

\vspace{.5pc}
\item (1pt) What does this tell you about whether or not it is reasonable to
assume $g$ is constant near the surface of the earth?

\vspace{.5pc}
It is totally reasonable.  According to part (c), near the surface of the earth the acceleration due to gravity decreases very slightly if $r$ increases by 1 km. To ease computations, assuming $g$ is constant near the surface of the earth will only give an error on the order of $10^{-6}$ kg/km$^2$. 
\end{enumerate}
 
\vspace{1pc}
\item \textbf{Computations/Algebra.} (1 pt each) Differentiate with respect to
$x$.  You must show work to get credit.
\begin{enumerate}
\item $f(x)=\frac{x^2 +3x+2}{x+1}$

\begin{eqnarray*}
 f'(x) &=& \frac{(x+1)\frac{d}{dx}(x^2+3x+2)-(x^2+3x+2)\frac{d}{dx}(x+1)}{(x+1)^2} \\
&=& \frac{(x+1)(2x+3)-(x^2+3x+2)(1)}{(x+1)^2} \\
&=& \frac{x^2+2x+1}{(x+1)^2} \\
&=& 1
\end{eqnarray*}
Alternate answer:  Notice
\begin{eqnarray*}
\frac{x^2+3x+2}{x+1} &=& \frac{(x+2)(x+1)}{x+1} \\
&=& x+2. 
\end{eqnarray*}
Then 
\begin{eqnarray*}
 f'(x) &=& \frac{d}{dx}(x+2) \\
&=& 1.
\end{eqnarray*}

\vspace{.5pc}
\item $g(x)=x^k +k^x$

\[g'(x)=kx^{k-1}+(\ln{k})k^x\]
\end{enumerate}

\end{enumerate}

\vspace{1pc}
----------------------------------------------------------------------------------------

\vspace{1pc}
\noindent \textbf{ChAlLeNgE PrObLeM:}  Use the identity 
\[\ln{(a^x)}=x\ln{a}\]
and the chain rule to write an alternate justification of the formula 
\[\frac{d}{dx}a^x=(\ln{a})a^x.\]

\vspace{.5pc}\textit{The point of this exercise was to segue into $\oint $3.6.  The answer is part of the text; see p. 147.}
\linebreak
Differentiate both sides of the identity with respect to $x$:
\begin{eqnarray*}
\frac{d}{dx}\left(\ln{(a^x)}\right. &=& \left.x\ln{a}\right) \\
\frac{d}{d(a^x)}\ln{(a^x)}\cdot \frac{d}{dx}a^x &=& \frac{d}{dx}x\ln{a} \\
\frac{d}{dx}a^x &=& \frac{\ln{a}}{\frac{d}{d(a^x)}\ln{(a^x)}}
\end{eqnarray*}
The last step is to verify 
\[\frac{d}{d(a^x)}\ln{(a^x)}=\frac{1}{a^x}.\]
To simplify notation, put $y=a^x$.  We want to find
\[\frac{d}{dy}\ln{y}.\]
Recall the identity $e^{\ln{y}}=y$.  Now differentiate both sides with respect to $y$, using the chain rule on the lefthand side.
\begin{eqnarray*}
\frac{d}{dy}\left(e^{\ln{y}}\right. &=& \left.y\right) \\
\frac{d}{dy}e^{\ln{y}} &=& \frac{d}{dy}y \\
e^{\ln{y}}\cdot \frac{d}{dy}(\ln{y}) &=& 1.
\end{eqnarray*}
Use the identity $e^{\ln{y}}=y$ again and solve for $\frac{d}{dy}\ln{y}$.
\begin{eqnarray*}
y\cdot \frac{d}{dy}\ln{y} &=& 1 \\
\frac{d}{dy}\ln{y} &=& \frac{1}{y}.
\end{eqnarray*}
Since $y=a^x$, indeed,
\[\frac{d}{d(a^x)}\ln{(a^x)}=\frac{1}{a^x}.\]
Therefore 
\begin{eqnarray*}
\frac{d}{dx}a^x &=& \frac{\ln{a}}{\frac{d}{d(a^x)}\ln{(a^x)}} \\
&=& \frac{\ln{a}}{\frac{1}{a^x}} \\
&=& (\ln{a})a^x. 
\end{eqnarray*}

\end{document}


