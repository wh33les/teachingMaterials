\documentclass[11pt,letterpaper]{article}
\usepackage{fullpage}
\usepackage{multicol}
\usepackage{amsmath}
\usepackage{amsfonts}
\usepackage{amssymb}
%\usepackage{pstricks, pst-node, pst-plot}

\newcommand{\ds}{\displaystyle}
\newcommand{\bv}{\mathbf}
\newcommand{\lv}{\langle}
\newcommand{\rv}{\rangle}

\begin{document}
\flushleft
\begin{multicols}{2}


\begin{large}\textbf{Math 115 Quiz 4: $\oint $ 2.5-6 and Barehanded Differentiation \\
Mon 11 October 2010}\end{large}

\textbf{Name:  }\underline{\hspace{35ex}}

\vspace{.5in}

\end{multicols}

\pagestyle{empty}

\flushleft

You have 25 minutes to complete this quiz.  Calculators are OK.  

\begin{enumerate}
\item  \textbf{Definitions/Concepts.} (1 pt) 
Let $g$ be the function defined by
\[ g(x)=\left\{\begin{array}{lcl}
              1 & \text{if} & x\leq 0 \\
              \cos x & \text{if} & 0 < x < \frac{\pi}{2} \\
              0 & \text{if} & x > \frac{\pi}{2}
             \end{array}\right..
 \]
Which of the following statements are true? Check all
that apply.
\begin{enumerate}
 \item $g$ is continuous at $x=0$ \emph{TRUE}
 \item $g$ is continuous at $x=\frac{\pi}{2}$ \emph{FALSE}
 \item $g$ is differentiable at $x=0$ \emph{TRUE} 
 \item $g$ is differentiable at $x=\frac{\pi}{2}$ \emph{FALSE}
\end{enumerate}

\vspace{1pc}
\item \textbf{Questions/Problems.} 
\noindent
A Purple-Headed Uniquely Nocturnal Chartreuse And
Luridly Colored
wombat is sighted moving across the diag. Its position,
measured in
feet from the West Engineering arch, is given as a
function of time
(in minutes past midnight) in the following table.
\smallskip
\begin{center}
\begin{tabular}{c|ccccccc}
$t$ & 0 & 5 & 10 & 15 & 20 & 25 & 30 \\
\hline
position & 0 & 7 & 15 & 27 & 30 & 31 & 218
\end{tabular}
\end{center}
\begin{enumerate}
\item (4 pts)
Estimate the wombat's velocity at $t=0$, $t=5$, $t=10$
and $t=15$.

At $t=0$ the velocity is 0, since the wombat hasn't moved yet.  For $t=5$ look at the average velocities from 0 to 5 seconds, and from 5 to 10 seconds.
\[\frac{7-0}{5-0}=\frac{7}{5} \]
\[ \frac{15-7}{10-5}=\frac{8}{5}\]
The velocity at $t=5$ should be somewhere in between so taking the mean of the two numbers above gives an estimated velocity of 1.5 ft/sec.  Similarly, at $t=10$ the velocity is about 2 ft/sec, and at $t=15$ is about 1.5 ft/sec.

\vspace{1pc}
\item (2 pts)
Estimate the wombat's acceleration at $t=5$ and $t=10$.

Use the values from part (a), which gives the following table:
\begin{center}
\begin{tabular}{c|cccc}
$t$ & 0 & 5 & 10 & 15 \\
\hline
estimated velocity & 0 & 1.5 & 2 & 1.5 
\end{tabular}
\end{center}
So the average acceleration at $t=5$ can be found by looking at the change in average velocity between 0 and 5 seconds, and between 5 and 10 seconds:
\[\frac{1.5-0}{5-0}=\frac{3}{10}\]
\[\frac{2-1.5}{10-5}=\frac{1}{10}\]
so taking the mean, the average acceleration at $t=5$ seconds is 0.2 ft/sec per sec.  Similarly, at $t=10$ the average acceleration is 0.1 ft/sec per sec.

\vspace{1pc}
\item (1 pt)
What do you think happened between $t=25$ and $t=30$?

During this time the position jumps dramatically.  This means the PHUNCALC wombat's velocity increased dramatically, possibly as a result of being chased by another rodent.
\end{enumerate}

\vspace{1pc}
\item \textbf{Computations/Algebra.} (2 pts) Use the limit definition of the derivative to compute the following.  You \emph{must} show all steps.
\[\frac{d}{dx}\left(\frac{x^2+3}{x^9}\right)\]

\textit{The algebra was wayyy too nasty to actually do this by hand.  The following answer is what you should include at the very least:}
\[=\lim_{h\rightarrow 0}\frac{\frac{(x+h)^2+3}{(x+h)^9}-\frac{x^2+3}{x^9}}{h}\]

\end{enumerate}

\end{document}


