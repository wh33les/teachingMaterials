\documentclass[11pt,letterpaper]{article}
\usepackage{fullpage}
\usepackage{multicol}
\usepackage{amsmath}
\usepackage{amsfonts}
\usepackage{amssymb}
\usepackage{graphicx}
%\usepackage{pstricks, pst-node, pst-plot}

\newcommand{\ds}{\displaystyle}
\newcommand{\bv}{\mathbf}
\newcommand{\lv}{\langle}
\newcommand{\rv}{\rangle}

\begin{document}
\flushleft
\begin{multicols}{2}


\begin{large}\textbf{Math 115 Quiz 7: $\oint $ 3.9, 4.1, 4.2 Using First and Second Derivatives \\
Mon 8 November 2010}\end{large}

\textbf{Name:  }\underline{\hspace{35ex}}

\vspace{.5in}

\end{multicols}

\pagestyle{empty}

\flushleft

You have 20 minutes to complete this quiz.  Make your variables clear and
consistent (so if you want to say, for example, $\frac{dy}{dx}$, you should also
mention $y=f(x)$, or ``$y$ is a function of $x$'').  Calculators are OK.  

\begin{enumerate}
\item \textbf{Definitions/Concepts.} 
\begin{enumerate}
 \item (2 pts) Complete this statement:  Suppose $f$ is differentiable at $a$.  Then, for values of $x$ near $a$, the tangent line approximation to $f(x)$ is

\vspace{3pc}
\item (1 pt) TRUE or FALSE: If the derivative of $f$ is zero at the point $x=a$, then $a$ is either a local maximum or a local minimum.
\end{enumerate}

\vspace{1pc}
\item \textbf{Questions/Problems.} (3 pts) For which powers $p$ is $y=x^p$ concave up on the
region $x\in (0,\infty)$?  Explain.

\vspace{6pc}
\noindent (3 pts) Are exponential functions of the form $y=m\,c^t$
always increasing if $m>0$? If yes, say why. If no, give a
concrete counterexample (equation and sketch of graph).

\vspace{6pc}
%\begin{flushright}
% turn over $\rightarrow $
%\end{flushright}

%\pagebreak
\item \textbf{Computations/Algebra.} (1 pt) Find all critical points of the following function.  Use the second derivative to tell if each critical point is a local maximum, local minimum, or cannot be determined. 
\[f(x)=(x^3-8)^4\]
\end{enumerate}

%----------------------------------------------------------------------------------------

\vspace{1pc}
%\noindent \textbf{ChAlLeNgE PrObLeM:}  

\end{document}


