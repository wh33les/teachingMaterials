\documentclass[%margin%,line,pifont,palatino,courier
]{article}
\usepackage{fullpage}
\usepackage{lastpage}
\usepackage[top=1in,bottom=1in,margin=1in]{geometry}
\usepackage{supertabular}
\usepackage{graphicx,tikz}	
%\usepackage{tkz-euclide}
%\usetkzobj{all}
%\usetikzlibrary{calc}
\usepackage{array,multicol}
\usepackage{amsmath,amssymb}
\usepackage{enumitem}
\usepackage{url}

\usepackage{fancyhdr}
\pagestyle{fancy}

\addtolength{\topmargin}{-0.25in}

\newcommand{\vect}[1]{\mathbf{#1}}
\DeclareMathOperator{\proj}{proj}

\fancypagestyle{plain}{
	\addtolength{\headheight}{0.485in}
	\rhead{\bf MATH 235 (Calculus I) \\
		%\vspace{0.5pc}
		due Tues 24 Oct 2017 \\}
	\rfoot{\footnotesize THQuiz 5, p. \thepage\ (of \pageref{LastPage})
	}
\renewcommand{\headrulewidth}{0pt}
}
\fancyhf{}
\renewcommand{\headrulewidth}{0pt}
\rfoot{\footnotesize THQuiz 5, p. \thepage\ (of \pageref{LastPage})$\;$}

\title{\vspace{-3.5pc} 
	\flushleft \bf \Large Take-Home Quiz 5: Story applications of derivatives %\\ 
	 (\S3.3, 3.5, 3.7-3.8)}
\date{}

% % % % %
\begin{document}
\maketitle

\vspace{-3pc}
\noindent{\bf Directions:} This quiz is due on October 24, 2017 at the beginning of lecture.  You may use whatever resources you like -- e.g., other textbooks, websites, collaboration with classmates -- to complete it \textbf{but YOU MUST DOCUMENT YOUR SOURCES}.  Acceptable documentation is enough information for me to find the source myself.  Rote copying another's work is unacceptable, regardless of whether you document it.  

\noindent\hrulefill

\begin{enumerate}
% % %
%\item {\bf \S3.3 \#26}\begin{enumerate}	
%	\item Find an equation of the tangent line to the curve $y=3x+6\cos x$ at the point $(\frac{\pi}{3}, \pi+3)$.
%	\item Illustrate part (a) by graphing the curve and the tangent line on the same screen \textit{(e.g., using \url{desmos.com})}.
%	\end{enumerate}

% % %
\item {\bf \S3.3 \#36} An elastic band is hung on a hook and a mass is hung on the lower end of the band.  When the mass is pulled downward and then released, it vibrates vertically.  The equation of motion is 
\[
s(t)=2\cos t+3\sin t\quad\text{ for $t\geq 0$,}
\]
where $s$ is measured in centimeters and $t$ is in seconds.
	\begin{enumerate}
	\item Draw a diagram illustrating the problem.  In your coordinate system, assume down is the positive direction.
	\item Find the velocity and acceleration functions.
	\item Graph the velocity and acceleration functions, along with the position function, on the same axes.  Be sure to label which graph is which.
	\item When does the mass pass through the equilibrium position for the first time?
	\item How far from its equilibrium position does the mass travel?
	\item When is its speed the greatest?
	\end{enumerate}

% % %
\item {\bf \S3.5 \#74}  Recall, the \textbf{normal} to the line $y=mx+b$ at a point $(x_0,y_0)$ has slope $-\frac{1}{m}$.  In general, a curve and the normal to its tangent line at $(x_0,y_0)$ intersect in the point $(x_0,y_0)$ at a 90 degree angle.  
	\begin{enumerate}
	\item Where does the normal line to the ellipse $x^2-xy+y^2=3$ at the point $(-1,1)$ intersect the ellipse a second time? \textit{(Give the coordinates of the point.)}
	\item Illustrate part (a) by graphing the ellipse and the normal line.  Label the intersection points.
	\end{enumerate}

% % %
\item {\bf \S3.7 \#26} The population of yeast cells in a laboratory culture is given by 
\[
n(t)=\frac{a}{1+be^{-0.7t}},
\]
where $t$ is measured in hours and $a$ and $b$ are some constants.  

At $t=0$ the population is modeled by the function is 20 cells and is increasing at a rate of 12 cells/hour.  
	\begin{enumerate}
	\item Find $a$ and $b$.
	\item According to this model, what happens to the yeast population in the long run?
	\end{enumerate}

\newpage
% % %
\item {\bf \S3.8 \#12} Scientists can determine the age of fossils using \textbf{carbon dating}.  The bombardment of the upper atmosphere by cosmic rays converts nitrogen to a radioactive isotope of carbon, $^{14}$C (or carbon-14), that has a half-life of about 5730 years.  Plants absorb carbon dioxide through the atmosphere, then animals assimilate $^{14}$C via the food chain.  When a plant or animal dies it stops replacing its carbon and so the amount of $^{14}$C present in the organism when it dies begins to decrease through radioactive decay.

Unfortunately, dinosaur fossils are too old to be reliably dated using carbon dating.  To see why:
	\begin{enumerate}
	\item Using other dating methods, scientists have determined that a particular dinosaur fossil is 68 million years old.  Using the carbon dating method, what fraction of the living dinosaur's $^{14}$C should be remaining today?  
	
	\textit{Hint: Your answer will be very tiny (the Windows calculator app gives an idea of how tiny).  Give the exact value, i.e., what you would plug into a calculator to get it.}
	\item Suppose the minimum detectable amount of $^{14}$C in a fossil is 0.1\% of the amount present when the organism died.  What is the maximum age of a fossil that we could date using carbon-dating?
	\end{enumerate}

% % %
\item {\bf \S3.8 \#16} In a murder investigation, the temperature of the corpse was 32.5$^{\circ}$C at 130p and 30.3$^{\circ}$C an hour later.  The temperature of the corpse's surroundings is 20.0$^{\circ}$C.  Normal body temperature is 37.0$^{\circ}$C.  When did the murder take place?

\textit{Hint: This problem uses Newton's Law of Cooling.}

% % % % %
\end{enumerate}
\end{document}