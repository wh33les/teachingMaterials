\documentclass[%margin%,line,pifont,palatino,courier
]{article}
\usepackage{fullpage}
\usepackage{lastpage}
\usepackage[top=1in,bottom=1in,margin=1in]{geometry}
\usepackage{supertabular}
\usepackage{graphicx,tikz}	
%\usepackage{tkz-euclide}
%\usetkzobj{all}
%\usetikzlibrary{calc}
\usepackage{array,multicol}
\usepackage{amsmath,amssymb}
\usepackage{enumitem}
\usepackage{url}

\usepackage{fancyhdr}
\pagestyle{fancy}

\addtolength{\topmargin}{-0.25in}

\newcommand{\vect}[1]{\mathbf{#1}}
\DeclareMathOperator{\proj}{proj}

\fancypagestyle{plain}{
	\addtolength{\headheight}{0.485in}
	\rhead{\bf MATH 235 (Calculus I) \\
		%\vspace{0.5pc}
		due Wed 15 Nov 2017 \\}
	\rfoot{\footnotesize THQuiz 7, p. \thepage\ (of \pageref{LastPage})
	}
\renewcommand{\headrulewidth}{0pt}
}
\fancyhf{}
\renewcommand{\headrulewidth}{0pt}
\rfoot{\footnotesize THQuiz 7, p. \thepage\ (of \pageref{LastPage})$\;$}

\title{\vspace{-3.5pc} 
	\flushleft \bf \Large Take-Home Quiz 7: Geometric applications of derivatives %\\ 
	 (\S4.3-4.6)}
\date{}

% % % % %
\begin{document}
\maketitle

\vspace{-3pc}
\noindent{\bf Directions:} This quiz is due on November 15, 2017 at the beginning of lecture.  You may use whatever resources you like -- e.g., other textbooks, websites, collaboration with classmates -- to complete it \textbf{but YOU MUST DOCUMENT YOUR SOURCES}.  Acceptable documentation is enough information for me to find the source myself.  Rote copying another's work is unacceptable, regardless of whether you document it.  

\noindent\hrulefill

\begin{enumerate}
% % %
% % %
\item \textbf{Interpreting 1st and 2nd derivatives}
	\begin{enumerate}
	\item {\bf \S4.3 \#66}  On an episode of \textit{The Simpsons}, Homer reads from a newspaper and announces ``Here's good news!  According to this eye-catching article, SAT scores are declining at a slower rate."  Interpret Homer's statement in terms of a function and its first and second derivatives.
	\item {\bf \S4.3 \#68} Let $f(t)$ be the temperature in Moody 201, in degrees Celsius, $t$ minutes after the Calc I Exam 3 has begun.  At $t=3$ you feel uncomfortably hot.  How do you feel about the given data in each of the following cases?
		\begin{enumerate}
		\item $f'(3)=0.2$, $f''(3)=0.4$
		\item $f'(3)=0.2$, $f''(3)=-0.4$
		\item $f'(3)=-0.2$, $f''(3)=0.4$
		\item $f'(3)=-0.2$, $f''(3)=-0.4$
		\end{enumerate}
	\end{enumerate}
	
	
% % % 
\item {\bf \S4.6 \#26} The function $f(x)=(\sin x)^{\sin x}$ is weird.
	\begin{enumerate}
	\item Because of the exponential, $f$ is only defined when $\sin x> 0$.  (\textit{Helpful hint: To see this, on \url{desmos} graph it on the same axes as $y=\sin x$.})  Write down the domain of $f(x)$ using interval notation.  \textit{Hint: For what values of $x$ is $\sin x> 0$?}
	\item Explain why $f(x)$ is periodic.  What is its period?
	\item Find $\lim_{x\to 0^+}f(x)$ and $\lim_{x\to\pi^-}f(x)$ using L'H\^{o}pital's Rule.  
	\item \begin{enumerate}
		\item Use logarithmic differentiation to show 
		\[
		f'(x)=(\sin x)^{\sin x}\cos x\left(\ln(\sin x)+1\right).
		\]
		\item Why doesn't $f(x)$ ever equal 0?
		\item For what value(s) of $x$ does $\cos x=0$?
		\item For what value(s) of $\sin x$ does $\ln(\sin x)=-1$?
		\item Use your answers to parts i.-iv. to identify the critical points for $f$ in the interval $(0,\pi)$.  \textit{Hint: There are three.  You may have to look up information about the function $\arcsin x$ to properly identify them.}
		\end{enumerate}
	\item The second derivative of $f$ is
	\[
	f''(x)=(\sin x)^{\sin x}\left(\ln(\sin x)+1\right)  \left(\cos^2x\left(\ln(\sin x)+1\right) +\frac{\cos^2x}{\sin x}-\sin x\right)
	\]
	(if you wish, you may verify this in private).  Use the 2nd Derivative Test to classify whether or not the critical points are local minima or maxima.  \textit{Hint: You can save some time by using part (d)iv.}  If the test is inconclusive, say so.
	\item Sketch the graph of $f(x)$ (you will need \url{desmos}).  Label the critical points and make sure your graph is consistent with your answers to parts (a)-(e).
	\end{enumerate}

% % % 
\item {\bf \S4.6 \#30} The graph of $f(x)=\ln(x^2+c)$ varies as $c$ varies.  In this problem, we go through the procedures in \S4.5 to investigate how.

\vspace{0.25pc}
\textit{Helpful hint:  In \url{desmos}, type in ``$f(x)=ln(x^2+c)$" and add the slider for $c$.  Click on the wrench icon in the top right corner of the page to change the graphing window to $-10\leq x\leq 10$, $-4\leq y\leq 6$.  Click on the endpoints that come with the slider to change them to $-5\leq c\leq 5$.  There will be a play button with the slider; press it.}

%\vspace{0.25pc}  
	\begin{enumerate}
	\item \textbf{Domain} Find the domain of $f$ (\textit{Hint: Compare it to the domain of $\ln x$.}) when
		\begin{itemize}
			\item $c>0$.
			\item $c=0$.
			\item $c<0$.  \textit{Hint: Draw a number line and shade the values that satisfy the inequality $|x|>\sqrt{-c}$.}			
		\end{itemize}
	\item \textbf{Intercepts} \begin{enumerate}
		\item \underline{$y$-intercepts}: Even though $f$ is a logarithmic function, in this example it will have $y$-intercepts for certain values of $c$.  To find them you must evaluate $f(0)$.
			\begin{itemize}
			\item What are the $y$-intercepts for the values of $c$, when $f$ does have them?
			\item For what values of $c$ does $f$ not have any $y$-intercepts?
			\end{itemize}
		\item \underline{$x$-intercepts}: To find the $x$-intercepts, set $f(x)=0$.  \textit{Hint: Recall that to solve logarithmic equations, you must ``$e$" both sides.}
			\begin{itemize}
			\item For what values of $c$ does $f$ have $x$-intercepts?
			\item What are the $x$-intercepts, when $f$ has them?
			\end{itemize}
		\end{enumerate}
	\item \textbf{Symmetry} \begin{enumerate}
		\item Simplify $f(-x)$.  Is $f$ an even or odd function (or neither)?
		\item Rather than checking $f(x+p)=f(x)$ as in the textbook, explain in words why $f$ isn't periodic in this example.
		\end{enumerate}
	\item \textbf{Asymptotes} \begin{enumerate}
		\item \underline{Horizontal asymptotes}:  Horizontal asymptotes are found by checking the \textbf{end behavior} (the limit as $x$ approaches positive infinity and negative infinity).  If $f$ is symmetric, you can save time by only computing $\lim_{x\to\infty}f(x)$, since 
		\begin{align*}
		\lim_{x\to-\infty}f(x) &= \lim_{x\to\infty}f(x)\quad\text{when $f(x)$ is even and} \\
		\lim_{x\to-\infty}f(x) &= -\lim_{x\to\infty}f(x)\quad\text{ when $f(x)$ is odd.}
		\end{align*}  
		Find the horizontal asymptotes for $f$.  If you use symmetry, then say so.
		\item \underline{Vertical asymptotes}: For the values of $c$ where $f$ has them, the vertical asymptotes of $f$ are at $x=\pm\sqrt{-c}$.  
			\begin{itemize}
			\item For what values of $c$ does $f$ have vertical asymptotes?  How do you know?
			\item In the case where $f$ does have vertical asymptotes, evaluate
			\[
			\lim_{x\to -\sqrt{-c}^-}f(x) \quad\text{ and }\quad \lim_{x\to \sqrt{-c}^+}f(x)
			\]
			(if you use symmetry, explain how).  Why are the one-sided limits necessary?
			\end{itemize}
		\item \underline{Slant asymptotes}: Explain in words why $f$ will not have any slant asymptotes.	
		\end{enumerate}	
	\item \textbf{Extrema}	\textit{Helpful hint: Type ``$y=f'(x)$" into \url{desmos}, using the same browser window as your graph for $f(x)$.} 
		\begin{enumerate}
		\item \underline{Critical points}:
			\begin{itemize}
			\item Compute $f'(x)$.
			\item Identify points \textbf{\textit{in the domain}} where $f'$ does not exist.  Be specific, since your answer depends on the different possible values of $c$.  
			\item Identify points \textbf{\textit{in the domain}} where $f'(x)=0$.  Be specific, since your answer depends on the different possible values of $c$.  
			\end{itemize}
		\item \underline{1st Derivative Test}:  Use it to identify local extrema.  If you are using symmetry as a shortcut, explain how.  Be specific about which values of $c$ you are considering and why.  
		\item \underline{2nd Derivative Test}:  \textit{Helpful hint: Type ``$y=f''(x)$" into \url{desmos}, using the same browser window as your graphs for $f(x)$ and $f'(x)$.}
			\begin{itemize}
			\item Find $f''(x)$.  
			\item Use the 2nd Derivative Test to identify local extrema.  Be specific about which values of $c$ you are considering and why.	
			\end{itemize}
		\item \underline{Global extrema}:
			\begin{itemize}
			\item What is the range of $f(x)$?  Be specific, since it depends on the values of $c$.
			\item For what values of $c$ does $f(x)$ have global extrema?  What are the global extrema in those cases (give the $x$- and $y$-coordinates for each extremum, and whether it is a max or min).
			\end{itemize}
		\end{enumerate}
	\item \textbf{Intervals of Increase or Decrease} Identify the intervals where $f$ is increasing and decreasing when
		\begin{itemize}
		\item $c>0$.
		\item $c=0$.
		\item $c<0$.
		\end{itemize}
	\item \textbf{Concavity and Points of Inflection} Identify the intervals where $f$ is concave up and concave down, as well as any inflection points, when
		\begin{itemize}
		\item $c>0$.
		\item $c=0$.
		\item $c<0$.
		\end{itemize}
	\item \textbf{Sketch the Curve} Sketch a graph of $f$ in each of the following cases.  Make sure in every case your curve is consistent with your answers to parts (a)-(g) by labelling	the domain, the intercepts, the asymptotes, the extrema, and the inflection points.
		\begin{itemize}
		\item $c>0$
		\item $c=0$
		\item $c<0$
		\end{itemize}
	\end{enumerate}	
	
% % % % %
\end{enumerate}
\end{document}