\documentclass[%margin%,line,pifont,palatino,courier
]{article}
\usepackage{fullpage}
\usepackage{lastpage}
\usepackage[top=1in,bottom=1in,margin=1in]{geometry}
\usepackage{supertabular}
\usepackage{graphicx,tikz}	
%\usepackage{tkz-euclide}
%\usetkzobj{all}
%\usetikzlibrary{calc}
\usepackage{array,multicol}
\usepackage{amsmath,amssymb}
\usepackage{enumitem}

\usepackage{fancyhdr}
\pagestyle{fancy}

\addtolength{\topmargin}{-0.25in}

\newcommand{\vect}[1]{\mathbf{#1}}
\DeclareMathOperator{\proj}{proj}

\fancypagestyle{plain}{
	\addtolength{\headheight}{0.485in}
	\rhead{\bf MATH 235 (Calculus I) \\
		%\vspace{0.5pc}
		due Mon 11 Sep 2017 \\}
	\rfoot{\footnotesize THQuiz 1, p. \thepage\ (of \pageref{LastPage})
	}
\renewcommand{\headrulewidth}{0pt}
}
\fancyhf{}
\renewcommand{\headrulewidth}{0pt}
\rfoot{\footnotesize THQuiz 1, p. \thepage\ (of \pageref{LastPage})$\;$}

\title{\vspace{-3.5pc} 
	\flushleft \bf \Large Take-Home Quiz 1: %\\ 
	A catalog of functions for modelling real-world phenomena (\S 1.1-1.3)}
\date{}

% % % % %
\begin{document}
\maketitle

\vspace{-3pc}
\noindent{\bf Directions:} This quiz is due on September 11, 2017 at the beginning of lecture.  You may use whatever resources you like -- e.g., other textbooks, websites, collaboration with classmates -- to complete it \textbf{but YOU MUST DOCUMENT YOUR SOURCES}.  Acceptable documentation is enough information for me to find the source myself.  Rote copying another's work is unacceptable, regardless of whether you document it.  

\noindent\hrulefill

\begin{enumerate}
% % %
\item {\bf 1.2 \#32} The following table shows the mean (average) distances $d$ of the planets from the sun (taking the unit of measurement to be the distance from the earth to the sun) and their periods $T$ (time of revolution in years).

\begin{center}
\begin{tabular}{l|rr}
 & {$d$} & {$T$} \\
\hline
Mercury & 0.387 & 0.241 \\
Venus & 0.723 & 0.615 \\
Earth & 1.000 & 1.000 \\
Mars & 1.523 & 1.881 \\
Jupiter & 5.203 & 11.861 \\
Saturn & 9.541 & 29.457 \\
Uranus & 19.190 & 84.008 \\
Neptune & 30.086 & 164.784
\end{tabular}
\end{center}
	
	\begin{enumerate}
	\item Fit a power model to the data:  
		\begin{itemize}
		\item Plot the data on a graph with horizontal axis $d$ and vertical axis $T$.  
		\item Choose an appropriate value of $n$ so that $T(d)=d^n$ best fits your data.  Be clear in your process on how you chose $n$.
		\item Draw the graph of $T(d)$ on the same axes as the points you plotted, to illustrate the fit. 
		\end{itemize}	
	\item Kepler's Third Law of Planetary Motion states that ``The square of the period of revolution of a planet is proportional to the cube of its mean distance from the sun."
	
	Does your model corroborate Kepler's Third Law?  Explain how it does or doesn't.
	\end{enumerate}
	
% % %
\item {\bf 1.3 \#26} A \emph{variable star} is one whose brightness alternately increases and decreases.  For the most visible variable star, Delta Cephei, the time between periods of maximum brightness is 5.4 days, the average brightness (or \emph{magnitude}) of the star is 4.0, and its brightness varies by $\pm$0.35 magnitude.  Find a function (formula) that models the brightness of Delta Cephei as a function of time.  Include a well-labelled graph of your function and argue why your function is consistent with information given.
	
% % %
%\item {\bf 1.3 \#56} A spherical balloon is being inflated and the radius of the balloon is increasing at a rate of 2 cm/s.
%	\begin{enumerate}
%	\item Express the radius $r$ of the balloon as a function of the time $t$ (in seconds).
%	\item If $V$ is the volume of the balloon as a function of the radius, find $V\circ r$ and interpret it.
%	\end{enumerate}

% % %
\item {\bf 1.3 \#60} The \emph{Heaviside function} $H$ is defined by 
\[
H(t) = \begin{cases}
	0 & \text{if } t<0 \\
	1 & \text{if } t\geq 0
	\end{cases}.
\]  
It is used in the study of electric circuits to represent the sudden surge of electric current, or voltage, when a switch is instantaneously turned on.  It can also be used to define the \emph{ramp function} $y=ctH(t)$, which represents a gradual increase in the voltage or current in a circuit.
	\begin{enumerate}
	\item On the same axes, sketch the graphs of the Heavidside function and the ramp function.  Make sure to label which graph is which.  
	\item What are the restrictions on the possible values of $c$?  Where do the two graphs in part (a) intersect?
	\item On a different set of axes, sketch the graph of the voltage $V(t)$ in a circuit if the switch is turned on at time $t=0$ and the voltage is gradually increased to 120 volts over a 60-second time interval.  Write a forumla for $V(t)$ in terms of $H(t)$ for $t\leq 60$.
	\item On another set of axes, sketch the graph of the voltage $V(t)$ in a circuit if the switch is turned on at time $t=7$ seconds and the voltage is gradually increased to 100 volts over a period of 25 seconds.  Write a formula for $V(t)$ in terms of $H(t)$ for $t\leq 32$.
	\end{enumerate}

% % %
\item {\bf 1.3 \#62} If you invest $x$ dollars at 4\% interest compounded annually, then the amount of the investment after one year is $A(x)=1.04x$.  Find $A\circ A$, $A\circ A\circ A$, and $A\circ A\circ A\circ A$.  What do these compositions represent?  Find a formula for the composition of $n$ copies of $A$.

% % %
\item {\bf 1.3 \#64} If $f(x)=x+4$ and $h(x)=4x-1$, find a function $g$ such that $g\circ f=h$.

% % % % %
\end{enumerate}
\end{document}