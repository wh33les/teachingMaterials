\documentclass[%margin%,line,pifont,palatino,courier
]{article}
\usepackage{fullpage}
\usepackage{lastpage}
\usepackage[top=1in,bottom=1in,margin=1in]{geometry}
\usepackage{supertabular}
\usepackage{graphicx,tikz}	
%\usepackage{tkz-euclide}
%\usetkzobj{all}
%\usetikzlibrary{calc}
\usepackage{array,multicol}
\usepackage{amsmath,amssymb}
\usepackage{enumitem}
\usepackage{url}

\usepackage{fancyhdr}
\pagestyle{fancy}

\addtolength{\topmargin}{-0.25in}

\newcommand{\vect}[1]{\mathbf{#1}}
\DeclareMathOperator{\proj}{proj}

\fancypagestyle{plain}{
	\addtolength{\headheight}{0.485in}
	\rhead{\bf MATH 235 (Calculus I) \\
		%\vspace{0.5pc}
		due Tues 5 Dec 2017 \\}
	\rfoot{\footnotesize THQuiz 8, p. \thepage\ (of \pageref{LastPage})
	}
\renewcommand{\headrulewidth}{0pt}
}
\fancyhf{}
\renewcommand{\headrulewidth}{0pt}
\rfoot{\footnotesize THQuiz 8, p. \thepage\ (of \pageref{LastPage})$\;$}

\title{\vspace{-3.5pc} 
	\flushleft \bf \Large Take-Home Quiz 8: Optimization and integrals %\\ 
	 (\S4.7, 4.9, 5.1-5.5)}
\date{}

% % % % %
\begin{document}
\maketitle

\vspace{-3pc}
\noindent{\bf Directions:} This quiz is due on December 5, 2017 at the beginning of lecture.  You may use whatever resources you like -- e.g., other textbooks, websites, collaboration with classmates -- to complete it \textbf{but YOU MUST DOCUMENT YOUR SOURCES}.  Acceptable documentation is enough information for me to find the source myself.  Rote copying another's work is unacceptable, regardless of whether you document it.  

\noindent\hrulefill

\begin{enumerate}

% % %
\item {\bf \S4.7 \#56} The equation of a line with slope $m$ passing through the point $(3,5)$ is given by 
\[
y-5=m(x-3).
\]
If $m$ is negative, then that line will cut out a triangle in the first quadrant.  Find the value of $m$ that minimizes the area of the triangle.


% % %
\item {\bf \S4.9 \#68} Prof Wheeler tosses a red ball from the edge of a 432 ft cliff at an upward speed of 48 ft/s.  One second later, she tosses a blue ball off the edge of the cliff at an upward speed of 24 ft/s.
	\begin{enumerate}
	\item Recall, the acceleration force due to gravity is 32 ft/s$^2$.  Use it to find the height, $h_{\text{red}}(t)$ of the red ball at time $t$ seconds (assume Wheeler tosses the ball at $t=0$).  \textit{Hint: This is done in Example 7.}
	\item Find the height $h_{\text{blue}}(t)$ of the blue ball, $t$ seconds after Wheeler has tossed the first ball.  In other words, assume Wheeler tosses the blue ball at $t=1$. 
	\item Do the balls ever pass eachother?  If so, at what time?
	\item Which ball hits the ground first?
	\item Sketch or print a graph of $h_{\text{red}}$ and $h_{\text{blue}}$ on the same axes.
	\end{enumerate}

% % %
\item {\bf \S5.1 \#26} \textit{How well can you estimate the area under the curve $f(x)=x^3$ from $0$ to $1$, using five approximation rectangles?}

\hspace{5pt} Before calculus was invented, people estimated the area under a curve by drawing rectangles to get an approximate shape, and then adding up the areas of the rectangles.  The more rectangles, the better the approximation.
  
	\begin{enumerate}
	\item Sketch or print the graph of $f(x)$ illustrating your choice of how to draw the five rectangles.  \textit{Hint: For a better estimate, your rectangles don't all have to be the same size!}
	\item In the following table, $k$ enumerates the rectangles you've drawn.  For example, $x_2$ and $x_3$ are the endpoints of Rectangle No. 3.  The width of Rectangle No. 3 is $\Delta x_3=x_3-x_2$.  The number you choose to plug into $f(x)$ for the height of Rectangle No. 3 is $x_3^*$, and must be between $x_2$ and $x_3$.  
	
	\vspace{0.5pc}
	Fill in the table to correspond to your choice of rectangles.
	\begin{center}
	\begin{tabular}{c | c | c | c | c}
	$k$ & $x_k$ & $x_k^*$ & $f(x_k^*)$ & $\Delta x_k^*$ \\[0.25pc]
	\hline
	 & & & & \\[-0.75pc]
	0 & 0 & - & & \\[0.25pc]
	1 & & & & \\[0.25pc]
	2 & & & & \\[0.25pc]
	3 & & & & \\[0.25pc]
	4 & & & & \\[0.25pc]
	5 & 1 & & & 
	\end{tabular}
	\end{center}
	
	\item The area is $\displaystyle \approx \sum_{k=1}^5f(x_k^*)\Delta x_k^*=$ ?
	
	\end{enumerate}
	
% % % 
\item {\bf \S5.5 \#62} Consider the integral $\displaystyle \int_0^{\frac{\pi}{2}}\cos x \sin(\sin x)\ dx$. 
	\begin{enumerate}
	\item This integral requires $u$-substitution to solve.  Find $u$ and $du$.
	\item Rewrite the integral in terms of $u$, \underline{including the bounds}.
	\item Sketch or print a graph of the integrands of both integrals on the same axes.  Under each curve, shade the area the integrals compute.  
	\item Compute the integral you wrote from part (b).  Use a calculator or online integrator (\url{desmos} will do it, for example) to check your answer equals the original integral.
	\end{enumerate}
	
% % % % %
\end{enumerate}
\end{document}