\documentclass[%margin%,line,pifont,palatino,courier
]{article}
\usepackage{fullpage}
\usepackage{lastpage}
\usepackage[top=1in,bottom=1in,margin=1in]{geometry}
\usepackage{supertabular}
\usepackage{graphicx,tikz}	
%\usepackage{tkz-euclide}
%\usetkzobj{all}
%\usetikzlibrary{calc}
\usepackage{array,multicol}
\usepackage{amsmath,amssymb}
\usepackage{enumitem}
\usepackage{url}

\usepackage{fancyhdr}
\pagestyle{fancy}

\addtolength{\topmargin}{-0.25in}

\newcommand{\vect}[1]{\mathbf{#1}}
\DeclareMathOperator{\proj}{proj}

\fancypagestyle{plain}{
	\addtolength{\headheight}{0.485in}
	\rhead{\bf MATH 235 (Calculus I) \\
		%\vspace{0.5pc}
		due Thurs 9 Nov 2017 \\}
	\rfoot{\footnotesize THQuiz 6, p. \thepage\ (of \pageref{LastPage})
	}
\renewcommand{\headrulewidth}{0pt}
}
\fancyhf{}
\renewcommand{\headrulewidth}{0pt}
\rfoot{\footnotesize THQuiz 6, p. \thepage\ (of \pageref{LastPage})$\;$}

\title{\vspace{-3.5pc} 
	\flushleft \bf \Large Take-Home Quiz 6: Technical applications of derivatives %\\ 
	 (\S3.10, 4.1-4.2)}
\date{}

% % % % %
\begin{document}
\maketitle

\vspace{-3pc}
\noindent{\bf Directions:} This quiz is due on November 9, 2017 at the beginning of lecture.  You may use whatever resources you like -- e.g., other textbooks, websites, collaboration with classmates -- to complete it \textbf{but YOU MUST DOCUMENT YOUR SOURCES}.  Acceptable documentation is enough information for me to find the source myself.  Rote copying another's work is unacceptable, regardless of whether you document it.  

\noindent\hrulefill

\begin{enumerate}
% % %
\item {\bf \S3.10 \#10} Let $f(x)=e^x\cos x$.
	\begin{enumerate}
	\item Find the linearization of $f(x)=e^x\cos x$ near $x=0$.  
	\item Determine the values of $x$ for which the approximation is accurate to within 0.1.  \textit{Hint: On \url{desmos.com} type in $|e^x\cos x-1-x|<0.1$.}
	\item Find the differential $dy$.
	\item Let $\Delta x=0.5$.  Evaluate $dy$ and $\Delta y$.	
	\item Graph $f(x)$ centered at $x=0$.  On the same graph illustrate your answers to parts (a)-(d).  Label $dx$ on your graph, too.	
	\end{enumerate}
	
% % % 
\item {\bf \S3.10 \#32} Let $f(x)=(x-1)^2$, $g(x)=e^{-2x}$, and $h(x)=1+\ln(1-2x)$.
	\begin{enumerate}
	\item Find the linearizations of $f$, $g$, and $h$ at $x=0$.  What do you notice?  How do explain what happened?
	\item Graph $f$, $g$, $h$, and their linear approximations on the same graph.  For which function is the linear approximation the best?  For which is it the worse?  Explain.
	\end{enumerate}

% % % 
\item {\bf \S4.1 \#40} Let $g(\theta)=4\theta-\tan{\theta}$.
	\begin{enumerate}
	\item What is the image (range) of $g$?
	\item Find the critical points of $g$.  Give both coordinates $(\theta,g(\theta))$ for each one.
	\item Graph $g$.  Which critical points are global extrema (if any)?  Which are only local extrema (if any)?
	\end{enumerate}

% % %
\item {\bf \S4.2 \#36} At 2p a car's speedometer reads 30 mph.  At 210p it reads 50 mph.  Show that at some time between 2p and 210p the acceleration is exactly 120 mi/hr$^2$.	
	
% % % % %
\end{enumerate}
\end{document}