\documentclass[12pt]{article}
\usepackage{fullpage}
\usepackage{enumerate}
\usepackage{multicol}
\usepackage{comment}
\usepackage{lastpage}
\usepackage{fancyhdr}
\pagestyle{fancy}

\addtolength{\topmargin}{-0.25in}
\usepackage{graphicx}	
\usepackage{array, multicol}
\usepackage{amsmath,amsfonts}

\everymath{\displaystyle}

\fancypagestyle{plain}{
	\fancyhf{}
	\addtolength{\headheight}{2.92\baselineskip}
	\lhead{\bf MATH 2574 (Calculus III) \\
		Fall 2015 \\
		}
	\rhead{{Name:} \underline{\hspace{40ex}} \\
		\vspace{0.5pc}
		Thurs 8 Oct 2015}
	\rfoot{Quiz 6 p.\thepage\ (of \pageref{LastPage})}
	}
\fancyhf{}
\renewcommand{\headrulewidth}{0pt}

\title{\flushleft\vspace{-2.5pc}\Large
	\bf Quiz 6: Directional Derivatives ($\textstyle\oint$12.4-12.6)}
\author{}
\date{}

\rfoot{Quiz 6 p.\thepage\ (of \pageref{LastPage})}

% % % % %
\begin{document}
\maketitle

\vspace{-4pc}
\noindent{\bf Directions:} You have 30 minutes to complete this quiz.  You may collaborate.  

%\vspace{1pc}
\begin{enumerate}[1.]
\item The level curves of the surface $z=x^2+y^2$ are circles in the $xy$-plane centered at the origin.  
	\begin{enumerate}%[(a)]
	\item {\bf (1 pt)} In the $xy$-plane, draw and label the three level curves corresponding to $z_0=1,\sqrt{2},\text{ and }4$.  
	\vspace{10pc}
	\item {\bf (2 pts)} On the same picture, draw the gradient vector at the point $(1,1)$.
	\item {\bf (2 pts)} Write down the gradient vector: $\langle \frac{\partial z}{\partial x},\frac{\partial z}{\partial y}\rangle=$
	\vspace{1pc}
	\end{enumerate}

\item {\bf (2 pts)} Compute the directional derivative of $h(x,y)$, in the direction of the vector $\mathbf v$, at the point $(\ln 2,\ln 3)$, where 
\[
\mathbf v=\langle 1,1\rangle \quad\text{and}\quad h(x,y)=e^{-x-y}.
\]
{\it Hint: Make sure your answer has the correct magnitude!}

\newpage
\item Suppose our Sun is centered at the origin in $\mathbb R^3$ and some other star is a distance $r$ away, at the coordinates $(x,y,z)$.  The {\bf gravitational potential} between the two stars (or any two objects) is the function
\[
V(r)=\frac{-GMm}{r},
\]
where $G$ is the gravitational constant, and in this case $m$ denotes the mass of the Sun and $M$ denotes the mass of the other star.
	\begin{enumerate}
	\item {\bf (1 pt)} Write down $V$ as a function of $x,y,\text{ and }z$.
	\vspace{3pc}
	\item {\bf (2 pts)} The gravitational force between the two stars is the vector-valued function 
	\[
	\mathbf F=-\nabla V(x,y,z).
	\]
	Write down the magnitude $|\mathbf F|$ as a function of $r$.
	\end{enumerate}
\end{enumerate}
% % % % %
\end{document}