%%% qformat gives the same style for all of the quizzes
\include{tformat}
\usepackage{graphicx}
\usepackage{color}
\usepackage{amsmath,esint}

\newcommand{\del}{\partial}

%%% this is a way of compressing drafts
\newif\ifusespace
\usespacetrue  %% comment this line out to compress

\def\bigSpace{\ifusespace\vspace{23ex}\fi}
\def\newPage{\ifusespace\newpage\fi}

\makeatletter
\newcommand*\dotp{\mathpalette\dotp@{.5}}
\newcommand*\dotp@[2]{\mathbin{\vcenter{\hbox{\scalebox{#2}{$\m@th#1\bullet$}}}}}
\makeatother

\fancyfoot[R]{\thepage}

% % % % % % % % % % % % % % % 
\begin{document}

%% #1 is the test version
%% the command \vitem#1{A}{B}{C}{D} spits out A,B, C or D
%% depending on the version. This is defined in qformat.

\newcommand{\testVersion}[1]{

\clearpage
\thispagestyle{empty}
\pagestyle{empty}
\setcounter{page}{0} 

%% thead#1 creates the title, with #1 as part of the heading
\thead{2574C}{3}{Fall 2015} \\

{\bf Directions:} No calculators, phones or other electronic aids are allowed.  Show all your work. If you use a formula from memory, write it down. \emph{Clearly indicate your final answer.} You will be graded not only on your final answer, but on the clarity of your solutions. 

\

\pagestyle{fancy}

\hfill Name \rule{2.75in}{1pt} \phantom{ } \hfill TA Name: \rule{1.25in}{1pt}  \\ 
\phantom{ } \hfill
%\hfill 
%{\Large Read the directions and good luck!} 
\hfill Drill Time: \rule{1.25in}{1pt} \newline \newline 

\vspace{20pt} 

%% This needs to be edited with the correct number of problems and point values

\begin{center}
\begin{tabular}{| l | r |}
\hline
\multicolumn{2}{| c |}
	{\phantom{$\displaystyle\int$xxxx}GRADE \phantom{$\displaystyle\int$xxxx}} \\ 
\hline
Problem 1 & \phantom{$\displaystyle\int$} \ \ \ \ \ / 20 \\
\hline
Problem 2 & \phantom{$\displaystyle\int$} \ \ \ \ \ / 10 \\
\hline
Problem 3 & \phantom{$\displaystyle\int$} \ \ \ \ \ / 10 \\
\hline
Problem 4 & \phantom{$\displaystyle\int$} \ \ \ \ \ / 25 \\
\hline
Problem 5 & \phantom{$\displaystyle\int$} \ \ \ \ \ / 20 \\
\hline
Problem 6 & \phantom{$\displaystyle\int$} \ \ \ \ \ / 15 \\
\hline
Total \ \ \ \ & \phantom{$\displaystyle\int$}\ \ \ \ \ /100 \\
\hline
\end{tabular}
\end{center}

\vfill
#1

% % % % % % % % % % 
\newpage

\ben

\item\textbf{(20 pts)} Evaluate the following integral exactly as written.
\[{\vitem#1
	{\int_0^{\ln 8} \int_0^{4} \int_0^{\ln 2} ye^{-x-z}\, dx\, dy\, dz} % 1
	{\int_0^{8} \int_0^{\ln 4} \int_0^{\ln 2} 2ze^{-x-y}\, dx\, dy\, dz} % 2
	{\int_0^{8} \int_0^{\ln 4} \int_0^{\ln 2} xe^{-y-2z}\, dz\, dy\, dx} % 3	
	{\int_0^{8} \int_0^{\ln 4} \int_0^{\ln 2} 2ye^{-x-2z}\, dz\, dx\, dy} % 4
}\]

% % % % %
\newpage

\item\textbf{(10 pts)} Compute the average value of 
$\vitem#1
	{f(x,y)=\sin x \sin y} % 1
	{g(x,y)=\cos x \sin y} % 2
	{f(x,y)=\sin x \cos y} % 3
	{g(x,y)=\cos x \cos y} % 4
	$ 
over the region 
\[\vitem#1
	{R=\{(x,y)\colon 0\le x\le \pi,\, 0\le y\le \pi \}} % 1
	{R=\{(x,y)\colon 0\le x\le \textstyle\frac{\pi}{2},\, 0\le y\le \pi \}} % 2
	{R=\{(x,y)\colon 0\le x\le \pi,\, 0\le y\le \textstyle\frac{\pi}{2} \}} % 3
	{R=\{(x,y)\colon 0\le x\le \textstyle\frac{\pi}{2},\, 0\le y\le \frac{\pi}{2} \}}	% 4
.\]

\vspace{22pc}

% % % 
\item \textbf{(10 pts)} Consider the integral 
\[\vitem#1
	{\int_0^1 \int_1^{e^y} f(x,y)\, dx\, dy} % 1
	{\int_1^e \int_0^{\ln x} f(x,y)\, dy\, dx} % 2
	{\int_0^1 \int_1^{e^x} f(x,y)\, dy\, dx} % 3
	{\int_1^e \int_0^{\ln y} f(x,y)\, dx\, dy} % 4
.\]
Sketch the region of integration and then rewrite the integral in the order 
$\vitem#1
	{dy\, dx} % 1
	{dx\, dy} % 2
	{dx\, dy} % 3
	{dy\, dx} % 4
	$.  
%\vspace{20pc}

% % % % %
\newpage

\item A spherical fish tank of radius% 
\vitem#1
	{1 ft} % 1
	{2 ft} % 2
	{10 in} % 3
	{20 in} % 4
is filled with water to a level% 
\vitem#1
	{6 in} % 1
	{1 ft} % 2
	{5 in} % 3
	{10 in} % 4
from the top.  
\ben
	\item\textbf{(4 pts)} On the sphere below, draw and label the tank's radius and water level, with units included. 
	
	\vspace{-1.5pc}
	\begin{center}
	\includegraphics[scale=0.8]{exam2Sphere}
	\end{center}
	
	\vspace{-2pc}
	\item\textbf{(2 pts)} Write the equation for your sphere, in spherical coordinates.
	\vspace{2pc}
	
	\item\textbf{(9 pts)} Write down a triple integral that will give the volume of the \textit{empty space} in the fish tank.
	\vspace{8pc}
	
	\item\textbf{(7 pts)} Evaluate the integral from (c).  
	\vfill
	
	\item\textbf{(3 pts)} What is the volume of the water in the tank?%Use a triple integral to find the volume of the water in the tank.  Include units in your final answer.  \textit{Hint: Subtract the volume of the empty space in the tank from $\frac{4}{3}\pi\rho^3$.}  
\een

% % % % % 
\newpage

\item\textbf{(20 pts)} Evaluate the following integral using a change of variables of your choice.  Sketch the original and new regions of integration, $R$ and $S$.
\[\vitem#1
	{\iint_R(z-w)\sqrt{2z-w}\, dA} % 1
	{\iint_R(x-y)\sqrt{y-2x}\, dA} % 2
	{\iint_R(z-w)\sqrt{z-2w}\, dA} % 3
	{\iint_R(x-y)\sqrt{2x-y}\, dA} % 4
\] 
$R$ is bounded by the lines%
\vitem#1
	{$w=2z-2$, $w=2z$, $w=z-3$, and $w=z-1$.} % 1
	{$y=2x-2$, $y=2x$, $y=x-3$, and $y=x-1$.} % 2
	{$w=\frac{z}{2}-2$, $w=\frac{z}{2}$, $w=z-3$, and $w=z-1$.} % 3
	{$y=2x+2$, $2x-y=0$, $y=x-3$, and $y=x-1$.} % 4

% % % % %
\newpage

\item\textbf{(15 pts)} For the integral below, sketch the region of integration and evaluate the integral using polar coordinates.
\[\vitem#1
	{\int_{-2}^2 \int_0^{\sqrt{4-y^2}} (4-x^2-y^2)\, dx\, dy} % 1
	{\int_{-3}^3 \int_0^{\sqrt{9-y^2}} (9-x^2-y^2)\, dx\, dy} % 2
	{\int_{-5}^5 \int_0^{\sqrt{25-y^2}} (25-x^2-y^2)\, dx\, dy} % 3
	{\int_{-4}^4 \int_0^{\sqrt{16-y^2}} (16-x^2-y^2)\, dx\, dy} % 4
\]

% % % % % % % % % %

\een

\newpage

}

\testVersion{1}
%\setcounter{page}{0}     %% Resets the page counter back to 1 after the first iteration
\testVersion{2}
%\setcounter{page}{0}     %% Resets the page counter back to 1 after the first iteration
\testVersion{3}
%\setcounter{page}{0}     %% Resets the page counter back to 1 after the first iteration
\testVersion{4}
%\setcounter{page}{0}     %% Resets the page counter back to 1 after the first iteration

\end{document} 