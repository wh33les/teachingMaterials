\documentclass[12pt]{beamer}
\usetheme{Warsaw}
\usecolortheme{beaver,lily}
\usefonttheme{professionalfonts}
\setbeamertemplate{frametitle continuation}[from second][(cont.)]

\setbeamerfont{section in toc}{size=\fontsize{10}{11}\selectfont}
\setbeamerfont{subsection in toc}{size=\fontsize{8}{9}\selectfont}
\setbeamerfont{subsubsection in toc}{size=\fontsize{6}{7}\selectfont}
\setbeamercolor{section number projected}{bg=red!50!black,fg=white}
\setbeamercolor{subsection in toc}{fg=red!50!black}
\setbeamercolor{subsubsection in toc}{fg=black}
\setbeamercolor{titlelike}{fg=red!50!black}
\setbeamertemplate{subsection in toc}{
	\inserttocsubsection
	
	} % don't get rid of the above line!
\setbeamertemplate{subsubsection in toc}{
	\hspace{1.5pc}
	\textbullet\ \inserttocsubsubsection
	
	}
	
\usepackage{subfiles}
\usepackage{multicol}
\usepackage{url,hyperref}
\usepackage{tikz}
\usepackage{tkz-euclide}
%\usetkzobj{all}
\usepackage{comment}
\usepackage{enumerate}

% % %	
\begin{comment}
\AtBeginSubsection{
	\begin{frame}{}\footnotesize
	\begin{multicols}{2}
	\tableofcontents[currentsubsection,hideothersubsections]
	\end{multicols}
	\end{frame}
	}
\end{comment}	
% % %

\theoremstyle{plain}
\newtheorem{thm}{Theorem}

\theoremstyle{definition}
\newtheorem{que}{Question}
\newtheorem{ex}{Example}
\newtheorem{exe}{Exercise}
\newtheorem{dfn}{Definition}

\everymath{\displaystyle}

% % % % % % % % % %
\title[Cal III Fall 2015]{Calculus III (Math 2574)}
\subtitle{Fall 2015}
\author[Wheeler]{\footnotesize Dr. Ashley K. Wheeler}
\institute{University of Arkansas}
\date{\footnotesize{\it last updated:} \today}

% % %
\begin{document}

%\frame{\titlepage}

% % %
%\begin{frame}[allowframebreaks=1.00]{Table of Contents}
%\begin{multicols}{2}
%\tableofcontents
%\end{multicols}
%\end{frame}

% % %
%\subsection{Fri 9 Oct}
\begin{frame}[allowframebreaks]{Chapter 12 (Exam 2) Review}\footnotesize
\begin{itemize}
\item $\sim$7 questions
\item Get comfortable with ``gnarly" computations.  The best way is to show work while practicing.  See Exam 1 solutions for other shortcuts in work.
\item No calculators.  
\item Equations of plane, intersection of planes.  Make sure you know relevant formulas (e.g., cross product).
\item Recall how to parametrize lines and curves.
\item What does the gradient vector represent?  Know the relationship between the gradient vector and the tangent plane.  
\framebreak
\item Linear approximation formula and how to use it.
\item Chain Rule for functions of more than one variable.  Partial derivatives, when to use $\partial$ versus when to use $d$.  
\item Make sure your notation is correct.
\item Limits for functions of two variables.
\item Equation of a tangent plane.
\item Recognize graphs and/or contour plots.  You won't need to draw any graphs.
\item Critical points, local maxima/minima, saddle points, Extreme Value Theorem.
\end{itemize}
\end{frame}

\end{document}