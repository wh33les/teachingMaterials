\documentclass[12pt]{article}
%\usepackage{amsrefs}
\usepackage{fullpage}
\usepackage{enumerate}
\usepackage{multicol}
\usepackage{comment}
\usepackage{lastpage}
\usepackage{fancyhdr}
\pagestyle{fancy}

\addtolength{\topmargin}{-0.25in}
\usepackage{graphicx}	
\usepackage{array, multicol}
\usepackage{amsmath,amsfonts}

\everymath{\displaystyle}
\renewcommand*{\thefootnote}{\fnsymbol{footnote}}
\newcommand{\vect}[1]{\mathbf{#1}}
\DeclareMathOperator{\curl}{curl}
\DeclareMathOperator{\Div}{div}

\fancypagestyle{plain}{
	\fancyhf{}
	\addtolength{\headheight}{2.92\baselineskip}
	\lhead{\bf MATH 2574 (Calculus III) \\
		Fall 2015 \\
		}
	\rhead{{Name:} \underline{\hspace{40ex}} \\
		\vspace{0.75pc}
		{Drill:} \underline{\hspace{40ex}} \\
		\vspace{0.5pc}
		Tues 1 Dec 2015}
	\rfoot{Quiz 12 p.\thepage\ (of \pageref{LastPage})}
	}
\fancyhf{}
\renewcommand{\headrulewidth}{0pt}

\title{\flushleft\vspace{-2pc}\Large
	\bf Quiz 12: Divergence, Curl, and Green's Theorem (\S14.4-14.5)}
\author{}
\date{}

\rfoot{Quiz 12 p.\thepage\ (of \pageref{LastPage})}

% % % % %
\begin{document}
\maketitle

\vspace{-3pc}
%\noindent{\bf Directions:} You may collaborate.  
%
%\vspace{1pc}
\begin{enumerate}
% % % 
\item \textbf{(4 pts)} $\vect F=\langle e^{-x+y},e^{-y+z},e^{-z+x}\rangle$ is a vector field in $\mathbb R^3$. %\textbf{(3 pts)} Suppose $\vect F=\langle f,g,h\rangle$ is a vector field in $\mathbb R^3$.  The \textit{del operator} is: 
%\vspace{0.5pc}
%\[
%\text{$\nabla=$\;{\hspace{50ex}}\;}
%\]  
%
%Write the following formulas in terms of $\nabla$:
	\begin{enumerate}
	\item $\curl{\vect F}=$
	\vspace{8pc}
	
	\item $\Div{\vect F}=$
	\vspace{8pc}
	\end{enumerate}

% % % 
\item \textbf{(2 pts)} \textbf{Green's Theorem:} Let $C$ be a simple closed smooth curve, oriented counterclockwise, that encloses a connected and simply connected region $R$ in the plane.  Let $\vect F=\langle f(x,y), g(x,y)\rangle$ denote a vector field, where $f$ and $g$ have continuous first partial derivatives on $R$.  Then
	\begin{enumerate}
	\item Green's Theorem says the circulation of $\vect F$ on $R$ is (write the equation):%$\oint_Cf\ dx+g\ dy=$
	\vspace{4pc}
	
	%and this is the\; \underline{\hspace{25ex}}\; form of Green's Theorem.
	\vspace{1pc}
	
	\item and that the flux of $\vect F$ across the boundary of $R$ is (write the equation):%$\oint_Cf\ dy-g\ dx=$
	\vspace{4pc}
	
	%and this is the\; \underline{\hspace{25ex}}\; form of Green's Theorem.
	\vspace{1pc}
	\end{enumerate}
\vfill
\newpage

% % %
\item \textbf{(4 pts)} Let $\vect{F}=\langle x-y,-x+2y \rangle$ denote a vector field on the paralleogram
\[
R=\left\{(x,y)\mid 1-x\leq y\leq 3-x,0\leq x\leq 1  \right\}.
\]
Compute (a) the circulation of $\vect{F}$ on $R$ and (b) the outward flux of $\vect F$ across the boundary of $R$.  




\end{enumerate}
% % % % %
\end{document}