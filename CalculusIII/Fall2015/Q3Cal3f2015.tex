\documentclass[12pt]{article}
\usepackage{fullpage}
\usepackage{enumerate}
\usepackage{multicol}
\usepackage{comment}
\usepackage{lastpage}
\usepackage{fancyhdr}
\pagestyle{fancy}

\addtolength{\topmargin}{-0.25in}
\usepackage{graphicx}	
\usepackage{array, multicol}
\usepackage{amsmath}

\everymath{\displaystyle}

\fancypagestyle{plain}{
	\fancyhf{}
	\addtolength{\headheight}{2.92\baselineskip}
	\lhead{\bf MATH 2574 (Calculus III) \\
		Fall 2015 \\
		}
	\rhead{{Name:} \underline{\hspace{40ex}} \\
		\vspace{0.5pc}
		Tues 15 Sep 2015}
	\rfoot{Quiz 3 p.\thepage\ (of \pageref{LastPage})}
	}
\fancyhf{}
\renewcommand{\headrulewidth}{0pt}

\title{\flushleft\vspace{-1.5pc}\Large
	\bf Quiz 3: Trajectories and Arc Length ($\textstyle\oint$11.7-11.8)}
\author{}
\date{}

\rfoot{Quiz 3 p.\thepage\ (of \pageref{LastPage})}

% % % % %
\begin{document}
\maketitle

\vspace{-3pc}
\noindent{\bf Directions:} You have 30 minutes to complete this quiz.  You may collaborate.  

\vspace{1pc}
\begin{enumerate}[1.]
%\item {\bf (2 pts)} Find the length of the spiral $\rho(\theta)=e^{-a\theta}$ for $\theta\geq 0$, $a>0$.
\item {\bf (2 pts)} A cycloid is the path traced by a point on a rolling circle (think of a light on the rim of a moving bicycle wheel).  The cycloid generated by a circle of radius $a$ is given by the parametric equation 
\[
x=a(t-\sin{t}) \quad y=a(1-\cos{t}).
\]
\begin{enumerate}[(a)]
\item The parameter range $0\leq t\leq 2\pi$ produces one arch of the cycloid.  Compute its length.  {\bf Hint:} You might need the half-angle formula
\[
\sin^2{\theta}=\frac{1}{2}\left(1-\cos^2{\theta}\right).
\]
\vspace{10pc}

\item Draw a well-labelled graph of the arch of the cycloid.
\end{enumerate}

\newpage
\item A golf ball has an initial position 
\[
\overrightarrow{r}(0)=\langle x_0,y_0\rangle = \langle 0,0\rangle = 0\hat i + 0\hat j \text{ ft}
\]
when it is hit at an angle of $30^{\circ}$ from the ground and with an initial speed of $150$ ft/s.  For the following, neglect air resistance and assume gravity is a constant $g=32$ ft/s$^2$.  {\bf You must include units in your answers to receive credit.}
	\begin{enumerate}[(a)]
	\item {\bf (1 pt)} The golf ball's acceleration vector is: $\overrightarrow{a}(t)=$
	\vspace{1pc} 
	\item {\bf (1 pt)} Its initial speed is: $|\overrightarrow{v}(0)|=$
	\vspace{1pc}
	\item {\bf (1 pt)} Its initial velocity is: $\overrightarrow{v}(0)=$
	\vspace{3pc}
	\item {\bf (1 pt)} The golf ball's velocity vector is $\overrightarrow{v}(t)=$
	\vspace{5pc}
	\item {\bf (1 pt)} The golf ball's position vector is $\overrightarrow{r}(t)=$
	\vspace{5pc}
	\item {\bf (1 pt)} Determine the golf ball's time of flight.
	\vspace{5pc}
	\item {\bf (1 pt)} How far does the golf ball travel?
	\vspace{2pc}
	\item {\bf (1 pt)} What is the maximum height of the golf ball? 
	\end{enumerate}
\end{enumerate}
% % % % %
\end{document}