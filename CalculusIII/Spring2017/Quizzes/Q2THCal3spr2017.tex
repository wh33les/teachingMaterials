\documentclass[%margin%,line,pifont,palatino,courier
]{article}
\usepackage{fullpage}
\usepackage{lastpage}
\usepackage[top=1in,bottom=1in,margin=1in]{geometry}
\usepackage{supertabular}
\usepackage{graphicx,tikz}	
%\usepackage{tkz-euclide}
%\usetkzobj{all}
%\usetikzlibrary{calc}
\usepackage{array,multicol}
\usepackage{amsmath,amssymb}
\usepackage{enumitem}

\usepackage{fancyhdr}
\pagestyle{fancy}

\addtolength{\topmargin}{-0.25in}

\newcommand{\vect}[1]{\mathbf{#1}}
\DeclareMathOperator{\proj}{proj}

\fancypagestyle{plain}{
	\addtolength{\headheight}{0.485in}
	\rhead{\bf MATH 2574 (Calculus III) \\
		%\vspace{0.5pc}
		due Wed 8 Feb 2017 \\}
	\rfoot{\footnotesize $\;$Quiz 2TH, p. \thepage\ (of \pageref{LastPage})
	}
\renewcommand{\headrulewidth}{0pt}
}
\fancyhf{}
\renewcommand{\headrulewidth}{0pt}
\rfoot{\footnotesize Quiz 2TH, p. \thepage\ (of \pageref{LastPage})$\;$}

\title{\vspace{-3.5pc} 
	\flushleft \bf \Large Take-Home Quiz 2: \\ Multivariable functions (\S 12.1-12.2)}
\date{}

% % % % %
\begin{document}
\maketitle

\vspace{-3pc}
\noindent{\bf Directions:} This quiz is due on February 8, 2017 at the beginning of lecture.  You may use whatever resources you like -- e.g., other textbooks, websites, collaboration with classmates -- to complete it \textbf{but YOU MUST DOCUMENT YOUR SOURCES}.  Acceptable documentation is enough information for me to find the source myself.  Rote copying another's work is unacceptable, regardless of whether you document it.  

\noindent\hrulefill

\begin{enumerate}
% % %
\item %{\bf 11.8} 
There is always more than one way to parametrize a given curve.  However, there is one ``correct" way, and that is by the curve's arc length.
\begin{enumerate}
	\item Given a parametrized curve $\vect r(t)=\langle x(t),y(t),z(t)\rangle$, the formula for its length from $t=a$ to $t=b$ is given by
	\[
	s=\int_a^b\sqrt{x'(t)^2+y'(t)^2+z'(t)^2}\ dt.
	\] 
	For each of the following parametrizations of the unit circle, find the arc length.
	\begin{enumerate}
		\item $\vect r_1(t)=\langle \cos t,\sin t\rangle$, $0\leq t\leq 2\pi$.
		%\item $\vect r_2(t)=\langle \cos{2t},\sin{2t}\rangle$, $0\leq t\leq 2\pi$.
		\item $\vect r_2(t)=\langle \cos{2t},\sin{2t}\rangle$, $0\leq t\leq \pi$.
	\end{enumerate}
	\item In part (a), $\vect r_1$ gives the correct parametrization, in the sense that the length of the curve drawn is equal to the length of time ($t$) elapsed.  We say that $\vect r_1$ is \textbf{parametrized by arc length}.  In general, the arc length of a curve $\vect r(t)$  is a function of time $t\geq a$, given by
	\[
	s(t)=\int_a^t\sqrt{x'(u)^2+y'(u)^2+z'(u)^2}\ du.
	\]  
	Find $s(t)$ for each of the parametrizations in part (a).
	\item What is $\dfrac{ds}{dt}$? (Hint: Use the Fundamental Theorem of Calculus.)
	\item Give a condition on $\vect r'(t)$ that guarantees $\vect r(t)$ is parametrized by arc length.
\end{enumerate}

\item %{\bf 12.1 \#90} 
Consider the following plane and curve:
\begin{align*}
y &= 2x+1 \\
\vect r(t) &= \langle 10\cos t,2\sin t, 1 \rangle\text{, for $0\leq t\leq 2\pi$}
\end{align*}  
Find the point(s) where the plane and curve intersect, if any exist.

% % %
\item %{\bf 12.1 \#76} 
Find an equation of the plane passing through $(0,-2,4)$ that is orthogonal to the planes $2x+5y-3z=0$ and $-x+5y+2z=8$.

%\vspace{1.5pc}
% % %
%\hspace{-25pt} 
\item For each of the following quadric surfaces,
\begin{itemize}
	\item Name the surface (see Table 12.1 in the text).
	\item Find the intercepts with the three coordinate axes, when they exist.
	\item Find the equations of the $xy$-, $zx$-, and $yz$-traces, when they exist.
	\item Sketch a graph of the surface.
\end{itemize} 
\begin{enumerate}
	\item %{\bf 12.1 \#64} 
	$4y^2+z^2=x^2$ 
	\item %{\bf 12.1 \#60} 
	$y=\dfrac{x^2}{16}-4z^2$
	\item %{\bf 12.1 \#52} 
	$z=\dfrac{x^2}{4}+\dfrac{y^2}{9}$
\end{enumerate}
%\vspace{1.5pc}

% % %
%\item %{\bf 12.2 \#72} 
%One measurement of the quality of a quarterback in the National Football League is known as the \emph{quarterback passer rating}.  The rating formula is
%\begin{align*}
%R(c,t,i,y) &= \frac{50+20c+80t-100i+100y}{24}\text{, where} \\
%c &= \text{the percentage of passes completed,} \\
%t &= \text{the percentage of passes thrown for touchdowns,} \\
%i &= \text{the percentage of intercepted passes, and} \\
%y &= \text{the average number of yards gained per attempted pass.}
%\end{align*}
%\begin{enumerate}
%	\item In 2012, Green Bay quarterback Aaron Rodgers had the highest passer rating.  He completed 67.21\% of his passes, 7.07\% of his passes were thrown for touchdowns, 1.45\% of his passes were intercepted, and he gained an average of 7.78 yards per passing attempt.  What was his passer rating in the 2012 season?
%	\item If $c$, $t$, and $y$ remain fixed, what happens to the quarterback passer rating as $i$ increases?  Explain your answer with and without mathematics.
%\end{enumerate}

% % %
\item %{\bf 12.2 \#70} 
Suppose you make a one-time deposit of $P$ dollars into a savings account that earns interest at an annual rate of $p\%$ compounded continuously.  The balance in the account after $t$ years is 
\[
B(P,r,t)=Pe^{rt}\text{, where $r=\frac{p}{100}$}
\]
(so for example, if the annual interest rate is 4\%, then $r=0.04$).  Let the interest rate be fixed at $r=0.04$.
\begin{enumerate}
	\item Find the set of all points $(P,t)$ that satisfy $B=2000$.  This curve gives all deposits $P$ and times $t$ that result in a balance of \$2000. 
	\item Repeat part (a) with $B=500,1000,1500,2500$.  Draw the resulting level curves of the balance function.
	\item In general, on one level curve, if $t$ increases, then does $P$ increase or decrease?
\end{enumerate}

% % % % %
\end{enumerate}
\end{document}