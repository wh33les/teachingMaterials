\documentclass[%margin%,line,pifont,palatino,courier
]{article}
\usepackage{fullpage}
\usepackage{lastpage}
\usepackage[top=1in,bottom=1in,margin=1in]{geometry}
\usepackage{supertabular}
\usepackage{graphicx,tikz}	
%\usepackage{tkz-euclide}
%\usetkzobj{all}
%\usetikzlibrary{calc}
\usepackage{array,multicol}
\usepackage{amsmath,amssymb}
\usepackage{enumitem}
\usepackage{framed}

\usepackage{fancyhdr}
\pagestyle{fancy}

\addtolength{\topmargin}{-0.25in}

\newcommand{\vect}[1]{\mathbf{#1}}
\DeclareMathOperator{\proj}{proj}

\fancypagestyle{plain}{
	\addtolength{\headheight}{0.485in}
	\rhead{\bf MATH 2574 (Calculus III) \\
		%\vspace{0.5pc}
		due Fri 3 Mar 2017 \\}
	\rfoot{\footnotesize $\;$Quiz 4TH, p. \thepage\ (of \pageref{LastPage})
	}
\renewcommand{\headrulewidth}{0pt}
}
\fancyhf{}
\renewcommand{\headrulewidth}{0pt}
\rfoot{\footnotesize Quiz 4TH, p. \thepage\ (of \pageref{LastPage})$\;$}

\title{\vspace{-3.5pc} 
	\flushleft \bf \Large Take-Home Quiz 4: \\ Optimization and Lagrange multipliers (\S 12.8-12.9)}
\date{}

% % % % %
\begin{document}
\maketitle

\vspace{-3pc}
\noindent{\bf Directions:} This quiz is due on March 3, 2017 at the beginning of lecture.  You may use whatever resources you like -- e.g., other textbooks, websites, collaboration with classmates -- to complete it \textbf{but YOU MUST DOCUMENT YOUR SOURCES}.  Acceptable documentation is enough information for me to find the source myself.  Rote copying another's work is unacceptable, regardless of whether you document it.  

\noindent\hrulefill

\begin{enumerate}
% % %
\item {\bf 12.8 \#36/ 12.9 \#41}
A lidless box is to be made using 2 m$^2$ of cardboard.  Find the dimensions of the box with the largest possible volume using the following methods:
\begin{enumerate}
\item Use techniques from \S12.8 for optimizing functions of two variables.
\item Use Lagrange multipliers.
\end{enumerate}
%\vspace{1pc}

% % %
\item {\bf 12.8 \#38/ 12.9 \#43}
Find the dimensions of the largest rectangular box in the first octant of the $xyz$-coordinate system that has one vertex at the origin and the opposite vertex on the plane $x+2y+3z=6$, using the following methods:
\begin{enumerate}
\item Use techniques from \S12.8 for optimizing functions of two variables.
\item Use Lagrange multipliers.
\end{enumerate}

% % % % %
\end{enumerate}
\end{document}